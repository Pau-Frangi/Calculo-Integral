\section{Integrales de línea: campos escalares y vectoriales}

\begin{definición}[Camino]
    Un camino (o curva paramétrica) en $\mathbb{R}^n$ es una función continua $\gamma: I \to \R^n$ donde $I \subset \R$ es un intervalo. \\
    Si $\gamma$ es diferenciable en un punto $t \in I$, entonces el vector velocidad de $\gamma$ en el punto (instante) $t$ es el vector tangente a la curva en ese punto, es decir, el vector:
$$\gamma'(t) = (\gamma_1'(t), \ldots, \gamma_n'(t)) \text{ si } \gamma = (\gamma_1, \ldots, \gamma_n)$$
    \end{definición}

    \begin{definición}[Longitud de un camino]
    Sea $\gamma: [a,b] \to \R^n$ un camino en $\R^n$. Sea $\sigma = \{a = t_1 < t_2 < \ldots < t_n = b\}$ partición de $[a,b]$. Definimos $$\Sigma(\gamma, \sigma) = \sum_{i = 1}^{n} \lVert \gamma(t_i) - \gamma(t_{i-1}) \rVert $$
    Definimos entones la longitud de $\gamma$ como: $$L(\gamma) = \sup\{\Sigma(\gamma, \sigma) \ | \  \sigma \text{ es una partición de } [a, b]\} \in [0, + \infty]$$
    Decimos que $\gamma$ es \textbf{rectificable} si $L(\gamma) < + \infty$.
    \end{definición}

    \begin{observación}
    Existen caminos continuos que no son rectificables. Por ejemplo, la curva de Peano, el copo de nieve de Koch o la dada por:
$$l(\gamma) \geq \sum_{n = 1}^{N}\frac{1}{n} \forall n \in \mathbb{N} \text{ luego } l(\gamma) \geq \sum_{n = 1}^{\infty}\frac{1}{n} = + \infty$$
    \end{observación}

    \begin{definición}[Camino $C^1$ a trozos]
    Decimos que un camino $\gamma: [a, b] \to \mathbb{R}^n$ es $C^1$ a trozos si:
$$\exists \mathcal{P} = \{a = t_0 < t_1 < \ldots < t_n = b\}$$ tal que $\gamma|_{[t_{i-1}, t_i]}$ es $C^1$ para todo $i = 1, \ldots, n$.
    \end{definición}

    \begin{observación}
    En cada intervalo $[t_{i-1}, t_i]$ la función $\gamma$ es $C^1$, es decir, en los extremos admite derivadas laterales, aunque puede ocurrir que sean distintas.
    \end{observación}

    \begin{teorema}
        Sea $\gamma: [a, b] \to \mathbb{R}^n$ un camino $C^1$ a trozos. Entonces $\gamma$ es rectificable y su longitud es:
        $$l(\gamma) = \int_{a}^{b} \lVert \gamma'(t) \rVert dt$$
    \end{teorema}

    \begin{observación}
    Tenemos que $t \to  \lVert \gamma'(t) \rVert $ existe, y es continua, salvo quiza en un número finito de puntos, luego en particular es integrable en sentido Riemann y en sentio Lebesgue.
    \\Además, si $\mathcal{P} = \{t_0 = a < t_1 < \ldots < t_n = b\}$ es partición de $[a,b]$ entonces:
    \[
        \int_{a}^{b}  \lVert \gamma'(t) \rVert dt = \sum_{i = 1}^{n} \int_{t_{i-1}}^{t_i} \lVert \gamma'(t) \rVert dt
    \]
    \end{observación}

    Para la demostración del teorema anterior, veamos un lema previo:
    \begin{lema}
        Sea $\gamma: [a, b] \to \mathbb{R}^n$ camino continuo entonces se cumple que:
        $$ \lVert \int_{a}^{b} \gamma(t)dt \rVert  \leq \int_{a}^{b} \lVert \gamma(t) \rVert dt$$
        donde:
        $$\int_{a}^{b} \gamma(t)dt = \left(\int_{a}^{b} \gamma_1(t)dt, \ldots, \int_{a}^{b} \gamma_n(t)dt \right)\in \mathbb{R}^n$$
    \end{lema}
    \begin{proof}
        Hagamos una distinción de casos:
        \begin{itemize}
            \item Si $u = \int_a^b \gamma(t)dt = 0$
            \item Si $u = \int_a^b \gamma(t)dt \neq 0$, sea $u \in \mathbb{R}^n$ con $ \lVert u
                      \rVert = 1 \implies$ $$ \lVert v \rVert = \langle u, v \rangle = \sum_{i =
                          1}^{u} u_i \int_{a}^{b} \gamma_i(t)dt = \int_{a}^{b} \sum_{i = 1}^{n} u_i
                      \gamma_i(t)dt \leq \int_{a}^{b} \lVert \gamma(t) \rVert dt = \lVert
                      \int_{a}^{b} \gamma(t)dt \rVert $$
        \end{itemize}
    \end{proof}

    \begin{proof}
        Veamos ahora la demostración del teorema: \\
        Podemos suponer que $\gamma: [a, b] \to \mathbb{R}^n$ es $C^1$ en casi todo $[a, b]$.
        \begin{enumerate}
            \item Veamos que $l(\gamma) \leq \int_{a}^{b} \lVert \gamma'(t) \rVert dt$: \\ Sea
                  $\mathcal{P} = \{a = t_0 < t_1 < \ldots < t_n = b\}$ partición de $[a, b]$.
                  Entonces: $$\Sigma(\gamma, \mathcal{P}) = \sum_{i = 1}^{n} \lVert \gamma(t_i) -
                      \gamma(t_{i-1}) \rVert = \sum_{i = 1}^{n} \lVert
                      \int_{t_{i-1}}^{t_i}\gamma'(t)dt \rVert \leq \sum_{ i =
                          1}^{n}\int_{y_{i-1}}^{t_i} \lVert \gamma'(t) \rVert dt = \int_{a}^{b} \lVert
                      \gamma'(t) \rVert dt \quad \forall \text{ partición } \mathcal{P}$$ Luego,
                  tomando el supremo de todas las particiones, obtenemos que $l(\gamma) \leq
                      \int_{a}^{b} \lVert \gamma'(t) \rVert dt$
            \item Como $t \to \lVert \gamma'(t) \rVert $ es continua en casi todo $[a,
                              b]$-compacto, luego es uniformemente continua en $[a, b]$.\\ Dado $\epsilon >
                      0, \exists \delta > 0$ tal que si $t,s \in [a, b]$ y $|t-s| < \delta \implies
                      \lVert \gamma'(t) - \gamma'(s) \rVert < \epsilon$\\ Sea $\mathcal{P} = \{a =
                      t_0 < t_1 < \ldots < t_n = b\}$ partición de $[a, b]$ con $t_i - t_{i-1} <
                      \delta \quad \forall i = 1, \ldots, n$\ $$\int_{t_{i-1}}^{t_i} \lVert \gamma'(t)
                      \rVert dt \leq \int_{t_{i-1}}^{t_i} \lVert \gamma'(t_i) \rVert + \epsilon dt =
                      \lVert \gamma(t_i) \rVert \cdot (t_i - t_{i -1}) + \epsilon(t_1 - t_{i-1})$$
                  $$= \lVert \int_{t_{i-1}}^{t_i}\gamma'(t)dt \rVert + \epsilon(t_i - t_{i-1}) =
                      \lVert (\gamma_1'(t_i), \ldots, \gamma_n'(t_i)) \rVert \cdot (t_i - t_{i-1}) +
                      \epsilon(t_i - t_{i-1})$$
                      $$ \leq \int_{t_{i -1}}^{t_i}(\gamma'(t_i) -
                      \gamma'(t)dt)dt +\int_{t_{i-1}}^{t_i} \lVert \gamma'(t) \rVert dt +
                      \epsilon(t_i - t_{i-1})$$ 
                    $$ \leq \int_{t_{i-1}}^{t_i} \lVert \gamma'(t_i) -
                    \gamma'(t) \rVert dt + \int_{t_{i-1}}^{t_i} \lVert \gamma'(t) \rVert dt +
                    \epsilon(t_i - t_{i-1}) \leq 2\epsilon(t_i - t_{i-1}) + \int_{t_{i-1}}^{t_i}
                    \lVert \gamma'(t) \rVert dt$$
                    $$ = 2\epsilon(t_i - t_{i-1})\cdot \lVert \gamma(t_i)
                    - \gamma(t_{i-1}) \rVert $$
                      Luego, $$\int_{a}^{b} \lVert \gamma'(t) \rVert dt =
                      \sum_{i = 1}^n \int_{t_{i-1}}^{t_i} \lVert \gamma'(t) \rVert dt \leq \sum_{i =
                          1}^{n}2\epsilon(t_i - t_{i-1}) + \lVert \gamma(t_i) - \gamma(t_{i-1}) \rVert =
                      2\epsilon(b-a) + \Sigma(\gamma, \mathcal{P}) \leq 2\epsilon(b-a) + l(\gamma)$$
                  Todo esto está sacado del libro de Facenda, Fremiche.
        \end{enumerate}
    \end{proof}

    \ejemplo{
        Sea la curva parametrizada
        \[
            \gamma: [0, 2\pi] \to \mathbb{R}^2, \quad \gamma(t) = (\cos t, \sin t).
        \]

        Además, se cumple que
        \[
            \gamma(0) = (1,0) = p.
        \]
        Derivando, obtenemos
        \[
            \gamma'(t) = (-\sin t, \cos t),
        \]
        y en particular,
        \[
            \gamma'(0) = (0,1) = \vec{v}.
        \]

        \subsection*{Cambio de parámetro}

        Consideremos el cambio de variable $t = 2\pi s$ con $0 \leq s \leq 1$.
        Definiendo la nueva curva
        \[
            \sigma(s) = (\cos(2\pi s), \sin(2\pi s)), \quad 0 \leq s \leq 1,
        \]
        obtenemos su derivada:
        \[
            \sigma'(s) = (2\pi (-\sin(2\pi s)), 2\pi \cos(2\pi s)).
        \]
        En particular, en $s = 0$,
        \[
            \sigma'(0) = 2\pi (0,1) = (0,2\pi).
        \]

        \subsection*{Otro cambio de parámetro}

        Si realizamos el cambio $t = -2\pi s$, obtenemos la curva
        \[
            \alpha(s) = (\cos(2\pi s), -\sin(2\pi s)).
        \]
        Calculamos su derivada:
        \[
            \alpha'(s) = (2\pi (-\sin(2\pi s)), -2\pi \cos(2\pi s)).
        \]
    }

    \begin{definición}
    Sea $\gamma : [a,b] \to \mathbb{R}^n$ camino $C^1$ a trozos y sea $f:Im\gamma \to \mathbb{R}$ continua \\
    Diremos que $f$ es un campo escalar sobre $Im\gamma$ \\
    Definimos:
$$ \int_{\gamma} f = \int_{a}^{b} f(\gamma(t)) \cdot  \lVert \gamma'(t) \rVert dt$$ \\

    Notacion: Podemos denotar $$\int_{\gamma} f = \int_{\gamma} f ds$$ \\ Ademas
$$\l(\gamma) = \int_{a}^{b} \lVert \gamma'(t) \rVert dt = \int_{a}^{b} ds$$
    \end{definición}

    \begin{definición}
    Dos caminos $\gamma: [a,b] \to \mathbb{R}^n$ y $\sigma:[c,d] \to \mathbb{R^n}$ son equivalentes si: \\
$$ \exists h:[c,d] \to [a,b] $$ homeomorfismo $C^1$ que cumple ademas que \\
$$h' \neq 0$$ en $[c,d]$ tal que ademas $$[a,b] \to_{\gamma} \mathbb{R}^n \leftarrow_\sigma [c,d] \leftrightarrow_{h} [a,b]$$ \\
    Tenemos ademas que: $\sigma = \gamma \circ h$ con $\sigma(s) = \gamma(h(s))$ $\forall s \in [c,d]$ \\
    Ahora, por el teorema de Bolzano tenemos dos posibilidades:
    \begin{enumerate}
        \item Si $h'>0$ es decir, $h$ es creciente, decimos que h conserva la orientacion (o
              que $\gamma$ y $\sigma$ tienen la misma orientacion)
        \item Si $h'<0$ es decir, $h$ es decreciente, decimos que $h$ invierte la orientacion
              ($\gamma$ y $\sigma$ tienen orientacion opuesta)
    \end{enumerate}
    \end{definición}

    \begin{observación}
    \vspace{-2.5em}
    \begin{enumerate}
        \item Si $h:[c,d] \to [a,b]$ es biyectiva y $C^1$ con $h' \neq 0$ entonces aplicando
              el Teorema de la funcion inversa obtenemos que $h$ admite inversa local al
              rededor de cada punto. \\ Ademas se cumple que $(h^{-1})'(h(s)) =
                  \frac{1}{h'(s)}$ $\forall s \in [c,d]$ \\ Como ademas $h$ es biyectiva la
              inversa local coincide con la inversa global, luego $h:[c,d] \to [a,b]$ es un
              difeomorfismo $C^1$, es decir, $\exists h^{-1}:[a,b] \to [c,d]$ que es $C^1$
        \item Usando esto obtenemos que la equivalencia de caminos es una relacion de
              equivalencia.
    \end{enumerate}
    \end{observación}

    \begin{observación}
    Si $K\subset \mathbb{R}^n$ compacto y $h:K \to H\subset \mathbb{R^n}$ es continua y biyectiva, entonces $h:K \to H$ es un homeomorfismo.
    \end{observación}

    \begin{proof}
        Tenemos que $h:K \to H$ es biyectiva, luego $\exists h^{-1}:H \to K$, veamos que es biyectiva. \\
        Dado $C \subset K$ cerrado $\implies$ $C$ es compacto $\implies$ $h(C)$ es compacto $\implies$ $(h^{-1})^{-1}(C)=h(C)$ que es compacto en $H$, luego es cerrado en $H$ \\
    \end{proof}

    \begin{teorema}
        Sean $\gamma: [a,b] \to \mathbb{R}^n$ y $\gamma:[c,d] \to \mathbb{R}^n$ caminos $C^1$ a trozos equivalentes. \\
        Sea ademas $f:Im(\gamma)=Im(\sigma) \to \mathbb{R}$ continua, entonces: \\
        $$\int_{\gamma}f=\int_{\sigma}f$$
    \end{teorema}

    \begin{observación}
    Si $\gamma$ y $\sigma$ son equivalentes $\implies Im(\gamma)=Im(\sigma)$
    \end{observación}

    \begin{proof}
        Tenemos $h:[c,d] \to [a,b]$ difeomorfismo $C^1$ con $\gamma \circ h = \sigma$ con ademas $\sigma(s) = \gamma(h(s)) \implies \sigma'(s) = \gamma'(h(s))h'(s)=h'(s)\gamma'(h(s))$ \\
        \begin{enumerate}
            \item Caso 1: $h$ es creciente $(h'>0)$ \\ $$\int_{\gamma} f = \int_{t=a}^{t=b}
                      f(\gamma(t)) \lVert \gamma'(t) \rVert dt = \int_{s=c}^{s=d} f(\gamma(h(s)))
                      \lVert \gamma'(h(s)) \rVert h'(s)ds$$ \\ Haciendo ahora el cambio $t=h(s)$ y
                  $dt=h'(s)ds$ obtenemos: $$\int_{s=c}^{s=d} f(\sigma(h(s))) \lVert \sigma'(s)
                      \rVert ds=\int_{\sigma}f$$
            \item Caso 2: $h$ es decreciente $(h'<0)$\\ Tenemos $$ \int_{\gamma} f=
                      \int_{t=a}^{t=b} f(\gamma(t)) \lVert (\gamma'(t)) \rVert
                      dt=\int_{s=d}^{s=c}f(\gamma(h(s))) \lVert \gamma'(h(s)) \rVert h'(s)ds$$ \\
                  Haciendo ahora el cambio $t=h(s)$ y $dt=h'(s)ds$ obtenemos: $$\int_{s=c}^{s=d}
                      f(\gamma(h(s))) \lVert \gamma'(h(s)) \rVert (-h'(s))ds = \int_{\sigma}f$$
        \end{enumerate}
    \end{proof}

    \begin{corolario}
        Si $\gamma$ y $\sigma$ son equivalentes y $C^1$ a trozos $\implies l(\gamma)=l(\sigma)$
    \end{corolario}

    \begin{proof}
        $$l(\gamma) =\int_{\gamma}1=\int_{a}^{b} \lVert \gamma'(t) \rVert dt=\int_{\sigma}1=l(\sigma)$$
    \end{proof}

    \begin{definición}
    Sea $\gamma:[a,b] \to \mathbb{R^n}$ camino $C^1$ a trozos. Definimos el camino inverso como: \\
$$(-\gamma):[a,b] \to \mathbb{R^n}$$ como $$ (-\gamma)(s)=\gamma(a+b-s)$$
    \end{definición}

    \begin{observación}
    De hecho, $(-\gamma)$ es equivalente a $\gamma$ con $(-\gamma)(s)=\gamma(h(s))$ luego $Im(-\gamma)=Im(\gamma)$
    \end{observación}

    \begin{definición}[Concatenacion de caminos]
    Sean $\gamma:[a,b] \to \mathbb{R^n}$ y $\sigma:[c,d] \to \mathbb{R^n}$ caminos $C^1$ a trozos con $\gamma(b)=\sigma(c)$\\
    Definimos su concatenacion como:\\
$$\gamma + \sigma:[a,b+(d-c)] \to \mathbb{R^n}$$
$$(\gamma + \sigma) = \begin{cases}
    \gamma(t), \text{ si } a \leq t \leq b \\
    \sigma(t - b + c) \text{ si } b\leq t \leq b+(d-c)
\end{cases}$$
\end{definición}

\begin{observación}
En este caso, si
\[
    f : \operatorname{Im}(\gamma_1) \cup \dots \cup \operatorname{Im}(\gamma_m) \longrightarrow \mathbb{R}
\]
es continua en las curvas, entonces se cumple:
\[
    \int_{\gamma_1 + \dots + \gamma_m} f = \sum_{i=1}^{m} \int_{\gamma_i} f
\]
\end{observación}

\ejemplo{

Dado el camino \(\gamma\) definido por:
\[
    \gamma : [0, 2\pi] \to \mathbb{R}^3 \qquad \gamma(t) = (\underbrace{\cos(t)}_{x(t)}, \, \underbrace{\sin(t)}_{y(t)}, \, \underbrace{t}_{z(t)})
\]

Y la funcion \(f : \mathbb{R}^3 \to \mathbb{R}\) dada por:

\[
    f(x,y,z) = x^2 + y^2 + z^2
\]

Entonces, calcular la integral de \(f\) a lo largo de \(\gamma\).

\[
    x^2(t) + y^2(t) = 1 \qquad  \gamma(0) = (1, 0, 0), \quad \gamma(2\pi) = (1, 0, 2\pi)
\]

\[
    \gamma'(t) = (-\sin(t), \, \cos(t), \, 1), \quad \|\gamma'(t)\| = \sqrt{\sin^2(t) + \cos^2(t) + 1} = \sqrt{2}
\]

\[
    \int_{\gamma} f = \int_0^{2\pi} \left( \cos^2(t) + \sin^2(t) + t^2 \right) \sqrt{2} \, dt = \int_0^{2\pi} (1 + t^2) \sqrt{2} \, dt = \left[ t + \frac{t^3}{3} \right]_0^{2\pi} \sqrt{2} = \left( 2\pi + \frac{8\pi^3}{3} \right) \sqrt{2}
\]

}

\subsection{Campos Vectoriales}

\begin{definición} [Campo Vectorial]
Sea $A \subset \mathbb{R}^n$, un campo vectorial continuo en $A$ es una función continua $\vec{F} : A \to \mathbb{R}^n$ que asigna a cada punto $x \in A$ un vector $\vec{F}(x) \in \mathbb{R}^n$.
\end{definición}
\begin{definición} [Integral de un Campo Vectorial a lo largo de un Camino]
Sea $\gamma : [a, b] \to \mathbb{R}^n$ un camino $\mathcal{C}^1$ a trozos y $\vec{F} : \operatorname{Im}(\gamma) \to \mathbb{R}^n$ un campo vectorial continuo. Se define la integral de $\vec{F}$ a lo largo de $\gamma$ como:
\[
    \int_\gamma \vec{F} = \int_a^b \langle \vec{F}(\gamma(t)), \gamma'(t) \rangle dt
\]
\end{definición}

\begin{observación}
El producto escalar $\langle \vec{F}(\gamma(t)), \gamma'(t) \rangle$ representa la proyección ortogonal del vector $\vec{F}(\gamma(t))$ en la dirección de la tangente a $\gamma \text{ en } \gamma(t)$.
\definecolor{ccqqqq}{rgb}{0.8,0,0}
\definecolor{ududff}{rgb}{0.30196078431372547,0.30196078431372547,1}
\definecolor{qqwuqq}{rgb}{0,0.39215686274509803,0}

\begin{center}
    \begin{tikzpicture}[line cap=round,line join=round,>=triangle 45,x=1cm,y=1cm]
        \begin{axis}[
                x=1cm,y=1cm,
                axis lines=middle,
                xmin=-2,
                xmax=3.5,
                ymin=-0.6805806893374612,
                ymax=3.6,
                xtick={-2.5,-2,...,4.5},
                ytick={-0.5,0,...,4.5},
                xticklabels=\empty, % Remove x-axis labels
                yticklabels=\empty % Remove y-axis labels
            ]
            \clip(-2.9040233036492373,-0.6805806893374612) rectangle (4.931567305660256,4.6842052156427245);
            \draw[line width=0.75pt,color=qqwuqq,smooth,samples=100,domain=-2.9040233036492373:4.931567305660256] plot(\x,{sin(((\x))*180/pi)+1.5});
            \draw[line width=0.75pt,color=ccqqqq,smooth,samples=100,domain=-2.9040233036492373:4.931567305660256] plot(\x,{sqrt(3)/2*(\x)+1/2-(3.141592653589793*sqrt(3))/12+1.5});
            \draw [->,line width=0.75pt] (0.5,2) -- (-0.15030877559829892,2.71444);
            \begin{scriptsize}
                \draw[color=qqwuqq] (0.7,1.7) node {$\gamma(t)$};
                \draw [fill=ududff] (0.5,2) circle (1pt);
                \draw[color=ccqqqq] (2.5,3) node {$\gamma'(t)$};
                \draw[color=black] (-0.7, 2.291892925953723) node {$\vec{F}(\gamma(t))$};
            \end{scriptsize}
        \end{axis}
    \end{tikzpicture}
\end{center}

\end{observación}

\textbf{Notación:}\\
Si
\[
    \gamma(t) = (x_1(t), \ldots, x_n(t)) \quad \text{y} \quad \gamma'(t) = (x_1'(t), \ldots, x_n'(t))
\]
entonces:
\[
    \int_\gamma \vec{F} = \int_a^b \langle \vec{F}(x_1(t), \ldots, x_n(t)), (x_1'(t), \ldots, x_n'(t)) \rangle dt
\]
\[
    = \int_a^b \left[ F_1(\gamma(t)) x_1'(t) + \cdots + F_n(\gamma(t)) x_n'(t) \right] dt = \int_\gamma F_1 \, dx_1 + \cdots + F_n \, dx_n
\]
donde $dx_i = x_i'(t) dt$, para $i = 1, \ldots, n$ y $\vec{F} = (F_1, \ldots,
    F_n)$.

\begin{teorema}
    Sean $\gamma : [a, b] \to \mathbb{R}^n$ y $\sigma : [c, d] \to \mathbb{R}^n$ caminos $\mathcal{C}^1$ a trozos y equivalentes, y sea $\vec{F} : \operatorname{Im}(\gamma) = \operatorname{Im}(\sigma) \to \mathbb{R}^n$ un campo vectorial continuo. Entonces:
    \vspace{-0.5em}
    \begin{enumerate}
        \item $\displaystyle \int_\gamma \vec{F} = \int_\sigma \vec{F}$ \quad si $\gamma$ y $\sigma$ tienen la misma orientación.
        \item $\displaystyle \int_\gamma \vec{F} = - \int_\sigma \vec{F}$ \quad si $\gamma$ y $\sigma$ tienen orientación opuesta.
    \end{enumerate}
\end{teorema}

\begin{proof}
    Sabemos que existe $h: [c,d] \to [a,b]$, biyección de clase $C^1$ con $h' \neq 0$, tal que:

    \[
        \begin{tikzcd}
            {[a,b]} \arrow[r, "\gamma"] & \operatorname{Im}(\sigma) \\
            {[c,d]} \arrow[u, "h"] \arrow[ur, "\sigma"']
        \end{tikzcd}
    \]
    Luego
    \[ \sigma'(s) = \gamma'(h(s)) h'(s), \quad \forall s \in [c,d]. \]

    Distinguimos dos casos según la orientación de los caminos:

    \begin{itemize}
        \item \textbf{Caso 1: Misma orientación}\\
              Si $r$ y $\sigma$ tienen la misma orientación, entonces $h' > 0$ (es decir, $h$ es creciente). Se tiene que:
              \[
                  \int_{\gamma} \vec{F} = \int_{t=a}^{t=b} \langle \vec{F} (\gamma (t)), \gamma'(t) \rangle dt = \int_{s=c}^{s=d} \langle \vec{F} (\gamma(h(s))), \gamma'(h(s)) \rangle h'(s) ds
              \]
              \[
                  = \int_{s=c}^{s=d} \langle \vec{F} (\sigma(s)), \sigma'(s) \rangle ds = \int_{\sigma} \vec{F}
              \]
              Donde el cambio de variable viende dado por:
              \[
                  \begin{cases}
                      t = h(s) \\
                      dt = h'(s) ds
                  \end{cases}
              \]
        \item \textbf{Caso 2: Orientación opuesta}\\
              Si $\gamma$ y $\sigma$ tienen orientación opuesta, entonces $h' < 0$ (es decir, $h$ es decreciente). En este caso:
              \[
                  \int_{\gamma} \vec{F} = \int_{t=a}^{t=b} \langle \vec{F} (\gamma(t)), \gamma'(t) \rangle dt = \int_{s=d}^{s=c} \langle \vec{F} (\gamma(h(s))), \gamma'(h(s)) \rangle h'(s) ds
              \]
              \[
                  = -\int_{s=c}^{s=d} \langle \vec{F} (\sigma(s)), \sigma'(s) \rangle ds = -\int_{\sigma} \vec{F}
              \]
    \end{itemize}
\end{proof}

\begin{observación}
Dado una camino continuo $\gamma : [a, b] \to \mathbb{R}^n$ cualesquiera y un campo vectorial continuo $\vec{F} : \operatorname{Im}(\gamma) \to \mathbb{R}^n$, se cumple que:
\vspace{-0.5em}
\begin{enumerate}
    \item $\int_{-\gamma} \vec{F} = -\int_{\gamma} \vec{F}$.
    \item $\int_{\gamma_1+\ldots+\gamma_2} \vec{F} = \sum_{i=1}^{n} \int_{\gamma_i} \vec{F}$.
\end{enumerate}
\end{observación}

\ejemplo{
Un camino puede ser diferenciable (ó $C^1$) y, sin embargo, su imagen puede presentar "picos". Por ejemplo, el camino $\gamma : [-1, 1] \to \mathbb{R}^2$ dado por $\gamma(t) = (t^3, |t^3|)$ es $C^1$ en el intervalo $[-1, 1]$, pero su imagen presenta un pico en el origen. En efecto,

\[
    \gamma'(t) = (\gamma_1'(t), \gamma_2'(t)) \quad \text{con} \quad \gamma_1'(t) = 3t^2 \quad \text{y} \quad \gamma_2'(t) = \begin{cases} 3t^2 & \text{si } t \geq 0 \\ -3t^2 & \text{si } t < 0 \end{cases}
\]

\[
    \gamma_2'(0) = \lim_{t \to 0} \frac{\gamma_2(t) - \gamma_2(0)}{t} = \lim_{t \to 0} \frac{t^2|t|-0}{t} = \lim_{t \to 0} t|t| = 0
\]

Luego $\gamma'(0)$ existe y además $\gamma'(0) = (0, 0)$. Sin embargo, la
imagen de $\gamma$ en el origen presenta un pico, lo que implica que la curva
no es regular en ese punto.

}

\begin{definición} [Camino Simple y Regular]
Diremos que una función \( \gamma: [a,b] \to \mathbb{R}^n \) es un \textbf{camino simple y regular} si:
\vspace{-0.5em}
\begin{itemize}
    \item \( \gamma \) es continua.
    \item \( \gamma \) es inyectiva (simple).
    \item \( \gamma \) es de clase \( C^1 \) en \( [a,b] \) y cumple que \( \gamma'(t) \neq 0 \) para todo \( t \in [a,b] \).
\end{itemize}
\end{definición}

\begin{observación}
\begin{enumerate}
    \vspace{-2.5em}
    \item En este caso, la función \( \gamma: [a,b] \to \operatorname{Im}(\gamma) \) es
          un homeomorfismo sobre su imagen.
    \item Diremos que \( C \subset \mathbb{R}^n \) es una \textbf{curva simple y regular}
          si \( C = \operatorname{Im}(\gamma) \), donde \( \gamma \) es un camino simple
          y regular. En este caso, \( \gamma \) es una \textbf{parametrización simple y
              regular} de \( C \).
\end{enumerate}
\end{observación}

\ejemplo{
    Consideremos la curva:
    \[
        C = \{(x,y) \in \mathbb{R}^2 \mid x^2 + y^2 = 1, \quad y > 0 \}.
    \]

    \begin{center}
        \begin{tikzpicture}
            % Axes
            \draw[->] (-1.5, 0) -- (1.5, 0) node[right]{$x$}; % x-axis
            \draw[->] (0, -0.5) -- (0, 1.5) node[above]{$y$}; % y-axis

            % Upper semicircle (centered at origin)
            \draw (1, 0) arc[start angle=0, end angle=180, radius=1];
            \node at (0, 1) [above right]{$1$};

            % Label
            \node at (1.2, 1.2) {$C$};
        \end{tikzpicture}
    \end{center}

    Una posible parametrización es:
    \[
        \gamma: [0, \pi] \to \mathbb{R}^2  \qquad \gamma(t) = (\cos (t) \sin (t))
    \]
    Su derivada es:
    \[
        \gamma'(t) = (-\sin (t), \cos (t)) \neq (0,0), \quad \forall t \in (0,\pi).
    \]
    Por lo tanto, \( \operatorname{Im}(\gamma) = C \), confirmando que \( \gamma \)
    es una parametrización simple y regular de \( C \). }

\begin{teorema}
    Sea \( C \subset \mathbb{R}^n \) una curva simple y regular y sean \( \gamma \) y \( \sigma \) parametrizaciones simples y regulares de \( C \). Entonces, \( \gamma \) y \( \sigma \) son equivalentes.
\end{teorema}

