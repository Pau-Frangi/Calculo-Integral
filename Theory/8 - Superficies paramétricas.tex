\section{Superficies paramétricas}

\begin{definición} [Superficie Paramétrica]
Una  parametrización de una superficie paramétrica $S$ en $\mathbb{R}^3$ es una aplicación $\varphi: U \to \mathbb{R}^3$ de clase $C^1$ definida en un abierto conexo $U \subset \mathbb{R}^2$ tal que:
$$ Im(\varphi) = \{ \varphi(u,v) \in \mathbb{R}^3 : (u,v) \in U \} = S $$
Diremos que la parametrización $\varphi$ es regular cuando la pareja de vectores $\left\{\frac{\partial \varphi}{\partial u}, \frac{\partial \varphi}{\partial v}\right\}$ es linealmente independiente en todo punto de $U$. Equivalentemente, cuando el vector normal asociado a $\varphi$ es no nulo en todo punto de $U$:
$$ \vec{N}_{\varphi} = \frac{\partial \varphi}{\partial u} \times \frac{\partial \varphi}{\partial v} \neq \vec{0} $$
En este caso, el plano tangente a la superficie en el punto $\varphi(u_0,v_0)$ tiene como ecuaciones paramétricas:
$$
    \begin{cases}
        x = \varphi_1(u_0,v_0) + \lambda \frac{\partial \varphi_1}{\partial u}(u_0,v_0) + \mu \frac{\partial \varphi_1}{\partial v}(u_0,v_0) \\
        y = \varphi_2(u_0,v_0) + \lambda \frac{\partial \varphi_2}{\partial u}(u_0,v_0) + \mu \frac{\partial \varphi_2}{\partial v}(u_0,v_0) \\
        z = \varphi_3(u_0,v_0) + \lambda \frac{\partial \varphi_3}{\partial u}(u_0,v_0) + \mu \frac{\partial \varphi_3}{\partial v}(u_0,v_0) \\
    \end{cases} \qquad \lambda, \mu \in \mathbb{R}
$$
\end{definición}

\ejemplo{
    Dada la superficie $z=x^2+y^2$, podemos parametrizarla con $\varphi:\mathbb{R}^2 \to \mathbb{R}^3$ dada por $\varphi(x,y) = (x,y,x^2+y^2)$. Calculemos el vector normal:
    $$ \vec{N}_{\varphi} = \frac{\partial \varphi}{\partial x} \times \frac{\partial \varphi}{\partial y} =  \begin{vmatrix}
            \vec{e}_1                             & \vec{e}_2                             & \vec{e}_3                             \\
            \frac{\partial \varphi_1}{\partial x} & \frac{\partial \varphi_2}{\partial x} & \frac{\partial \varphi_3}{\partial x} \\
            \frac{\partial \varphi_1}{\partial y} & \frac{\partial \varphi_2}{\partial y} & \frac{\partial \varphi_3}{\partial y} \\
        \end{vmatrix} = \begin{vmatrix}
            \vec{e}_1 & \vec{e}_2 & \vec{e}_3 \\
            1         & 0         & 2x        \\
            0         & 1         & 2y        \\
        \end{vmatrix} = \vec{e}_1 - 2x\vec{e}_3 + 2y\vec{e}_2 \neq (0,0,0)
    $$}

\ejemplo{
    \underline{Superficies explícitas:} Sean $U \subset \mathbb{R}^2$ abierto conexo y $f: U \to \mathbb{R}$ de clase $C^1$. Entonces la gráfica de $f$ es una superficie regular con parametrización $\varphi: U \to \mathbb{R}^3$ dada por $\varphi(x,y) = (x,y,f(x,y))$.\\
    Veamos que $\vec{N}_{\varphi} \neq (0,0,0)$:
    $$ \vec{N}_\varphi = \frac{\partial \varphi}{\partial x} \times \frac{\partial \varphi}{\partial y} = \begin{vmatrix}
            \vec{e}_1                             & \vec{e}_2                             & \vec{e}_3                             \\
            \frac{\partial \varphi_1}{\partial x} & \frac{\partial \varphi_2}{\partial x} & \frac{\partial \varphi_3}{\partial x} \\
            \frac{\partial \varphi_1}{\partial y} & \frac{\partial \varphi_2}{\partial y} & \frac{\partial \varphi_3}{\partial y} \\
        \end{vmatrix} = \begin{vmatrix}
            \vec{e}_1 & \vec{e}_2 & \vec{e}_3                     \\
            1         & 0         & \frac{\partial f}{\partial x} \\
            0         & 1         & \frac{\partial f}{\partial y} \\
        \end{vmatrix} = \vec{e}_1 - \frac{\partial f}{\partial x}\vec{e}_3 + \frac{\partial f}{\partial y}\vec{e}_2 \neq (0,0,0)
    $$
    $$ Im(\varphi) = \{(x,y,z) \in \mathbb{R}^3 : (x,y) \in U, z = f(x,y)\} $$
}

\ejemplo{
    Dado el cilindro de ecuaciones $x^2 + y^2 = 1, \ 0 < z < 1$, buscamos una parametrización de la superficie.\\
    Tomando la siguiente parametrización:
    $$
        \begin{cases}
            x = \cos(\theta) \\
            y = \sin(\theta) \\
            z = z
        \end{cases} \qquad \theta \in \mathbb{R}, \quad z \in (0,1)
    $$
    entonces vemos que $\underbrace{x^2 + y^2}_{1} = r^2 \implies r = 1$.\\
    Por tanto, obtenemos que nuestra parametrización es:
    $$ \varphi : \mathbb{R} \times (0,1) \to \mathbb{R}^3 \quad \varphi(\theta,z) = (\cos(\theta),\sin(\theta),z) $$
    Calculemos el vector normal:
    $$ \vec{N}_{\varphi} = \begin{vmatrix}
            \vec{e}_1                                  & \vec{e}_2                                  & \vec{e}_3                                  \\
            \frac{\partial \varphi_1}{\partial \theta} & \frac{\partial \varphi_2}{\partial \theta} & \frac{\partial \varphi_3}{\partial \theta} \\
            \frac{\partial \varphi_1}{\partial z}      & \frac{\partial \varphi_2}{\partial z}      & \frac{\partial \varphi_3}{\partial z}      \\
        \end{vmatrix} = \begin{vmatrix}
            \vec{e}_1     & \vec{e}_2    & \vec{e}_3 \\
            -\sin(\theta) & \cos(\theta) & 0         \\
            0             & 0            & 1         \\
        \end{vmatrix} = (\cos(\theta),\sin(\theta),0) \neq (0,0,0)
    $$
}

\ejemplo{
    Tomando el cilindro $x^2 + y^2 = 1, \ 0 < z < 1$ del ejemplo anterior, podemos parametrizarlo de otra forma.\\
    Consideramos el siguiente conjunto:
    $$ U = \{ (u,v) : 1 < \sqrt{u^2 + v^2} < 2, \quad 0 < v < 2\pi \} $$
    entonces definimos nuestra parametrización $\varphi: U \to \mathbb{R}^3$ sobre este conjunto tal que
    $$\varphi(u,v) = \left(\frac{u}{\sqrt{u^2 + v^2}}, \frac{v}{\sqrt{u^2 + v^2}}, \sqrt{u^2 + v^2}-1\right) $$
}

\begin{definición}[Equivalencia de parametrizaciones]
    Dieremos que dos parametrizaciones $\varphi: U \to \mathbb{R}^3$ y $\psi: V \to \mathbb{R}^3$ son equivalentes si existe una aplicación $h: V \to U$ difeomorfismo de clase $C^1$ tal que $\psi = \varphi \circ h$.\\
\end{definición}

\begin{observación}
    \begin{enumerate}
        \item En este caso, $\varphi(U) = \psi(V)$.
        \item En la definicion se pide que $U$ y $V$ sean conexos. Como $\forall(s,t) \in V$, $Dh(s,t): \mathbb{R}^2 \to \mathbb{R}^2$ es un isomorfismo lineal, sabemos que $det(Dh(s,t)) \neq 0$. Por conexión $det(Dh(s,t))$ conserva el signo en $V$.
    \end{enumerate}
\end{observación}

\begin{definición}[Conservación de la orientación]
    \begin{enumerate}
        \item Se dice que h conserva la orientacion si $det(Dh(s,t)) > 0, \quad \forall (s,t) \in V$. \\
        Es decir, si $\varphi$ y $\psi$ tienen la misma orientacion.
        \item Se dice que h invierte la orientacion si $det(Dh(s,t)) < 0, \quad \forall (s,t) \in V$. \\
        Es decir, si $\varphi$ y $\psi$ tienen orientacion opuesta.
    \end{enumerate}
\end{definición}

\begin{lema}
    En el caso anterior se tiene que:
    $$\frac{\partial \psi}{\partial s} \times \frac{\partial \psi}{\partial t} (s,t) = det(Dh(s,t)) \cdot \frac{\partial \varphi}{\partial u} \times \frac{\partial \varphi}{\partial v} (h(s,t))$$\\
    Equivalentemente:
    $$ \vec{N}_{\psi}(s,t) = det(Dh(s,t)) \cdot \vec{N}_{\varphi}(h(s,t)) $$\\  
\end{lema}

\begin{proof}
    Ejercicio, como consecuencia dela regla de la cadena: $D\psi(s,t) = D\varphi(h(s,t)) \cdot Dh(s,t)$\\
\end{proof}

\begin{definición}[Vectores normales asociados]
    Asociados a $\varphi$ y $\psi$ obtenemos los vectores normales asociados:
    $$ \vec{n}_{\varphi} = \frac{\vec{N}_{\varphi}}{||\vec{N}_{\varphi}||} \qquad \vec{n}_{\psi} = \frac{\vec{N}_{\psi}}{||\vec{N}_{\psi}||} $$
    Entonces:
    \begin{itemize}
        \item   $\varphi$ y $\psi$ tienen la misma orientacion $\Leftrightarrow$ $\vec{n}_{\psi}(s,t) = \vec{n}_{\varphi}(h(s,t))$\\
    \end{itemize}
    O analogamente:
    \begin{itemize}
        \item   $\varphi$ y $\psi$ tienen orientacion opuesta $\Leftrightarrow \vec{n}_{\psi}(s,t) = -\vec{n}_{\varphi}(h(s,t))$\\
    \end{itemize}
\end{definición}

\begin{definición}[Superficies como conjuntos]
    \begin{enumerate}
        \item Diremos que $S \subset \mathbb{R}^3$ es una superficie simple regular si: $S = \varphi(\overline{D})$, donde $D=int(C)$ con $C \subset \mathbb{R}^2$ curva de Jordan regular a trozos y $\varphi: U \to \mathbb{R}^3$ es una parametrización $C^1$, que es inyectiva y regular en $\overline{D}$.
        \item[1'] En el caso anterior, se dice que la curva $\varphi(C)$ es el borde de $S$. Asi $\Gamma = \varphi(C)$ es una curva cerrada y regular a trozos en $\mathbb{R}^3$.
        \item Diremos que $S \subset \mathbb{R}^3$ es una superficie casi simple regular si $S=\varphi(\overline{D})$, donde $D=int(C)$ con $C \subset \mathbb{R}^2$ curva de Jordan regular a trozos y $\varphi: U \to \mathbb{R}^3$ es una parametrización $C^1$, que es inyectiva y regular en $D$.
    \end{enumerate}
\end{definición}

\begin{definición}[Area de una superficie]
    En los casos anteriores definimos:
    \begin{enumerate}
        \item El area de $S$ es: 
        $$ a(S) = \int_D ||\frac{\partial \varphi}{\partial u} \times \frac{\partial \varphi}{\partial v}|| dudv $$
        En la cercania de un punto $(u_0,v_0)$ ocurre que $\varphi \approx D_\varphi (u_0,v_0)$
        \item Si $f: S \to \mathbb{R}$ es una función continua, entonces la integral de superficie de $f$ sobre $S$ es:
        $$ \int_S f = \int_D f(\varphi(u,v)) ||\frac{\partial \varphi}{\partial u} \times \frac{\partial \varphi}{\partial v}|| dudv $$
    \end{enumerate}
\end{definición}

\ejemplo{
    Calculemos el area de $S = \{(x,y,z): x^2 + y^2 = z^2, \quad 1 \leq z \leq 2\}$.\\
    Usaremos el cambio a cordenadas cilindricas:
    $$ \begin{cases}
            x = r\cos(\theta) \\
            y = r\sin(\theta) \\
            z = z
        \end{cases} \implies
        \begin{cases}
            1 \leq r \leq 2 \\
            0 \leq \theta < 2\pi \\
            r^2 = z^2 \implies r = z
        \end{cases} \implies
        \begin{cases}
            x = z\cos(\theta) \\
            y = z\sin(\theta) \\
            z = z
        \end{cases}
    $$

    Entonces tenemos: $\overline{D}=[0,2\pi] \times [1,2]$, y $S=\varphi(\overline{D}$.\\
    Ahora tenemos el vector normal:
    $$ \vec{N}_{\varphi} = \begin{vmatrix}
            \vec{e}_1      & \vec{e}_2     & \vec{e}_3 \\
            -z\sin(\theta) & z\cos(\theta) & 0         \\
            \cos(\theta)   & \sin(\theta)  & 1         \\
        \end{vmatrix} = (z\cos(\theta), z\sin(\theta), -z)$$
    Cuyo modulo es: 
    $$ ||\vec{N}_{\varphi}||^2 = z^2\cos^2(\theta) + z^2\sin^2(\theta) + z^2 = 2z^2 \implies ||\vec{N}_{\varphi}|| = \sqrt{2}z \neq 0, \quad \forall (\theta,z) \in D$$
    Ademas $\varphi$ es inyectiva y regular en $D$ (aunque no lo es en $\overline{D}$).\\
    Por tanto, el area de $S$ es:
    $$ a(S) = \int_D ||\vec{N}_{\varphi}|| = \int_{\theta = 0}^{\theta = 2\pi} \int_{z=1}^{z=2} \sqrt{2}z dz d\theta = \sqrt{2}\cdot2\pi \left[ \frac{z^2}{2} \right]_{1}^{2} = \sqrt{2}2\pi (4-1) = 3\sqrt{2}\pi $$
}

\ejemplo{
    Sea $f(x,y,z) = x^2 + y^2 + z^2$ y $S$ del ejercicio anterior:
    $$ \int_S f = \int_D f(\varphi(\theta,z)) ||\vec{N}_{\varphi}(\theta,z)|| d\theta dz = \int_{\theta = 0}^{\theta = 2\pi} \int_{z=1}^{z=2} 2z^2 \cdot \sqrt{2}z dz d\theta = 15\sqrt{2}\pi $$
    Ademas: $\int_S fdA = \int_S fdS$
}