\section{Superficies paramétricas}

\begin{definición}
    Una superficie parametrica en $\mathbb{R}^3$  es $\phi: U \to \mathbb{R}^3$ de clase $C^1$, definida en un abierto $U \subset \mathbb{R}^2$.\\
    Diremos que $\phi$ es una parametrización de otra superficie.\\
    Ademas diremos que la parametrización $\phi$ es regular si $ \left\{ \frac{\partial \phi}{\partial u}, \frac{\partial \phi}{\partial v} \right\}$ son linealmente independientes en todo punto de $U$.\\
    Equivalentemente, cuando el vector normal asociado a $\phi$ no es nulo:\\
    $$ \vec{N}_\phi = \frac{\partial \phi}{\partial u} \times \frac{\partial \phi}{\partial v} \neq 0 $$
    En este caso, el plano tangente a la superficie en $\phi(u_0, v_0)$ tiene como ecuaciones:\\
    $$
    \begin{cases}
        x = \phi_1(u_0, v_0) + \lambda \frac{\partial \phi_1}{\partial u}(u_0, v_0) + \mu \frac{\partial \phi_1}{\partial v}(u_0, v_0) \\
        y = \phi_2(u_0, v_0) + \lambda \frac{\partial \phi_2}{\partial u}(u_0, v_0) + \mu \frac{\partial \phi_2}{\partial v}(u_0, v_0) \\
        z = \phi_3(u_0, v_0) + \lambda \frac{\partial \phi_3}{\partial u}(u_0, v_0) + \mu \frac{\partial \phi_3}{\partial v}(u_0, v_0)
    \end{cases}
    $$
\end{definición}

\ejemplo{
    Sea la superficie $z=x^2+y^2$.\\
    Sea la parametrización natural $\varphi: \mathbb{R}^2 \to \mathbb{R}^3$ dada por $\varphi(x,y)=(x,y,x^2+y^2)$.\\
    Entonces el vector normal asociado a $\varphi$ es:
    $$ 
    \vec{N}_\varphi = 
    \begin{vmatrix}
        \vec{e_1} & \vec{e_2} & \vec{e_3} \\
        \frac{\partial \varphi_1}{\partial x} & \frac{\partial \varphi_2}{\partial x} & \frac{\partial \varphi_3}{\partial x} \\
        \frac{\partial \varphi_1}{\partial y} & \frac{\partial \varphi_2}{\partial y} & \frac{\partial \varphi_3}{\partial y} \\
    \end{vmatrix}
    =
    \begin{vmatrix}
        \vec{e_1} & \vec{e_2} & \vec{e_3} \\
        1 & 0 & 2x \\
        0 & 1 & 2y \\
    \end{vmatrix}
    = \left( -2x, -2y, 1 \right) \neq \left( 0, 0, 0 \right)
    $$ 

    Asi el vector normal asociado en $(0,0)$ es $\vec{N}_\varphi(0,0)=(0,0,1)$.
}

\ejemplo{
    Superficies explicitas:\\
    Sean $U \subset \mathbb{R}^2$ y $f: U \to \mathbb{R}$ de clase $C^1$.\\
    Entonces la grafica de $f$ es la una superficie regular con parametrización:\\
    $$ \varphi: U \to \mathbb{R}^3, \quad \varphi(x,y)=(x,y,f(x,y)) $$\\
    Ademas:\\
    $$ \frac{\partial \varphi}{\partial x} \times \frac{\partial \varphi}{\partial y} = 
    \begin{vmatrix}
        \vec{e_1} & \vec{e_2} & \vec{e_3} \\
        1 & 0 & \frac{\partial f}{\partial x} \\
        0 & 1 & \frac{\partial f}{\partial y} \\
    \end{vmatrix}
    = \left( -\frac{\partial f}{\partial x}, -\frac{\partial f}{\partial y}, 1 \right) \neq (0,0,0) $$\\
    Por lo tanto, la grafica de $f$ es una superficie regular.\\
}

\ejemplo{
    Tengamos el cilindro: $$ x^2+y^2=1, \quad 0 < z < 1 $$\\
    Tenemos ahora:\\
    $$ \begin{cases}
        x = r\cos(\theta) \\
        y = r\sin(\theta) \\
        z = z \\
    \end{cases}
    \quad \text{con} \quad 0 < z < 1, \quad \theta \in \mathbb{R}$$\\
    Tenemos que $x^2+y^2=r^2=1 \implies r=1$.\\
    Tomemos ahora:\\
    $$\varphi:U=\mathbb{R} \times (0,1) \to \mathbb{R}^3, \quad \varphi(\theta,z)=(\cos(\theta),\sin(\theta),z)$$\\
    Entonces el vector normal asociado a $\varphi$ es:
    $$ \vec{N}_\varphi =
    \begin{vmatrix}
        \vec{e_1} & \vec{e_2} & \vec{e_3} \\
        -\sin(\theta) & \cos(\theta) & 0 \\
        0 & 0 & 1 \\
    \end{vmatrix}
    = \left( \cos(\theta), \sin(\theta), 0 \right)$$\\ 
}

\ejemplo{
    Sea $U = \left\{ (u,v): 1 < \sqrt{u^2+v^2} < 2 \right\}$ y la parametrización:
    $$ \varphi: U \to \mathbb{R}^3, \quad \varphi(u,v) = \left( \frac{u}{\sqrt{u^2+v^2}}, \frac{v}{\sqrt{u^2+v^2}}, \sqrt{u^2 + v^2} - 1 \right) $$\\
}