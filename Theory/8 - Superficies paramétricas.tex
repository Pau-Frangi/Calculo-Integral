\section{Superficies paramétricas}

\begin{definición} [Superficie Paramétrica]
Una  parametrización de una superficie paramétrica $S$ en $\mathbb{R}^3$ es una aplicación $\varphi: U \to \mathbb{R}^3$ de clase $C^1$ definida en un abierto conexo $U \subset \mathbb{R}^2$ tal que:
$$ Im(\varphi) = \{ \varphi(u,v) \in \mathbb{R}^3 : (u,v) \in U \} = S $$
Diremos que la parametrización $\varphi$ es regular cuando la pareja de vectores $\left\{\frac{\partial \varphi}{\partial u}, \frac{\partial \varphi}{\partial v}\right\}$ es linealmente independiente en todo punto de $U$. Equivalentemente, cuando el vector normal asociado a $\varphi$ es no nulo en todo punto de $U$:
$$ \vec{N}_{\varphi} = \frac{\partial \varphi}{\partial u} \times \frac{\partial \varphi}{\partial v} \neq \vec{0} $$
En este caso, el plano tangente a la superficie en el punto $\varphi(u_0,v_0)$ tiene como ecuaciones paramétricas:
$$
    \begin{cases}
        x = \varphi_1(u_0,v_0) + \lambda \frac{\partial \varphi_1}{\partial u}(u_0,v_0) + \mu \frac{\partial \varphi_1}{\partial v}(u_0,v_0) \\
        y = \varphi_2(u_0,v_0) + \lambda \frac{\partial \varphi_2}{\partial u}(u_0,v_0) + \mu \frac{\partial \varphi_2}{\partial v}(u_0,v_0) \\
        z = \varphi_3(u_0,v_0) + \lambda \frac{\partial \varphi_3}{\partial u}(u_0,v_0) + \mu \frac{\partial \varphi_3}{\partial v}(u_0,v_0) \\
    \end{cases} \qquad \lambda, \mu \in \mathbb{R}
$$
\end{definición}

\ejemplo{
    Dada la superficie $z=x^2+y^2$, podemos parametrizarla con $\varphi:\mathbb{R}^2 \to \mathbb{R}^3$ dada por $\varphi(x,y) = (x,y,x^2+y^2)$. Calculemos el vector normal:
    $$ \vec{N}_{\varphi} = \frac{\partial \varphi}{\partial x} \times \frac{\partial \varphi}{\partial y} =  \begin{vmatrix}
            \vec{e}_1                             & \vec{e}_2                             & \vec{e}_3                             \\
            \frac{\partial \varphi_1}{\partial x} & \frac{\partial \varphi_2}{\partial x} & \frac{\partial \varphi_3}{\partial x} \\
            \frac{\partial \varphi_1}{\partial y} & \frac{\partial \varphi_2}{\partial y} & \frac{\partial \varphi_3}{\partial y} \\
        \end{vmatrix} = \begin{vmatrix}
            \vec{e}_1 & \vec{e}_2 & \vec{e}_3 \\
            1         & 0         & 2x        \\
            0         & 1         & 2y        \\
        \end{vmatrix} = \vec{e}_1 - 2x\vec{e}_3 + 2y\vec{e}_2 \neq (0,0,0)
    $$}

\ejemplo{
    \underline{Superficies explícitas:} Sean $U \subset \mathbb{R}^2$ abierto conexo y $f: U \to \mathbb{R}$ de clase $C^1$. Entonces la gráfica de $f$ es una superficie regular con parametrización $\varphi: U \to \mathbb{R}^3$ dada por $\varphi(x,y) = (x,y,f(x,y))$.\\
    Veamos que $\vec{N}_{\varphi} \neq (0,0,0)$:
    $$ \vec{N}_\varphi = \frac{\partial \varphi}{\partial x} \times \frac{\partial \varphi}{\partial y} = \begin{vmatrix}
            \vec{e}_1                             & \vec{e}_2                             & \vec{e}_3                             \\
            \frac{\partial \varphi_1}{\partial x} & \frac{\partial \varphi_2}{\partial x} & \frac{\partial \varphi_3}{\partial x} \\
            \frac{\partial \varphi_1}{\partial y} & \frac{\partial \varphi_2}{\partial y} & \frac{\partial \varphi_3}{\partial y} \\
        \end{vmatrix} = \begin{vmatrix}
            \vec{e}_1 & \vec{e}_2 & \vec{e}_3                     \\
            1         & 0         & \frac{\partial f}{\partial x} \\
            0         & 1         & \frac{\partial f}{\partial y} \\
        \end{vmatrix} = \vec{e}_1 - \frac{\partial f}{\partial x}\vec{e}_3 + \frac{\partial f}{\partial y}\vec{e}_2 \neq (0,0,0)
    $$
    $$ Im(\varphi) = \{(x,y,z) \in \mathbb{R}^3 : (x,y) \in U, z = f(x,y)\} $$
}

\ejemplo{
    Dado el cilindro de ecuaciones $x^2 + y^2 = 1, \ 0 < z < 1$, buscamos una parametrización de la superficie.\\
    Tomando la siguiente parametrización:
    $$
        \begin{cases}
            x = \cos(\theta) \\
            y = \sin(\theta) \\
            z = z
        \end{cases} \qquad \theta \in \mathbb{R}, \quad z \in (0,1)
    $$
    entonces vemos que $\underbrace{x^2 + y^2}_{1} = r^2 \implies r = 1$.\\
    Por tanto, obtenemos que nuestra parametrización es:
    $$ \varphi : \mathbb{R} \times (0,1) \to \mathbb{R}^3 \quad \varphi(\theta,z) = (\cos(\theta),\sin(\theta),z) $$
    Calculemos el vector normal:
    $$ \vec{N}_{\varphi} = \begin{vmatrix}
            \vec{e}_1                                  & \vec{e}_2                                  & \vec{e}_3                                  \\
            \frac{\partial \varphi_1}{\partial \theta} & \frac{\partial \varphi_2}{\partial \theta} & \frac{\partial \varphi_3}{\partial \theta} \\
            \frac{\partial \varphi_1}{\partial z}      & \frac{\partial \varphi_2}{\partial z}      & \frac{\partial \varphi_3}{\partial z}      \\
        \end{vmatrix} = \begin{vmatrix}
            \vec{e}_1     & \vec{e}_2    & \vec{e}_3 \\
            -\sin(\theta) & \cos(\theta) & 0         \\
            0             & 0            & 1         \\
        \end{vmatrix} = (\cos(\theta),\sin(\theta),0) \neq (0,0,0)
    $$
}

\ejemplo{
    Tomando el cilindro $x^2 + y^2 = 1, \ 0 < z < 1$ del ejemplo anterior, podemos parametrizarlo de otra forma.\\
    Consideramos el siguiente conjunto:
    $$ U = \{ (u,v) : 1 < \sqrt{u^2 + v^2} < 2, \quad 0 < v < 2\pi \} $$
    entonces definimos nuestra parametrización $\varphi: U \to \mathbb{R}^3$ sobre este conjunto tal que
    $$\varphi(u,v) = \left(\frac{u}{\sqrt{u^2 + v^2}}, \frac{v}{\sqrt{u^2 + v^2}}, \sqrt{u^2 + v^2}-1\right) $$
}