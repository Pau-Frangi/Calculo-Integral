\section{Funciones integrables en varias variables}

\subsection{Medibilidad de Funciones}

\begin{definición}[Espacio medible\label{Espacio medible}]
Un espacio medible es un par $(X, \Sigma)$ donde $X$ es un conjunto y $\Sigma$ es una $\sigma$-álgebra de subconjuntos de $X$.
\end{definición}
Vamos a considerar los siguientes espacios medibles:

\begin{itemize}
    \item $(X, \Sigma) = (E, M|_E)$, donde $E \subset \mathbb{R}^n$ es un conjunto medible y $M|_E$ es la familia de subconjuntos medibles de $E$.
    \item $(X, \Sigma) = (A, B|_A)$, donde $A \subset \rn$ es un conjunto boreliano y $B|_A$ es la familia de subconjuntos borelianos de $A$.
\end{itemize}

\begin{definición}[Función medible\label{Función medible}]
Sea $(X, \Sigma)$ un espacio medible. Una función $f: X \to [-\infty, +\infty]$ es medible si para todo $\alpha \in \mathbb{R}$, el conjunto $\{x \in X : f(x) < \alpha\}$ es un conjunto medible.
\end{definición}

\proposicion{
    Sea $(X, \Sigma)$ un espacio medible y $f: X \to [-\infty, +\infty]$, entonces son equivalentes
    \begin{enumerate}
        \item $f$ es medible.
        \item Para todo $\alpha \in \mathbb{R}$, el conjunto $\{x \in X : f(x) \geq \alpha\}$
              es un conjunto medible.
        \item Para todo $\alpha \in \mathbb{R}$, el conjunto $\{x \in X : f(x) > \alpha\}$ es
              un conjunto medible.
        \item Para todo $\alpha \in \mathbb{R}$, el conjunto $\{x \in X : f(x) \leq \alpha\}$
              es un conjunto medible.
        \item Para todo $\alpha, \beta \in \mathbb{R}$, los conjuntos $\{x \in X : \beta \leq
                  f(x) < \alpha\}$, $\{x \in X : f(x) = +\infty\}$ y $\{x \in X : f(x) =
                  -\infty\}$ son conjuntos medibles.
        \item Para todo $G \subset \mathbb{R}$ abierto, los conjuntos $f^{-1}(G)$, $\{x \in X
                  : f(x) = +\infty\}$ y $\{x \in X : f(x) = -\infty\}$ son conjuntos medibles.
    \end{enumerate}
    \label{prop:MedibilidadFunciones}
}

\begin{proof}
    Recordemos que, por definición, $f$ es medible si para todo $a \in \mathbb{R}$ el conjunto $\{x \in X : f(x) < a\}$ es medible.

    Observemos que:
    \[
        \{x \in X : f(x) \geq \alpha\} = X \setminus \{x \in X : f(x) < \alpha\}.
    \]
    Dado que $\Sigma$ es una $\sigma$-álgebra cerrada bajo complementarios, tenemos
    que:
    \[
        (1) \iff (2).
    \]
    De manera análoga:
    \[
        \{x \in X : f(x) > \alpha\} = \bigcup_{k=1}^{\infty} \{x \in X : f(x) \geq \alpha + \tfrac{1}{k}\},
    \]
    y
    \[
        \{x \in X : f(x) \leq \alpha\} = X \setminus \{x \in X : f(x) > \alpha\}.
    \]
    Por lo tanto:
    \[
        (3) \iff (2) \quad \text{y} \quad (4) \iff (3).
    \]
    De este modo, resulta que:
    \[
        (1) \iff (2) \iff (3) \iff (4).
    \]

    Veamos más explícitamente las implicaciones clave:

    \begin{itemize}
        \item $(1) \implies (4)$:
              Para cualquier $\alpha \in \mathbb{R}$,
              \[
                  \{x \in X : f(x) \leq \alpha\} = \bigcap_{k=1}^{\infty} \{x \in X : f(x) < \alpha + \tfrac{1}{k}\}.
              \]
              Cada conjunto de la intersección es medible por la definición de medibilidad de
              $f$. Así, la intersección numerable de conjuntos medibles es medible.

        \item $(4) \implies (1)$:
              Análogamente, para cualquier $\alpha \in \mathbb{R}$,
              \[
                  \{x \in X : f(x) < \alpha\} = \bigcup_{k=1}^{\infty} \{x \in X : f(x) \leq \alpha - \tfrac{1}{k}\}.
              \]
              Cada conjunto en la unión es medible por (4). Por tanto, la unión numerable es
              medible, y $f$ es medible.
    \end{itemize}

    Con esto, hemos establecido la equivalencia entre (1), (2), (3) y (4).

    Ahora, para (5):
    \begin{itemize}
        \item Notemos que para $\alpha, \beta \in \mathbb{R}$,
              \[
                  \{x \in X : \beta \leq f(x) < \alpha\} = \{x \in X : f(x) \geq \beta\} \cap \{x \in X : f(x) < \alpha\}.
              \]
              La intersección de conjuntos medibles es medible por las equivalencias
              anteriores.

        \item Además,
              \[
                  \{x \in X : f(x) = +\infty\} = \bigcap_{k=1}^{\infty} \{x \in X : f(x) > k\},
              \]
              y
              \[
                  \{x \in X : f(x) = -\infty\} = \bigcap_{k=1}^{\infty} \{x \in X : f(x) < -k\},
              \]
              que son intersecciones numerables de conjuntos medibles, por lo que son
              medibles.
    \end{itemize}
    Así, (1) - (4) $\implies$ (5). La construcción de los conjuntos muestra que (5) $\implies$ (2), pues podemos expresar:
    \[
        \{x \in X : f(x) \geq \alpha\} = \bigcup_{m=1}^{\infty} \{x \in X : \alpha \leq f(x) < \alpha + m\} \cup \{x \in X : f(x) = +\infty\}.
    \]
    Por lo tanto:
    \[
        (1) \iff (2) \iff (3) \iff (4) \iff (5).
    \]

    Finalmente, para (6):
    \begin{itemize}
        \item $(6) \implies (5)$:
              Sea $G=(\beta, \alpha)$ un intervalo abierto. Entonces:
              \[
                  f^{-1}(G) = \{x \in X : \beta < f(x) < \alpha\} = \{x \in X : \beta \leq f(x) < \alpha\},
              \]
              que es medible por (5). Además, (6) ya contiene explícitamente que los
              conjuntos $\{x : f(x)=\pm \infty\}$ son medibles.

        \item $(5) \implies (6)$:
              Sea $G \subset \mathbb{R}$ un abierto cualquiera.
              Como $\mathbb{R}$ es segundo contable, $G = \bigcup_{j=1}^{\infty} (\alpha_j, \beta_j)$ para ciertos intervalos abiertos. Así:
              \[
                  f^{-1}(G) = \bigcup_{j=1}^{\infty} f^{-1}((\alpha_j, \beta_j)) = \bigcup_{j=1}^{\infty} \{x \in X : \alpha_j < f(x) < \beta_j\},
              \]
              que es unión numerable de conjuntos medibles por (5). Además, $\{x :
                  f(x)=\pm\infty\}$ son medibles por (5).
    \end{itemize}

    Por tanto:
    \[
        (1) \iff (2) \iff (3) \iff (4) \iff (5) \iff (6).
    \]
\end{proof}

\begin{corolario}
    Sea $E \subset \mathbb{R}^n$ un conjunto medible y $f: E \to \mathbb{R}$ una función continua. Entonces $f$ es medible.
\end{corolario}

\begin{proof}
    Consideremos la $\sigma$-álgebra de Lebesgue en $\mathbb{R}^n$, que denotamos por $\mathcal{L}^n$.
    Sabemos que $E$ es medible, es decir, $E \in \mathcal{L}^n$ y que la restricción de una función continua a un subconjunto es continua en ese subconjunto (con la topología subespacio).

    Para ver que $f$ es medible, recordemos la caracterización equivalente:
    \[
        f \text{ es medible } \iff \forall G \subset \mathbb{R} \text{ abierto}, \quad f^{-1}(G) \in \Sigma.
    \]
    Cabe notar que esta doble implicación es válida porque $f$ es continua y, por
    lo tanto, los conjuntos $$\{x \in X : f(x) = +\infty\} \text{ y }\{x \in X :
        f(x) = -\infty\}$$ son medibles por ser conjuntos vacíos. Aquí $\Sigma$ es la
    $\sigma$-álgebra de Lebesgue de $E$, es decir:
    \[
        \Sigma = \{ A \subset E : \exists B \in \mathcal{L}^n \text{ con } A = E \cap B \}.
    \]
    Tomemos $G \subset \mathbb{R}$ abierto. Dado que $f$ es continua y $G$ es
    abierto en $\mathbb{R}$, se tiene que:
    \[
        f^{-1}(G) = \{ x \in E : f(x) \in G \}
    \]
    es un conjunto abierto en la topología subespacio de $E$. Por lo tanto, existe
    un conjunto abierto $O \subset \mathbb{R}^n$ tal que:
    \[
        f^{-1}(G) = E \cap O.
    \]
    Como $O$ es abierto (y, por tanto, medible en $\mathbb{R}^n$) y $E$ es medible,
    se sigue que:
    \[
        f^{-1}(G) = E \cap O \in \Sigma.
    \]
    Así, $f^{-1}(G)$ es medible para todo abierto $G$, por lo que $f$ es medible
    según la caracterización (6) de la proposición anterior.
\end{proof}

\begin{proposición}
Sea $(X, \Sigma)$ un espacio medible y $f_1, f_2, \dots, f_n \colon X \to \mathbb{R}$ funciones medibles. Si $\Phi \colon \mathbb{R}^n \to \mathbb{R}$ es una función continua, entonces la función compuesta
\[
    h = \Phi \circ (f_1, f_2, \dots, f_n) \colon X \to \mathbb{R}
\]
es medible.
\end{proposición}

\begin{proof}
    Para demostrar que $h$ es medible, probaremos que la preimagen $h^{-1}(G)$ de cualquier abierto $G \subset \mathbb{R}$ pertenece a $\Sigma$.

    Definimos la función vectorial $f \colon X \to \mathbb{R}^n$ dada por
    \[
        f(x) = \big( f_1(x), f_2(x), \dots, f_n(x) \big),
    \]
    de modo que $h = \Phi \circ f$.

    Como cada $f_i \colon X \to \mathbb{R}$ es medible, la función $f$ es medible.
    En efecto, para cualquier rectángulo abierto $R = (a_1, b_1) \times \cdots
        \times (a_n, b_n) \subset \mathbb{R}^n$, se tiene
    \[
        f^{-1}(R) = \bigcap_{i=1}^n f_i^{-1}\big( (a_i, b_i) \big) \in \Sigma,
    \]
    pues cada $f_i^{-1}\big( (a_i, b_i) \big)$ es medible por al apartado (6) de la
    \cref{prop:MedibilidadFunciones} y $\Sigma$ es cerrado bajo intersecciones
    finitas. Dado que los rectángulos abiertos generan la topología de
    $\mathbb{R}^n$, esto implica que $f$ es medible.

    Como $\Phi$ es continua, para cualquier abierto $G \subset \mathbb{R}$, el
    conjunto
    \[
        U = \Phi^{-1}(G) \subset \mathbb{R}^n
    \]
    es abierto en $\mathbb{R}^n$. Por lo tanto, $U$ puede expresarse como una unión
    numerable de rectángulos abiertos (\cref{teoremaCubosAbiertosCerrados}):
    \[
        U = \bigcup_{j=1}^\infty R_j, \quad \text{donde } R_j = \prod_{i=1}^n (a_i^j, b_i^j).
    \]

    La preimagen de $U$ bajo $f$ es
    \[
        h^{-1}(G) = f^{-1}(U) = f^{-1}\left( \bigcup_{j=1}^\infty R_j \right) = \bigcup_{j=1}^\infty f^{-1}(R_j).
    \]
    Como cada $f^{-1}(R_j)$ es medible, y como $\Sigma$ es cerrado bajo uniones
    numerables, se sigue que $h^{-1}(G) \in \Sigma$.

    Hemos demostrado que para todo abierto $G \subset \mathbb{R}$, $h^{-1}(G) \in
        \Sigma$. Por lo tanto, $h$ es una función medible.
\end{proof}

\begin{corolario}
    Sean $(X, \Sigma)$ espacio medible y $f, g: X \to \mathbb{R}$ funciones medibles, entonces $f + g$, $f \circ g$, $max\{f, g\}$, $min\{f, g\}$, $f^+=\max\{f, 0\}$, $f^- = \max\{-f, 0\}$ son todo funciones medibles.
\end{corolario}
\begin{observación}
Dada una función \( f \colon X \to [-\infty,+\infty] \), se puede descomponer como diferencia de sus partes positiva y negativa:
\[
    f = f^+ - f^-, \quad \text{y además} \quad |f| = f^+ + f^-,
\]
donde
\[
    f^+(x) = \max\{f(x), 0\}, \quad f^-(x) = \max\{-f(x), 0\}.
\]
Esta descomposición es útil en integración y teoría de la medida, ya que \(
f^+, f^- \geq 0 \) y son funciones medibles siempre que \( f \) lo sea.
\end{observación}

\begin{teorema}
    Sea $(X, \Sigma)$ un espacio medible y $(f_j)_{j \in \mathbb{N}}: X \to [-\infty, +\infty]$ una sucesión de funciones medibles. Entonces:
    \vspace{-0.5em}
    \begin{enumerate}
        \item $\sup_{j \in \mathbb{N}} f_j$ es una función medible.
        \item $\inf_{j \in \mathbb{N}} f_j$ es una función medible.
        \item $\limsup_{j \to \infty} f_j$ es una función medible.
        \item $\liminf_{j \to \infty} f_j$ es una función medible.
        \item Si $f_j \to f$ puntualmente, entonces $f$ es medible.
    \end{enumerate}
\end{teorema}

\begin{proof}
    \leavevmode
    \begin{enumerate}
        \item Sea $h(x) = \sup_{j \in \mathbb{N}} f_j(x)$. Queremos probar que $h$ es
              medible. Para ello, tomemos $\alpha \in \mathbb{R}$. Observamos que:
              \[
                  \{ x \in X : h(x) > \alpha \} = \{ x \in X : \sup_j f_j(x) > \alpha \} = \bigcup_{j \in \mathbb{N}} \{ x \in X : f_j(x) > \alpha \}
              \]
              Como cada $f_j$ es medible, los conjuntos $\{ f_j > \alpha \}$ son medibles.
              Por lo tanto, su unión numerable también lo es. Así, $h$ es medible.

        \item Sea $g(x) = \inf_{j \in \mathbb{N}} f_j(x)$. Para probar que $g$ es medible,
              tomamos $\alpha \in \mathbb{R}$ y escribimos:
              \[
                  \{ x \in X : g(x) < \alpha \} = \{ x \in X : \inf_j f_j(x) < \alpha \} = \bigcup_{j \in \mathbb{N}} \{ x \in X : f_j(x) < \alpha \}
              \]
              o equivalentemente,
              \[
                  \{ x \in X : g(x) \geq \alpha \} = \bigcap_{j \in \mathbb{N}} \{ x \in X : f_j(x) \geq \alpha \}
              \]
              Como los $f_j$ son medibles, estos conjuntos también lo son, y su intersección
              numerable es medible. Así, $g$ es medible.

        \item Por definición,
              \[
                  \limsup_{j \to \infty} f_j(x) = \lim_{j \to \infty} \sup_{k \geq j} f_k(x) = \inf_{j \in \mathbb{N}} \left( \sup_{k \geq j} f_k(x) \right)
              \]
              Como hemos probado que $\sup_{k \geq j} f_k$ es medible para cada $j$, y el
              ínfimo de funciones medibles es medible, se deduce que $\limsup f_j$ es
              medible.

        \item Análogamente,
              \[
                  \liminf_{j \to \infty} f_j(x) = \lim_{j \to \infty} \inf_{k \geq j} f_k(x) = \sup_{j \in \mathbb{N}} \left( \inf_{k \geq j} f_k(x) \right)
              \]
              Usando la medibilidad de $\inf_{k \geq j} f_k$ y que el supremo de funciones
              medibles es medible, concluimos que $\liminf f_j$ es medible.

        \item Si $f_j(x) \to f(x)$ puntualmente, entonces:
              \[
                  f(x) = \lim_{j \to \infty} f_j(x) = \limsup_{j \to \infty} f_j(x) = \liminf_{j \to \infty} f_j(x)
              \]
              Como $\limsup f_j$ y $\liminf f_j$ son medibles, y coinciden puntualmente con
              $f$, se concluye que $f$ también es medible.
    \end{enumerate}
\end{proof}

\begin{proposición}
Sean $f, g: \mathbb{R}^n \to [+\infty, -\infty]$ funciones medibles-Lebesgue tales que $f = g$ en casi todo punto. Entones $g$ es medible-Lebesgue.
\end{proposición}

\begin{proof}
    Como $f = g$ casi en todo punto, el conjunto
    \[
        Z := \{x \in \mathbb{R}^n : f(x) \neq g(x)\}
    \]
    es un conjunto de medida de Lebesgue nula.

    Sea $\alpha \in \mathbb{R}$. Consideramos el conjunto
    \[
        \{x \in \mathbb{R}^n : g(x) < \alpha\} = \{x \in Z : g(x) < \alpha\} \cup \{x \in Z^c : g(x) < \alpha\}
    \]

    Observamos que en $Z^c$, como $f(x) = g(x)$ para todo $x \in Z^c$, se tiene:
    \[
        \{x \in Z^c : g(x) < \alpha\} = \{x \in Z^c : f(x) < \alpha\} \subseteq \{x \in \mathbb{R}^n : f(x) < \alpha\}
    \]
    el cual es medible porque $f$ es medible.

    Por otro lado, $\{x \in Z : g(x) < \alpha\} \subseteq Z$, y como $Z$ es de
    medida nula, este conjunto también es medible.

    Entonces, la unión
    \[
        \{x \in \mathbb{R}^n : g(x) < \alpha\} = \{x \in Z : g(x) < \alpha\} \cup \{x \in Z^c : f(x) < \alpha\}
    \]
    es la unión de dos conjuntos medibles, por lo tanto es medible.

    Como esto ocurre para todo $\alpha \in \mathbb{R}$, se concluye que $g$ es
    medible.
\end{proof}

\begin{corolario}
    Sea $(f_j)_{j \in \mathbb{N}}: \mathbb{R}^n \to [-\infty, +\infty]$ una sucesión de funciones medibles que converge puntualmente a una función $f$ en casi todo punto. Entonces, $f$ es medible.
\end{corolario}

\begin{proof}
    Sea
    \[
        Z := \left\{x \in \mathbb{R}^n : \lim_{j \to \infty} f_j(x) \text{ no existe o es distinto de } f(x) \right\},
    \]
    el cual, por hipótesis, tiene medida de Lebesgue nula.

    Definimos la función
    \[
        g(x) :=
        \begin{cases}
            \lim_{j \to \infty} f_j(x), & \text{si } x \in Z^c, \\
            0,                          & \text{si } x \in Z.
        \end{cases}
    \]
    Entonces, $g = f$ casi en todo punto (ya que $f_j \to f$ fuera de $Z$), y
    definimos también la sucesión
    \[
        g_j(x) :=
        \begin{cases}
            f_j(x), & \text{si } x \in Z^c, \\
            0,      & \text{si } x \in Z.
        \end{cases}
    \]

    Cada función $g_j$ es medible, ya que se obtiene modificando $f_j$ en un
    conjunto de medida nula. Por tanto, $g_j \to g$ puntualmente, y como el límite
    puntual de funciones medibles es medible, se concluye que $g$ es medible.

    Finalmente, dado que $f = g$ casi en todo punto, y $g$ es medible, la
    proposición anterior garantiza que $f$ también es medible.
\end{proof}

\begin{definición}[Función característica]\label{def:FuncionCaracteristica}
Sea $(X, \Sigma)$ un espacio medible. La función característica de un conjunto $E \in \Sigma$ se define como
\[
    \chi_E(x) =
    \begin{cases}
        1, & \text{si } x \in E,    \\
        0, & \text{si } x \notin E.
    \end{cases}
\]
\end{definición}

\begin{observación}
La función característica $\chi_E$ es medible si y solo si $E \in \Sigma$.
\end{observación}

\begin{proof}
    Sea $G \subset \mathbb{R}$ un conjunto abierto. Como $\chi_E$ solo toma los valores $0$ y $1$, su preimagen $\chi_E^{-1}(G)$ solo puede ser uno de los siguientes conjuntos:
    \[
        \chi_E^{-1}(G) =
        \begin{cases}
            X,         & \text{si } 0 \in G \text{ y } 1 \in G,       \\
            E^c,       & \text{si } 0 \in G \text{ y } 1 \notin G,    \\
            E,         & \text{si } 0 \notin G \text{ y } 1 \in G,    \\
            \emptyset, & \text{si } 0 \notin G \text{ y } 1 \notin G.
        \end{cases}
    \]

    En todos los casos, $\chi_E^{-1}(G)$ es medible si y solo si $E \in \Sigma$, ya
    que tanto $E$ como su complemento $E^c$ deben pertenecer a $\Sigma$. Por tanto,
    $\chi_E$ es medible si y solo si $E$ es medible.
\end{proof}

\begin{observación}
Sean $E \subset \mathbb{R}^{n}$ un conjunto medible y $f: E \to [-\infty, +\infty]$ una función. Entonces, son equivalentes:
\vspace{-0.5em}
\begin{enumerate}
    \item $f$ es medible-Lebesgue.
    \item La función extendida $f \circ \chi_E : \mathbb{R}^n \to [-\infty, +\infty]$
          definida por
          \[
              f \circ \chi_E(x) =
              \begin{cases}
                  f(x), & \text{si } x \in E,    \\
                  0,    & \text{si } x \notin E,
              \end{cases}
          \]
          es medible-Lebesgue.
\end{enumerate}
\end{observación}

\begin{proof}
    \leavevmode
    \begin{itemize}
        \item $(1) \implies (2)$: Como $E^c$ es medible y $f$ es medible en $E$, se tiene que para todo $\alpha \in \mathbb{R}$:
              \[
                  \{x \in \mathbb{R}^n : f \circ \chi_E(x) > \alpha\}
                  = \{x \in E : f(x) > \alpha\} \cup \{x \in E^c : 0 > \alpha\}
              \]
              donde ambos conjuntos son medibles: el primero por hipótesis y el segundo por
              ser subconjunto de $E^c$ (que es medible). Por tanto, la unión es medible y $f
                  \circ \chi_E$ es medible.

        \item $(2) \implies (1)$: Si $f \circ \chi_E$ es medible, entonces para todo $\alpha \in \mathbb{R}$,
              \[
                  \{x \in \mathbb{R}^n : f \circ \chi_E(x) > \alpha\}
                  = \{x \in E : f(x) > \alpha\} \cup \{x \in E^c : 0 > \alpha\}
              \]
              y esta unión es medible por hipótesis. Como $\{x \in E^c : 0 > \alpha\}$ es
              medible y disjunto de $E$, se deduce que $\{x \in E : f(x) > \alpha\}$ es
              medible. Por tanto, $f$ es medible en $E$.
    \end{itemize}
\end{proof}

\begin{definición}[Función simple]\label{Funcion Simple}
Sea $(X, \Sigma)$ un espacio medible y $f: X \to [0, +\infty]$ una función. Decimos que $f$ es una \emph{función simple} si toma un número finito de valores reales no negativos, es decir, si:
\[
    f(X) = \{\alpha_1, \alpha_2, \dots, \alpha_n\} \subset [0, +\infty),
\]
para ciertos $\alpha_i \in [0, +\infty)$.

Definimos los subconjuntos medibles $E_i = f^{-1}(\alpha_i)$ para $i = 1,
    \dots, n$, los cuales forman una partición disjunta de $X$, es decir,
\[
    X = \bigsqcup_{i=1}^{n} E_i.
\]

En tal caso, $f$ puede escribirse como combinación lineal finita de funciones
características:
\[
    f = \sum_{i=1}^{n} \alpha_i \chi_{E_i}.
\]

Por lo tanto, una función simple es aquella que puede expresarse como una suma
finita de múltiplos escalares de funciones características de conjuntos
medibles disjuntos.
\end{definición}

\begin{observación}
Sea \( f: X \to [0, +\infty] \) una función simple tal que \( f = \sum_{i=1}^{n} \alpha_i \chi_{E_i} \). Entonces:

\[
    f \text{ es medible} \quad \iff \quad E_i \in \Sigma \text{ para todo } i = 1, \dots, n.
\]

Es decir, \( f \) es medible si y sólo si los subconjuntos \( E_1, E_2, \dots,
E_n \) son medibles.
\end{observación}

\begin{teorema}
    Sea \((X, \Sigma)\) un espacio medible y \(f: X \to [0, +\infty]\) una función medible. Entonces existe una sucesión de funciones simples \((f_n)_{n \in \mathbb{N}}\) tal que:
    \begin{enumerate}[label=(\roman*)]
        \item \(0 \leq f_1 \leq f_2 \leq \dots \leq f_n \leq \dots \leq f\) para todo \(n\).
        \item Para todo \(x \in X\), se tiene \(\displaystyle \lim_{n \to \infty} f_n(x) =
              f(x)\).
        \item Si además \(f\) es acotada, entonces la convergencia es uniforme casi en todo
              punto.
    \end{enumerate}
\end{teorema}

\begin{proof}
    Para cada \(n \in \mathbb{N}\), consideramos el intervalo \([0, n]\) y lo dividimos en subintervalos de longitud \(\frac{1}{2^n}\). Para \(1 \leq i \leq n 2^n\), definimos los conjuntos
    \[
        E_{n,i} := f^{-1}\left(\left[\frac{i-1}{2^n}, \frac{i}{2^n}\right)\right) = \{x \in X : \tfrac{i-1}{2^n} \leq f(x) < \tfrac{i}{2^n}\}
    \]
    y también
    \[
        E_n := f^{-1}([n, +\infty)) = \{x \in X : f(x) \geq n \}
    \]
    Todos estos conjuntos son medibles, pues \(f\) es medible.

    Definimos entonces la función simple
    \[
        f_n := \sum_{i=1}^{n 2^n} \frac{i-1}{2^n} \chi_{E_{n,i}} + n \chi_{E_n}
    \]

    Veamos primero la convergencia puntual. Fijemos \(x \in X\). Tenemos dos casos:
    \begin{itemize}
        \item Si \(f(x) = +\infty\), entonces para todo \(n \in \mathbb{N}\), \(f(x) \geq
              n\), y por definición \(f_n(x) = n\). Por lo tanto,
              \[
                  \lim_{n \to \infty} f_n(x) = +\infty = f(x)
              \]
        \item Si \(f(x) < +\infty\), existe \(n_x \in \mathbb{N}\) tal que \(f(x) < n_x\).
              Además, existe \(k \in \mathbb{N}\) con
              \[
                  \frac{k-1}{2^{n_x}} \leq f(x) < \frac{k}{2^{n_x}}
              \]
              y entonces
              \[
                  f_{n_x}(x) = \frac{k-1}{2^{n_x}}
              \]
              Por lo tanto,
              \[
                  0 \leq f(x) - f_{n_x}(x) \leq \frac{1}{2^{n_x}}
              \]
              Así,
              \[
                  \lim_{n \to \infty} f_n(x) = f(x)
              \]

              Además, si \(f\) está acotada, es decir, si existe \(M > 0\) tal que \(f(x)
              \leq M\) para todo \(x\), entonces para todo \(n \geq M\) y todo \(x \in X\),
              \[
                  0 \leq f(x) - f_n(x) \leq \frac{1}{2^n}
              \]
              lo que implica que \(f_n \to f\) uniformemente casi en todo punto.
    \end{itemize}

    Ahora, veamos que \(f_n\) es monótona creciente. Para todo \(x \in X\) y todo
    \(n \in \mathbb{N}\), por construcción de \(f_n\), tenemos que:
    \[
        f_n(x) =
        \begin{cases}
            \frac{i-1}{2^n} & \text{si } x \in E_{n,i} \\
            n               & \text{si } x \in E_n
        \end{cases}
    \]
    y
    \[
        f_{n+1}(x) =
        \begin{cases}
            \frac{2i - 2}{2^{n+1}} & \text{si } x \in E_{n,i} \\
            n+1                    & \text{si } x \in E_{n+1}
        \end{cases}
    \]
    Notemos que para todo \(n\) y \(i\),
    \[
        \frac{i-1}{2^n} \leq \frac{2i - 2}{2^{n+1}}
    \]
    por lo que
    \[
        f_n(x) \leq f_{n+1}(x)
    \]
    para todo \(x \in X\) y todo \(n \in \mathbb{N}\).
\end{proof}

\subsection{Integración de Funciones Positivas}

\begin{definición}[Integral de una función simple\label{Integral de una Función Simple}]
Sea $(\mathbb{R}^n, \mathcal{L}^n)$ el espacio medible con la $\sigma$-álgebra $\mathcal{L}^n$ de los conjuntos medibles de $\mathbb{R}^n$ y la medida de Lebesgue \(m\). Sea
\[
    s: \mathbb{R}^n \to [0, +\infty] \qquad s = \sum_{i=1}^n \alpha_i \chi_{A_i}
\]
una función simple, medible y no negativa, donde los conjuntos
\((A_i)_{i=1}^n\) son medibles y forman una unión disjunta que cubre
\(\mathbb{R}^n\), es decir,
\[
    \mathbb{R}^n = \bigsqcup_{i=1}^n A_i
\]

Entonces, definimos la integral de \(s\) respecto a la medida de Lebesgue como:
\[
    \int_{\mathbb{R}^n} s(x) \, dx \;:=\; \sum_{i=1}^n \alpha_i \, m(A_i)
\]
Sea $E \subset \mathbb{R}^n$ un conjunto medible. Definimos la integral de
\(s\) sobre \(E\) como:
\[
    \int_E s(x) \, dx \;:=\; \int_{\mathbb{R}^n} s \cdot \chi_E(x) \, dx = \sum_{i=1}^n \alpha_i \, m(A_i \cap E)
\]
\end{definición}

\begin{observación}
La integral de la función nula sobre \( \mathbb{R}^n \) es cero:
\[
    \int_{\mathbb{R}^n} 0 \, dx = 0.
\]
Del mismo modo, si \( s \) es una función simple y \( E \subset \mathbb{R}^n \)
tiene medida nula, entonces:
\[
    \int_E s(x) \, dx = 0.
\]
Esto se deduce directamente de la definición, ya que \( m(A_i \cap E) = 0 \)
para todo \( i \).
\end{observación}

\begin{lema}
    Sea $\mathbb{R}^n = \bigsqcup_{k = 1}^{\infty} X_k$ una unión disjunta de conjuntos medibles. Sea $s: \mathbb{R}^n \to [0, +\infty]$ una función simple, medible y no negativa. Entonces,
    \[
        \int_{\mathbb{R}^n} s \, dx = \sum_{k = 1}^{\infty} \int_{X_k} s \, dx.
    \]
    \label{intSimpleDisjunta}
\end{lema}

\begin{proof}
    Sea la representación canónica de $s$:
    \[
        s = \sum_{i=1}^m \alpha_i \chi_{A_i}, \quad \text{con } A_i \subset \mathbb{R}^n \text{ medibles y disjuntos.}
    \]

    Entonces, usando la definición de la integral de funciones simples:
    \[
        \int_{\mathbb{R}^n} s \, dx = \sum_{i=1}^m \alpha_i \cdot m(A_i).
    \]

    Ahora, para cada \( i \), tenemos que:
    \[
        A_i = \bigsqcup_{k=1}^{\infty} (A_i \cap X_k),
    \]
    donde la unión es disjunta porque los \( X_k \) son disjuntos. Como la medida
    es aditiva sobre uniones disjuntas (\cref{unionNumerableMedible}):
    \[
        m(A_i) = \sum_{k=1}^{\infty} m(A_i \cap X_k).
    \]

    Por tanto:
    \[
        \int_{\mathbb{R}^n} s \, dx = \sum_{i=1}^m \alpha_i \sum_{k=1}^{\infty} m(A_i \cap X_k)
        = \sum_{k=1}^{\infty} \sum_{i=1}^m \alpha_i \cdot m(A_i \cap X_k).
    \]

    Pero la expresión
    \[
        \sum_{i=1}^m \alpha_i \cdot m(A_i \cap X_k)
    \]
    es, por la \cref{Integral de una Función Simple}, \( \int_{X_k} s \, dx \). Por
    lo tanto,
    \[
        \int_{\mathbb{R}^n} s \, dx = \sum_{k=1}^{\infty} \int_{X_k} s \, dx.
    \]
\end{proof}

\begin{corolario}
    Sean \( s, t: \mathbb{R}^n \to [0, +\infty] \) funciones simples, medibles y no negativas. Entonces,
    \[
        \int_{\mathbb{R}^n} (s + t) \, dx = \int_{\mathbb{R}^n} s \, dx + \int_{\mathbb{R}^n} t \, dx.
    \]
    \label{sumaIntSimples}
\end{corolario}

\begin{proof}
    Supongamos que las representaciones canónicas de \( s \) y \( t \) son
    \[
        s = \sum_{i=1}^{m} \alpha_i \, \chi_{A_i}, \quad
        t = \sum_{j=1}^{k} \beta_j \, \chi_{B_j},
    \]
    donde los conjuntos \( A_i \) y \( B_j \) son medibles y forman particiones
    disjuntas de \( \mathbb{R}^n \), es decir,
    \[
        \mathbb{R}^n = \bigsqcup_{i=1}^{m} A_i = \bigsqcup_{j=1}^{k} B_j
    \]
    Entonces,
    \[
        s + t = \sum_{i=1}^{m} \sum_{j=1}^{k} (\alpha_i + \beta_j) \cdot \chi_{A_i \cap B_j},
    \]
    ya que en cada conjunto \( A_i \cap B_j \), se tiene \( s(x) = \alpha_i \), \(
    t(x) = \beta_j \), y por tanto \( (s + t)(x) = \alpha_i + \beta_j \)

    Como la familia \( \{ A_i \cap B_j \} \) es una partición disjunta de \(
    \mathbb{R}^n \), y los conjuntos son medibles, aplicamos la definición de
    integral de función simple:
    \[
        \int_{\mathbb{R}^n}(s + t) = \sum_{i=1}^{m} \sum_{j=1}^{k} (\alpha_i + \beta_j) \cdot m(A_i \cap B_j)
    \]

    Distribuyendo:
    \[
        = \sum_{i=1}^{m} \sum_{j=1}^{k} \alpha_i \cdot m(A_i \cap B_j)
        + \sum_{i=1}^{m} \sum_{j=1}^{k} \beta_j \cdot m(A_i \cap B_j)
    \]

    Reordenando las sumas:
    \[
        = \sum_{j=1}^{k} \left( \sum_{i=1}^{m} \alpha_i \cdot m(A_i \cap B_j) \right)
        + \sum_{i=1}^{m} \left( \sum_{j=1}^{k} \beta_j \cdot m(A_i \cap B_j) \right)
    \]

    Pero por el lema de aditividad de la integral sobre uniones disjuntas que
    particionan el espacio (\cref{intSimpleDisjunta}), esto es:
    \[
        = \sum_{j=1}^{k} \int_{B_j} s + \sum_{i=1}^{m} \int_{A_i} t
        = \int_{\mathbb{R}^n} s + \int_{\mathbb{R}^n} t
    \]
\end{proof}

\begin{definición}[Integral de Lebesgue]
Sea $f: \mathbb{R}^n \to [0, +\infty)$ una función medible y no-negativa. Definimos la integral de Lebesgue de $f$ en $\mathbb{R}^n$ como:
$$\int_{\mathbb{R}^n} f := \sup \left\{ \int_{\mathbb{R}^n} s \ \middle| \ s \text{ es simple, medible y } 0 \leq s \leq f \right\}$$
Si $E \subset \mathbb{R}^n$ es medible y $f: E \to [0, +\infty)$ es una función medible y no-negativa, definimos la integral de Lebesgue de $f$ sobre $E$ como:
$$\int_E f := \sup \left\{ \int_{\mathbb{R}^n} s \cdot \chi_E \ \middle| \ s \text{ es simple, medible y } 0 \leq s \leq f \cdot \chi_E \right\}$$\label{Integral de Lebesgue}
\end{definición}

\begin{proposición}
Para funciones medibles, no-negativas y conjuntos medibles se tiene que:
\vspace{-0.5em}
\begin{enumerate}
    \item Si $0 \leq f \leq g$ y $E \subset \mathbb{R}^n$ es medible $\implies \int_E f
              \leq \int_F g$
    \item Si $f, g \geq 0 \implies \int_E (f + g) = \int_E f + \int_E g$
    \item Si $c \geq 0, \ f \geq 0 \implies \int_E cf = c \int_E f$
    \item Si $m(E) = 0 \implies \int_E f = 0$ (Incluso si $f = +\infty$)
    \item Si $f\big|_E = 0 \implies \int_E f = 0$ (Incluso si $m(E) = +\infty$)
    \item Si $A \subset B$ y $f \geq 0 \implies \int_A f \leq \int_B f$
    \item Si $A, B$ son conjuntos medibles y disjuntos y $f\geq 0 \implies \int_{A \cup
                  B} f = \int_A f + \int_B f$
    \item Si $f = g$ en casi todo punto de $E \subset \mathbb{R}^n$ medible $\implies \int_E f = \int_E g$
    \item [1'.] Si $0 \leq f \leq g$ en casi todo punto de $E \subset \mathbb{R}^n$ medible, entonces $\int_E f \leq \int_E g$. 
\end{enumerate}
\label{prop:PropiedadesIntegralesLebesgue}
\end{proposición}
\begin{proof}
    \leavevmode
    \begin{enumerate}
        \item  Sean $f, g \colon \mathbb{R}^n \to [0, \infty]$ funciones medibles tales que
              $f(x) \leq g(x)$ para todo $x \in \mathbb{R}^n$, y sea $E \subset \mathbb{R}^n$
              un conjunto medible. Definimos los siguientes conjuntos de funciones simples:

              \[
                  \mathcal{S}_f = \left\{ s \text{ simple y medible} \ \middle| \ 0 \leq s \leq f \cdot \chi_E \right\} \quad
                  \mathcal{S}_g = \left\{ t \text{ simple y medible} \ \middle| \ 0 \leq t \leq g \cdot \chi_E \right\}
              \]

              Como $f(x) \leq g(x)$ para todo $x$, se tiene que
              \[
                  f(x)\chi_E(x) \leq g(x)\chi_E(x), \quad \forall x \in \mathbb{R}^n
              \]
              Por lo tanto, cualquier función simple $s \in \mathcal{S}_f$ también satisface
              $0 \leq s \leq g \cdot \chi_E$, es decir,
              \[
                  \mathcal{S}_f \subset \mathcal{S}_g
              \]

              La integral de Lebesgue sobre $E$ se define como:
              \[
                  \int_E f = \sup_{s \in \mathcal{S}_f} \int_{\mathbb{R}^n} s, \quad
                  \int_E g = \sup_{t \in \mathcal{S}_g} \int_{\mathbb{R}^n} t
              \]

              Como el supremo de un conjunto está acotado por el supremo de un conjunto que
              lo contiene, se concluye que:
              \[
                  \int_E f \leq \int_E g
              \]

        \item  Véase la demostración del \cref{cor:TCM}.

        \item Si $f = c \cdot 0$, entonces es trivial. Si $c > 0$, tomamos $s =
                  \sum_{i=1}^{m} \alpha_i \cdot \chi_{A_i}$, con $0 \leq s \leq f$.

              Entonces, $c \cdot s = \sum_{i=1}^{m} c \cdot \alpha_i \cdot \chi_{A_i}$, con
              $0 \leq c \cdot s \leq c \cdot f$.

              Así, $$ \int_{\mathbb{R}^n} c \cdot s = \sum_{i=1}^{m} c \cdot \alpha_i \cdot
                  m(A_i) = c \sum_{i=1}^{m} \alpha_i \cdot m(A_i) = c \int_{\mathbb{R}^n} s $$

              Tomando el supremo, obtenemos $$ \int_{\mathbb{R}^n} c \cdot f = c \sup \left\{
                  \int_{\mathbb{R}^n} s \ \middle| \ s \text{ es simple, medible y } 0 \leq s
                  \leq f \right\} = c \int_{\mathbb{R}^n} f $$

        \item Dado $E \subset \mathbb{R}^n$ un conjunto medible tal que $m(E) = 0$, y sea $s
                  = \sum_{i=1}^m \alpha_i \cdot \chi_{A_i}$ una función simple tal que $0 \leq s
                  \leq f \cdot \chi_E$. Entonces para todo $i = 1, \dots, m$ se tiene que $m(A_i
                  \cap E) = 0$ por ser $E$ de medida nula. Por lo tanto,
              \[
                  \int_E s = \int_{\mathbb{R}^n} s \cdot \chi_E = \sum_{i=1}^m \alpha_i \cdot m(A_i \cap E) = 0
              \]

              Esto es cierto para toda función simple $s$ tal que $0 \leq s \leq f \cdot
                  \chi_E$. Por lo tanto, el supremo de dichas integrales también es cero:
              \[
                  \int_E f = \sup \left\{ \int_E s \ \middle| \ s \text{ es simple, medible y } 0 \leq s \leq f \cdot \chi_E
                  \right\} = 0
              \]

        \item Supongamos que \( f|_E = 0 \), es decir, \( f(x) = 0 \) en casi todo punto de
              \( E \).

              Sea \( s = \sum_{i=1}^{m} \alpha_i \cdot \chi_{A_i} \) una función simple y
              medible tal que \( 0 \leq s \leq f \cdot \chi_E \). Por definición de
              restricción, se tiene que
              \[
                  s(x) \leq f(x) = 0 \quad \text{para casi todo } x \in E
              \]
              por lo que
              \[
                  s(x) = 0 \quad \text{para casi todo } x \in E
              \]
              Luego $\forall i = 1, \dots, m$ tal que $\alpha_i > 0$ se tiene que $m(A_i \cap
                  E) = 0$

              entonces,
              \[
                  \int_E s = \int_{\mathbb{R}^n} s \cdot \chi_E = \sum_{i=1}^{m} \alpha_i \cdot m(A_i \cap E) = 0
              \]

              Finalmente, buscando el supremo de las funciones simples obtenemos:
              \[
                  \int_E f = \sup \left\{ \int_E s \ \middle| \ s \text{ es simple, medible y } 0 \leq s \leq f \cdot \chi_E
                  \right\} = 0
              \]

        \item Dados $A \subset B \subset \mathbb{R}^n$ conjuntos medibles.

              Definimos las funciones:
              \[
                  f_1 := f \cdot \chi_A, \quad f_2 := f \cdot \chi_B
              \]

              Como $f \geq 0$ y $\chi_A(x) \leq \chi_B(x)$ para todo $x \in \mathbb{R}^n$
              (porque $A \subset B$), se tiene que:
              \[
                  f_1(x) = f(x)\chi_A(x) \leq f(x)\chi_B(x) = f_2(x), \quad \forall x \in \mathbb{R}^n
              \]

              Es decir, $f_1 \leq f_2$ en todo $\mathbb{R}^n$.

              Entonces, por el apartado (1), se sigue que:
              \[
                  \int_{\mathbb{R}^n} f_1 \leq \int_{\mathbb{R}^n} f_2
              \]

              Como $\int_A f = \int_{\mathbb{R}^n} f \cdot \chi_A = \int_{\mathbb{R}^n} f_1$
              y $\int_B f = \int_{\mathbb{R}^n} f \cdot \chi_B = \int_{\mathbb{R}^n} f_2$,
              concluimos que:
              \[
                  \int_A f \leq \int_B f
              \]

        \item  Si $A, B$ son medibles y disjuntos, entonces $$ \chi_{A \cup B} = \chi_A +
                  \chi_B $$

              Así, $$ \int_{A \cup B} f = \int_{\mathbb{R}^n} f \cdot \chi_{A \cup B} =
                  \int_{\mathbb{R}^n} f (\chi_A + \chi_B) = \int_{\mathbb{R}^n} (f \chi_A + f
                  \chi_B) $$

              Por el apartado (2) se sigue, $$ = \int_{\mathbb{R}^n} f \chi_A +
                  \int_{\mathbb{R}^n} f \chi_B = \int_A f + \int_B f $$

              Por lo tanto, $$ \int_{A \cup B} f = \int_A f + \int_B f $$

        \item Tenemos $f, g : \mathbb{R}^n \to [0, \infty]$ funciones medibles tales que
              $f(x) = g(x)$ para casi todo $x \in E$. Es decir, existe un conjunto $Z \subset
                  E$ tal que $m(Z) = 0$ y $f(x) = g(x)$ para todo $x \in E \setminus Z$.

              Definimos $A := E \setminus Z$, de modo que $E = A \cup Z$ con $A \cap Z =
                  \emptyset$ y $m(Z) = 0$.

              Entonces por (7), podemos descomponer la integral como:
              \[
                  \int_E f = \int_A f + \int_Z f \quad \text{y} \quad \int_E g = \int_A g + \int_Z g.
              \]

              Pero sobre $A$, se cumple que $f = g$, por lo que:
              \[
                  \int_A f = \int_A g.
              \]

              Además, como $m(Z) = 0$, se sigue por el apartado (4) que:
              \[
                  \int_Z f = \int_Z g = 0.
              \]

              Por tanto,
              \[
                  \int_E f = \int_A f + \int_Z f = \int_A f + 0 = \int_A g + 0 = \int_A g + \int_Z g = \int_E g.
              \]
        \item [1'.] Sea $Z = \{x \in E \mid f(x) > g(x)\}$ el conjunto excepcional donde no se cumple la desigualdad. Por hipótesis, $m(Z) = 0$.
    
        Definamos la función $h = g - f \geq 0$ en $E \setminus Z$ y $h = 0$ en $Z$. Entonces:

        \begin{itemize}
            \item $h$ es medible pues es suma/resta de funciones medibles en $E \setminus Z$ y es cero en $Z$.
            \item $h \geq 0$ en todo $E$ por construcción.
        \end{itemize}

        Entonces $g = f + h$ en casi todo punto de $E$ (excepto en $Z$ que tiene medida nula). Luego aplicando primero (8) y después (2) se tiene que:

        \[
            \int_E g = \int_E (f + h) = \int_E f + \int_E h
        \]

        Como $h \geq 0$, por la propiedad (1) es evidente que:
        \[ \int_E h \geq 0 \]
          
        Por lo tanto:
        \[ \int_E g = \int_E f + \int_E h \geq \int_E f \]

    \end{enumerate}
\end{proof}

\begin{teorema} [Convergencia Monótona\label{TCM}]
    Sea $(f_k)_{k \in \mathbb{N}} : \mathbb{R}^n \to [0, +\infty]$ una sucesión de funciones medibles tales que:
    \begin{enumerate}
        \item $f_1(x) \leq f_2(x) \leq \dots$. (en $\mathbb{R}^n$)
        \item $\lim_{k \to \infty} f_k = f$ (puntualmente en $\mathbb{R}^n$)
    \end{enumerate}
    Entonces se cumple que:
    $$\lim_{k \to \infty} \int_{\mathbb{R}^n} f_k = \int_{\mathbb{R}^n} f.$$
\end{teorema}
\begin{proof}
    La sucesión $(f_k)_{k \in \mathbb{N}}$ es monótona creciente en $[0, +\infty)$.
    Por lo tanto, existe el límite:
    $$ l = \lim\limits_{k \to \infty} f_k, \in [0, +\infty].$$
    Dado que $f_k(x) \leq f(x) \quad \forall x \in \mathbb{R}^n$, tenemos que:
    $$\int_{\mathbb{R}^n} f_k \leq \int_{\mathbb{R}^n} f.$$ Queda demostrar la otra desigualdad para probar el teorema.\\
    Sea $s$ una función simple y medible en $\mathbb{R}^n$ con $0 \leq s \leq f$, y fijemos un $c \in (0, 1)$.
    $\forall k \in \mathbb{N}$, definimos la sucesión de conjuntos $$E_k = \{ x \in \mathbb{R}^n : f_k(x) \geq c \cdot s(x) \}$$ Esta sucesión es medible (debido a que tanto $f_k$ como $s$ son medibles) y es creciente (debido a que $f_k \leq f_{k+1}$ y $c \cdot s \leq c \cdot f \leq f$).
    Ahora veamos que:
    $$
        \bigcup\limits_{k=1}^{\infty} E_k = \mathbb{R}^n.
    $$
    Sea $x \in \mathbb{R}^n$. Entonces,
    \[
        \begin{cases}
            \text{Si } f(x) = 0 \implies f_k(x) = c \cdot s(x) = 0 \implies x \in E_k \quad \forall k \\
            \text{Si } f(x) > 0 \implies \exists k \in \mathbb{N} : c \cdot s(x) \leq f_k(x) \leq f(x) \implies x \in E_k
        \end{cases}
    \]
    Por lo tanto, $x \in E_k$. Veamos que: $$\int_{\mathbb{R}^n} s = \lim\limits_{k
            \to \infty} \int_{E_k} s.$$ Dado que $s = \sum_{j=1}^{m} \alpha_j \cdot
        \chi_{A_j}$ con $s^{-1}(\alpha_j) = A_j$ y $(E_k)_{k \in \mathbb{N}}$ es una
    sucesión creciente, entonces, para cada $j = 1, \ldots, m$, tenemos por el
    \cref{limUnionMedible}: $$ m(A_j) = m \left(\bigcup_{k = 1}^{\infty}(E_k\cap
            A_j)\right) = \lim\limits_{k \to \infty} m(E_k \cap A_j). $$
    Luego:$$\int_{\mathbb{R}^n} s = \sum_{j=1}^{m} \alpha_j \cdot m(A_j) =
        \sum_{j=1}^{m} \alpha_j \cdot \lim\limits_{k \to \infty} m(E_k \cap A_j) =
        \lim\limits_{k \to \infty} \sum_{j=1}^{m} \alpha_j \cdot m(E_k \cap A_j) =
        \lim\limits_{k \to \infty} \int_{E_k} s$$ Finalmente, obtenemos que: $$
        \int_{\mathbb{R}^n}f_k \geq \int_{E_k} f_k \geq \int_{E_k} c \cdot s = c \cdot
        \int_{E_k} s$$ Tomando límites el límite cuando $k \to \infty$, obtenemos que:
    $$ l \geq c \cdot \int_{\mathbb{R}^n} s$$ Por último, si tomamos el límite $c
        \to 1$ obtenemos que: $$ l \geq \int_{\mathbb{R}^n} s$$ Dado que $s$ es una
    función simple y medible arbitraria, se tiene esta propiedad $\forall s$
    función simple, medible y no-negativa (por ser $0 \leq s \leq f$). Por tanto,
    obtenemos la desigualdad buscada: $l \geq \int_{\mathbb{R}^n} f$.
\end{proof}
\begin{teorema} [Convergencia Monótona Versión Refinada\label{TCM Refinada}]
    Sea $E \subset \mathbb{R}^n$ medible y $f_k: E \to [0, +\infty]$ sucesión de funcion medibles y $f: E \to [0, +\infty]$ tales que:
    \begin{enumerate}
        \item $f_1(x) \leq f_2(x) \leq \dots$ (en casi todo punto de $E$)
        \item $\lim_{k \to \infty}f_k = f$ (en casi todo punto de $E$)
    \end{enumerate}
    Entonces se cumple que: $$\lim_{k \to \infty}\int_{E}f_k = \int_{E}f.$$
\end{teorema}
\begin{proof}
    Denotamos el conjunto $$ N = \{ x \in E \mid (1) \text{ y } (2) \text{ no se cumplen} \} $$
    Sabemos que \( m(N) = 0 \). Definimos la sucesión de funciones $$ \hat{f}_k = f_k \cdot \chi_{E \setminus  N}, \quad \forall k \in \mathbb{N} \text{ y } \hat{f} = f \cdot \chi_{E\setminus N}$$
    Podemos aplicar el \cref{TCM}, lo que nos permite concluir que:
    1. \( \hat{f}_k \to f \) puntualmente.
    2. Se cumple la convergencia de integrales.
    Por lo tanto, tomando límites en la integral:
    $$ \int_E f = \int_{E \setminus N} f = \int_{\mathbb{R}^n}\hat{f} = \lim_{k \to \infty} \int_{\mathbb{R}^n} \hat{f}_k = \lim_{k \to \infty} \int_E f_k. $$
\end{proof}
\begin{corolario}
    \vspace{-2.5em}
    \begin{enumerate}
        \item Si $f, g: \mathbb{R}^n \to [0, +\infty]$ son medibles y no-negativas se tiene
              que: $$\int_{\mathbb{R}^n}f+g = \int_{\mathbb{R}^n}f + \int_{\mathbb{R}^n}g$$.
        \item Si $(f_k)_{k\in \mathbb{N}}: \mathbb{R} \to [0, +\infty]$ sucesión de funciones
              mediles $\forall k \in \mathbb{N}$ se tiene que:
              $$\int_{E}\sum_{k=1}^{\infty}f_k = \sum_{k=1}^{\infty}\int_{E}f_k$$.
    \end{enumerate}
    \label{cor:TCM}
\end{corolario}

\begin{proof}
    \leavevmode
    \begin{enumerate}
        \item Dado que $f$ y $g$ son funciones medibles y no negativas, existen sucesiones crecientes de funciones simples y medibles $(s_j)_{j \in \mathbb{N}}$ y $(t_j)_{j \in \mathbb{N}}$ tales que
        \[
            s_j \uparrow f \quad \text{y} \quad t_j \uparrow g \quad \text{puntualmente.}
        \]
        Como la suma de funciones simples es simple, $s_j + t_j$ es también una sucesión creciente de funciones simples que converge puntualmente a $f + g$. Por el Teorema de la Convergencia Monótona (\cref{TCM}), se tiene:
        \[
            \int_{\mathbb{R}^n} (f + g) = \lim_{j \to \infty} \int_{\mathbb{R}^n} (s_j + t_j)
        \]
        Como la integral de Lebesgue es aditiva sobre funciones simples, se cumple que
        \[
            \int_{\mathbb{R}^n} (s_j + t_j) = \int_{\mathbb{R}^n} s_j + \int_{\mathbb{R}^n} t_j
        \]
        Luego, aplicando el límite,
        \[
            \int_{\mathbb{R}^n} (f + g) = \lim_{j \to \infty} \int_{\mathbb{R}^n} s_j + \lim_{j \to \infty} \int_{\mathbb{R}^n} t_j = \int_{\mathbb{R}^n} f + \int_{\mathbb{R}^n} g
        \]

        \item Sea $(f_k)_{k \in \mathbb{N}}$ una sucesión de funciones medibles no negativas. Definamos, para cada $m \in \mathbb{N}$,
        \[
            F_m := \sum_{k = 1}^m f_k
        \]
        Como cada $f_k \geq 0$, la sucesión $(F_m)_{m \in \mathbb{N}}$ es creciente y converge puntualmente a la serie
        \[
            F := \sum_{k = 1}^{\infty} f_k
        \]
        Por el apartado anterior, tenemos:
        \[
            \int_{\mathbb{R}^n} F_m = \sum_{k = 1}^{m} \int_{\mathbb{R}^n} f_k
        \]
        Como $(F_m)$ converge monótonamente a $F$, podemos aplicar el Teorema de la Convergencia Monótona:
        \[
            \int_{\mathbb{R}^n} \sum_{k = 1}^{\infty} f_k = \lim_{m \to \infty} \int_{\mathbb{R}^n} F_m = \lim_{m \to \infty} \sum_{k = 1}^{m} \int_{\mathbb{R}^n} f_k = \sum_{k = 1}^{\infty} \int_{\mathbb{R}^n} f_k
        \]
    \end{enumerate}
\end{proof}


\begin{lema} [Lema de Fatou]
    Sea $(f_k)_{k\in\mathbb{R}^n}$ sucesión de funciones medibles no negativas, entonces: $$\int_{\mathbb{R}^n}\liminf_{k\to\infty}f_k \leq \liminf_{k\to\infty}\int_{\mathbb{R}^n}f_k$$
\end{lema}
\begin{proof}
    Sea $$ f := \liminf_{k \to \infty} f_k = \lim_{k \to \infty} \left(\inf_{j \geq k} f_j\right) = \lim_{k \to \infty} g_k \quad \text{ donde } \quad g_k := \inf_{j \geq k} f_j $$
    Dado que $ g_k \geq 0 $, la sucesión $ (g_k)_{k \in \mathbb{N}} $ está compuesta por funciones medibles y no negativas para todo $ k \in \mathbb{N} $. Además, es una sucesión creciente en el sentido de que
    $$ g_k \leq g_{k+1}, \quad \forall k \in \mathbb{N} $$ Por el \cref{TCM} (TCM), se tiene que:
    $$ \lim_{k \to \infty} \int_{\mathbb{R}^n} g_k = \int_{\mathbb{R}^n} \lim_{k \to \infty} g_k$$ Por construcción de la sucesión $ (g_k)_{k \in \mathbb{N}} $ (en particular por ser monótona creciente), se cumple la igualdad:
    $$\liminf_{k \to \infty} \int_{\mathbb{R}^n} g_k = \lim_{k \to \infty} \int_{\mathbb{R}^n} g_k$$
    Finalmente, dado que $ g_k \leq f_k $, se concluye que:
    $$\int_{\mathbb{R}^n} g_k \leq \int_{\mathbb{R}^n} f_k \implies \int_{\mathbb{R}^n} \lim_{k \to \infty} g_k = \lim_{k \to \infty} \int_{\mathbb{R}^n} g_k = \liminf_{k \to \infty} \int_{\mathbb{R}^n} g_k \leq \liminf_{k \to \infty} \int_{\mathbb{R}^n} f_k$$
    Nótese que para dos sucesiones $ (a_k)_{k \in \mathbb{N}} $ y $ (b_k)_{k \in \mathbb{N}} $ tal que $ a_k \leq b_k $ para todo $ k \in \mathbb{N} $, se cumple que:
    $$ \liminf_{k \to \infty} a_k \leq \liminf_{k \to \infty} b_k $$
\end{proof}
\begin{observación}
El resultado análogo con $\limsup$ no es válido en general. Fijémonos que si intentásemos una demostración análoga, no se podría aplicar el \cref{TCM} (TCM), pues la sucesión de funciones $ (h_k)_{k \in \mathbb{N}} $ definida por $ h_k = \sup_{j \geq k} f_j $ no es creciente, sino decreciente. Podemos tomar de contraejemplo la función $f_k = k \cdot \chi_{[k, \infty]}$.
\end{observación}

\subsection{Funciones Integrables-Lebesgue}

\begin{definición} [Integral de Lebesgue de funciones medibles]
    Sea $E \subset \mathbb{R}^n$ un conjunto medible y $f: E \to [-\infty, +\infty]$ una función medible. Sean $f^+ = \max\{f, 0\}$ y $f^- = \max\{-f, 0\}$ las partes positiva y negativa de $f$. Definimos la integral de Lebesgue de $f$ en $E$ como:
    \[
        \int_E f := \int_E f^+ - \int_E f^- = \int_{\mathbb{R}^n} f^+ \circ \chi_E - \int_{\mathbb{R}^n} f^- \circ \chi_E
    \]
\end{definición}

\begin{definición}[Función Integrable\label{Función Integrable}]
Sean $E \subset \mathbb{R}^n$ conjunto medible y $f: E \to [-\infty, +\infty]$ función medible. Se dice que f es integrable (o absolutamente integrable) en $E$ cuando 
$$\int_{E}f < +\infty$$ 
Es decir cuando 
$$\int_{\mathbb{R}^n}f \circ \chi_E < +\infty$$
\end{definición}

\begin{observación}
    Una función \( f \) (no necesariamente no negativa) es integrable en \( E \) si y sólo si \( |f| \) es integrable en \( E \), lo cual también es equivalente a que sus partes positiva y negativa, \( f^+ = \max\{f, 0\} \) y \( f^- = \max\{-f, 0\} \), sean ambas integrables en \( E \):
    \[
    f \in L^1(E) \iff |f| \in L^1(E) \iff f^+, f^- \in L^1(E)
    \]
\end{observación}

\begin{lema}
    Sean $E \subset \mathbb{R}^n$ y $f = g - h$ con $g, h: E \to [0, +\infty]$ funciones integrables no negativas. Entonces, $$ \int_{E}f = \int_{E}g - \int_{E}h$$
    \label{lema:IntegralResta}
\end{lema}

\begin{proof}
    Primero probemos que $f$ es integrable. Como $f = g - h$, tenemos:
    \[ |f| = |g - h| \leq |g| + |h|. \]
    Dado que $g$ y $h$ son integrables, $|g|$ y $|h|$ también lo son, y por lo tanto $|g| + |h|$ es integrable. La desigualdad anterior implica que $f$ es integrable.

    Por otro lado, expresando $f$ en términos de sus partes positiva y negativa:
    \[ f = f^+ - f^- = g - h. \]
    Reordenando términos obtenemos:
    \[ f^+ + h = f^- + g. \]
    Al tratarse de funciones no negativas, podemos aplicar la linealidad de la integral de Lebesgue (\cref{cor:TCM})
    \[ \int_{E} f^+ + h = \int_{E} f^+ + \int_{E} h = \int_{E} f^- + g = \int_{E} f^- + \int_{E} g. \]
    Finalmente, restando $\int_{E} h$ y $\int_{E} f^-$ en ambos lados:
    \[ \int_{E} f^+ - \int_{E} f^- = \int_{E} g - \int_{E} h, \]
    lo cual es equivalente a:
    \[ \int_{E} f = \int_{E} g - \int_{E} h, \]
    completando la demostración.
\end{proof}

\begin{proposición}
Para funciones $f$ y $g$ integrables en $E$, se cumplen las siguientes propiedades:
\vspace{-0.5em}
\begin{enumerate}
    %1%
    \item Si $f, g$ son integrables en $E$, entonces $f+g$ también es integrable y $$
              \int_E (f+g) = \int_E f + \int_E g $$

          %2%
    \item Si $f$ es integrable en $E$ y $c \in \mathbb{R}$, entonces $cf$ es integrable
          en $E$ y $$ \int_E (cf) = c \int_E f $$

          %3%
    \item Si $f \leq g$ en casi todo punto de $E$, entonces $$ \int_E f \leq \int_E g $$

          %4%
    \item Si $|f|$ es integrable en $E$, entonces $f$ también es integrable y $$ \left|
              \int_E f \right| \leq \int_E |f| $$

          %5%
    \item Si $f = g$ en casi todo punto de $E$ y $f$ es integrable en $E$, entonces $g$
          también es integrable en $E$ con, $$ \int_E f = \int_E g $$

          %6%
    \item Si $m(E) = 0$ y $f$ es medible, entonces es integrable en $E$ y $$ \int_E f = 0
          $$

          %7%
    \item Si $f$ es integrable en $E$ entonces $|f| < \infty$ en casi todo punto de $E$

          %8%
    \item Si $\int_E |f| = 0$, entonces $f = 0$ en casi todo punto de $E$.
\end{enumerate}
\end{proposición}
\begin{proof}
    \leavevmode
    \begin{enumerate}
        \item Dado que $f = f^+ - f^-$ y $g = g^+ - g^-$, se sigue que:
        \[ f + g = (f^+ + g^+) - (f^- + g^-), \]
        donde ambos términos $(f^+ + g^+)$ y $(f^- + g^-)$ son no negativos. Por el lema de integración de funciones no negativas, tenemos:
        \[ \int_E (f + g) = \int_E (f^+ + g^+) - \int_E (f^- + g^-). \]
              Aplicando el \cref{cor:TCM} a cada integral y reordenando los términos:
                $$ = \left(\int_E f^+ + \int_E g^+\right) - \left(\int_E f^- + \int_E g^-\right) = \left(\int_E f^+ - \int_E f^-\right) + \left(\int_E g^+ - \int_E g^-\right)$$
                De nuevo, por el \cref{lema:IntegralResta}, obtenemos que:
                $$ \int_E (f+g) = \int_E f + \int_E g $$
        \item     Consideremos primero el caso $c > 0$. Como $f = f^+ - f^-$, se tiene que:
        \[ cf = cf^+ - cf^-, \]
        donde tanto $cf^+$ como $cf^-$ son funciones no negativas.
    
        Aplicando el apartado (3) de la \cref{prop:PropiedadesIntegralesLebesgue}, obtenemos:
        \[ \int_E cf = \int_E cf^+ - \int_E cf^- = c\int_E f^+ - c\int_E f^-. \]
        
        Factorizando la constante $c$, concluimos que:
        \[ \int_E cf = c\left(\int_E f^+ - \int_E f^-\right) = c\int_E f. \]
    
        Para el caso $c < 0$, observemos primero que:
        \[ cf = (-|c|)f = -(|c|f) = -(|c|f^+ - |c|f^-) = |c|f^- - |c|f^+. \]
        
        Integrando ambos lados y por la proposición aplicada anteriormente, tenemos:
        \[ \int_E cf = \int_E |c|f^- - \int_E |c|f^+ = |c|\int_E f^- - |c|\int_E f^+. \]
        
        Factorizando y teniendo en cuenta que $c = -|c|$:
        \[ \int_E cf = -|c|\left(\int_E f^+ - \int_E f^-\right) = c\int_E f. \]

        \item Como $g - f \geq 0$ en casi todo punto de $E$, podemos hacer la descomposición de $f$ y $g$ en sus partes positiva y negativa, de forma que $g - f = (g^+ - g^-) - (f^+ - f^-) = (g^+ + f^-) - (g^- + f^+) \geq 0$ en casi todo punto de $E$. Aplicando el apartado (1') de la \cref{prop:PropiedadesIntegralesLebesgue}, tenemos que:
        $$ 0 \leq g^- + f^+ \leq g^+ + f^- \text{ en casi todo punto de } E \implies \int_E (g^- + f^+) \leq \int_E (g^+ + f^-) $$
        Luego haciendo uso de (2) de la \cref{prop:PropiedadesIntegralesLebesgue} en cada integral:
        $$ \int_E g^- + \int_E f^+ \leq \int_E g^+ + \int_E f^- $$
        Reordenando los términos, obtenemos:
        $$ \int_E f^+ - \int_E f^- \leq \int_E g^+ - \int_E g^-  \implies \int_E f \leq \int_E g $$

        \item Se tiene que $|f| = f^+ + f^-$. Usando la linealidad de la integral, $$ \left|\int_E
                  f\right| = \left|\int_E f^+ + \int_E f^- \right| $$ Como $f = f^+ - f^-$, aplicamos la
              desigualdad triangular: $$ \left| \int_E f \right| = \left| \int_E f^+ - \int_E
                  f^- \right| \leq \int_E f^+ + \int_E f^- = \int_E |f| $$

        \item Como $f = g$ en casi todo punto de $E \implies$ $f^+ = g^+$ y $f^- = g^-$ en
              casi todo punto de $E$, por lo que sólo queda aplicar el propiedad (8) de la \cref{prop:PropiedadesIntegralesLebesgue}:
              $$ \int_E f^+ = \int_E g^+ \text{ y } \int_E f^- = \int_E g^-$$
              Así,
                $$ \int_E f = \int_E f^+ - \int_E f^- = \int_E g^+ - \int_E g^- = \int_E g $$

        \item Sea $E \subset \mathbb{R}^n$ conjunto de medida nula, puesto que $0 \leq |f|$, entonces por (4) de la \cref{prop:PropiedadesIntegralesLebesgue} se cumple que:
        $$ \int_E |f| = 0 $$
        Aplicando el apartado anterior se deduce el resultado:
        $$ 0 \leq \left|\int_{E}f \right| \leq \int_{E}|f| = 0 \implies \int_{E}f = 0 $$

        \item Por definición, sabemos que la función $f$ es integrable en $E$ si $$ \int_E f <
                  +\infty \iff \int_{E} |f| < +\infty$$ Sea $Z \subset E$ el conjunto de puntos
              de $E$ donde $|f| = \infty$. Entonces tenemos por la propiedad (7) de la \cref{prop:PropiedadesIntegralesLebesgue} que: $$ \int_E |f| = \int_{E
                      \setminus Z} |f| + \int_Z |f| < +\infty \implies \int_{Z} |f| < +\infty
                  \implies m(Z) = 0. $$ Por lo tanto, $|f| < \infty$ en casi todo punto de $E$.
        \item Sea $$ A = \{ x \in E : |f(x)| > 0 \}. $$ Definimos los conjuntos $$ A_k = \{ x
                  \in E : |f(x)| > \frac{1}{k} \}, \quad \forall k \in \mathbb{N}, $$ por lo que
              $$ A = \bigcup_{k=1}^{\infty} A_k. $$

              Ahora, evaluamos la medida de $ A_k $ utilizando la integral: $$ m(A_k) =
                  \int_{A_k} 1 \leq \int_{A_k} k \cdot |f| = k \int_{A_k} |f| \leq \int_{A_k} |f|
                  \leq \int_E |f|$$ Tomando el límite cuando $ k \to \infty $ (y de la
              subaditividad) se concluye que $$ m(A) = \lim_{k \to \infty} m(A_k) = 0. $$
    \end{enumerate}
\end{proof}
\begin{teorema} [Convergencia Dominada\label{Convergencia Dominada}]
    Sean $E \subset \mathbb{R}^n$ medible y $\forall k \in \mathbb{N}, f_k: E \to [-\infty, +\infty]$ funciones medibles. Supongamos que $\exists g: E \to [0, +\infty]$ integrable en E tal que $|f_k| < g$ en casi todo punto de $E$ y $\forall k \in \mathbb{N}$. Si además suponemos que $\lim_{k \to \infty} f_k = f$ en casi todo punto de $E$, entonces:
    \vspace{-0.5em}
    \begin{enumerate}
        \item $f_k \text{ y } f \text{ son integrables en }E$
        \item $\lim_{k \to \infty} \int_{E} |f_k - f| = 0$
        \item $\lim_{k \to \infty} \int_{E} f_k = \int_{E} f$
    \end{enumerate}
\end{teorema}
\begin{proof}
    \leavevmode

    \begin{enumerate}
        \item Dado que \( |f_k| \leq |g| = g \quad \forall k \in \mathbb{N} \), se concluye
              que \( f_k \) es integrable en \( E \). Además, como \( |f| \leq g \), se sigue
              que \( f \) también es integrable en \( E \).
        \item Observamos que, como $|f_k| \leq g$ y $|f| \leq g$ para casi todo $x \in E$,
              entonces:
              \[
                  |f_k(x) - f(x)| \leq |f_k(x)| + |f(x)| \leq g(x) + g(x) = 2g(x)
                  \quad \text{para casi todo } x \in E
              \]
              Por lo tanto, la función $|f_k - f|$ está acotada superiormente por $2g$, que
              es integrable. Sea ahora:
              \[
                  h_k(x) := 2g(x) - |f_k(x) - f(x)| \geq 0
              \]
              Como $f_k(x) \to f(x)$ casi en todo punto de $E$, se tiene:
              \[
                  |f_k(x) - f(x)| \to 0 \quad \implies \quad h_k(x) \to 2g(x)
              \]

              Aplicamos el Lema de Fatou a la sucesión de funciones no negativas $(h_k)$:
              \[
                  \int_E \liminf_{k \to \infty} h_k \leq \liminf_{k \to \infty} \int_E h_k
              \]
              Dado que $h_k(x) \to 2g(x)$, entonces:
              \[
                  \int_E 2g \leq \liminf_{k \to \infty} \int_E h_k
              \]
              Pero como $h_k = 2g - |f_k - f|$, se tiene:
              \[
                  \int_E h_k = \int_E (2g - |f_k - f|) = \int_E 2g - \int_E |f_k - f|
              \]
              Sustituyendo en la desigualdad anterior y utilizando el siguiente lema
              \begin{lema}
                  Si \( a_k \to a \), entonces
                  \[
                      \liminf_k (a_k + b_k) \geq \liminf_k a_k + \liminf_k b_k
                  \]
              \end{lema}
              se cumple que:
              \[
                  \int_E 2g \leq \liminf_{k \to \infty} \left( \int_E 2g - \int_E |f_k - f| \right)
                  \leq \liminf_{k \to \infty} \left(\int_E 2g\right) + \liminf_{k \to \infty} \left(-\int_E |f_k - f|\right)$$
              $$= \int_E 2g - \limsup_{k \to \infty} \int_E |f_k - f|
              \]
              Restando $\int_E 2g$ en ambos lados:
              \[
                  0 \leq - \limsup_{k \to \infty} \int_E |f_k - f| \quad \implies \quad \limsup_{k \to \infty} \int_E |f_k - f| \leq 0
              \]
              Como la integral de una función no negativa también es no negativa:
              \[
                  0 \leq \int_E |f_k - f| \leq \limsup_{k \to \infty} \int_E |f_k - f| \leq 0 \quad \implies \quad \lim_{k \to \infty} \int_E |f_k - f| = 0
              \]
        \item Finalmente, aplicamos la propiedad de la integral a la diferencia \( f_k - f
              \):
              \[
                  \left| \int_E f_k - \int_E f \right| = \left| \int_E (f_k - f) \right| \leq \int_E |f_k - f| \xrightarrow{k \to \infty} 0
              \]
              Por lo tanto, se concluye que:
              \[
                  \lim_{k \to \infty} \int_E f_k = \int_E f
              \]
    \end{enumerate}
\end{proof}
\begin{definición}[Integral Paramétrica\label{Integral Paramétrica}]
Sea $f$ función integrable, se define una función por su integral paramétrica como:
$$ F(u) = \int_{E}f(x, u)dx$$
\end{definición}

\begin{teorema}
    Sean $E \subset \mathbb{R}^n$ conjunto medible, $U \subset \mathbb{R}^n$ conjunto cualquiera, $f: E \times U \to \mathbb{R}$ y suponemos que:
    \vspace{-0.5em}
    \begin{enumerate}
        \item $\forall u \in U \ f(\cdot, u): E \to \mathbb{R}$ es medible.
        \item $\forall x \in E \ f(x, \cdot): U \to \mathbb{R}$ es continua.
        \item $\exists g: E \to [0, +\infty]$ integrable en $E$ tal que $|f(x, u)| \leq g(x)$ en casi todo punto de $E$ y $\forall u \in U$.
    \end{enumerate}
    Entonces podemos decir que:
    $$ F(u) = \int_{E}f(x, u)dx $$ es una función continua en $U$.
\end{teorema}
\begin{proof}
    Sea \( ( u_k )_{k \in \mathbb{N}} \subset U \) tal que \( u_k \to u_0 \in U \).
    ¿Se sigue que \( ( F(u_k) )_{k \in \N} \xrightarrow{k \to \infty} F(u_0) \) ?

    Para cada \( k \in \mathbb{N} \), definimos
    \[
        f_k = f(\cdot, u_k): E \to \mathbb{R}
    \]
    que es una función medible. Por la condición (2), se cumple que \( \forall x
    \in E \),
    \[
        f_k(x) = f(x, u_k) \xrightarrow{k \to \infty} f(x, u_0).
    \]
    Es decir, la sucesión \( \{ f_k \} \) converge puntualmente en \( E \) a
    \[
        f_0(x) = f(x, u_0).
    \]

    Además, se cumple que
    \[
        |f_k(x)| = |f(x, u_k)| \leq g(x), \quad \forall k \in \mathbb{N}, \quad \forall x \in E.
    \]

    Aplicando el \cref{Convergencia Dominada} (TCD), se concluye que \( f_k \) es
    integrable para todo \( k \in \mathbb{N} \) y
    \[
        \int_E f_k \to \int_E f.
    \]
    Es decir,
    \[
        F(u_0) = \int_E f(x, u_0) \,dx.
    \]
    Por lo tanto, se deduce que
    \[
        F(u_k) = \int_E f(x, u_k) \,dx \quad \Rightarrow \quad F(u) = \int_E f(x, u) \,dx
    \]
\end{proof}
\begin{observación}
$\forall u_0 \in U \ \lim_{u \to u_0} \int_{E}f(x, u)dx = F(u) = F(u_0) = \int_{E}f(x, u_0)dx$
\end{observación}
% Regla de Leibniz
\begin{teorema} [Regla de Leibniz]
    Sean $E \subset \mathbb{R}^n$ conjunto medible, $U = (a, b) \subset \mathbb{R}$ conjunto abierto y $f: E \times U \to \mathbb{R}$ medible. Y además supongamos que:
    \vspace{-0.5em}
    \begin{enumerate}
        \item $\forall u \in U \ f(\cdot, u): E \to \mathbb{R}$ es integrable en E.
        \item $\forall x \in E \ f(x, \cdot): U \to \mathbb{R}$ es de clase $C^1$ en $U$.
        \item $\exists g: E \to [0, +\infty]$ integrable en $E$ tal que $|\frac{\partial f}{\partial u}(x, u)| \leq g(x)$ en casi todo punto de $E$ y $\forall u \in U$.
    \end{enumerate}
    Entonces se cumple que:
    $$ F(t) = \int_{E}f(x,t)dx $$ es de clase $C^1$ en $U$ y $\forall t \in U$ se cumple que: $$ F'(t) = \int_{E}\frac{\partial f}{\partial t}(x, t)dx $$
\end{teorema}
\begin{proof}
    Fijamos \( t_0 \in (a,b) \) y definimos la función \( h: E \times (a,b) \to \mathbb{R} \) como:

    \[
        h(x,t) =
        \begin{cases}
            \frac{f(x,t) - f(x,t_0)}{t - t_0},    & t \neq t_0 \\
            \frac{\partial}{\partial t} f(x,t_0), & t = t_0
        \end{cases}
    \]

    \begin{enumerate}

        \item Medibilidad de \( h(x,t) \)

              Queremos ver que \( h(x,t) \) es medible para todo \( t \in (a,b) \).

              - Si \( t \neq t_0 \), es claro.
              - Si \( t = t_0 \), tenemos que:

              \[
                  h(x,t_0) = \lim_{k \to \infty} \frac{f(x,t_0 + 1/k) - f(x,t_0)}{1/k}
              \]

              lo cual es medible.

        \item Continuidad de \( h(x, \cdot) \)

              Para todo \( x \in E \), si \( h(x, \cdot) \) es acotada en \( (a,b) \),
              entonces es continua.

              - Si \( t \neq t_0 \), es claro.
              - Si \( t = t_0 \), tenemos:

              \[
                  h(x,t_0) = \frac{\partial}{\partial t} f(x,t_0) = \lim_{t \to t_0} h(x,t),
              \]

              lo cual prueba la continuidad.

        \item Acotación y aplicación de la Regla de Leibniz

              \[
                  |h(x,t)| \leq g(x)
              \]

              - Si \( t = t_0 \), es claro.
              - Si \( t \neq t_0 \), por el Teorema del Valor Medio, existe \( c \in (t,t_0) \) tal que:

              \[
                  \left| \frac{f(x,t) - f(x,t_0)}{t - t_0} \right| = \left| \frac{\partial}{\partial t} f(x,s) \right| \leq g(x).
              \]

              Por la Regla de Leibniz, obtenemos:

              \[
                  F'(t_0) = \lim_{t \to t_0} \frac{F(t) - F(t_0)}{t - t_0} = \lim_{t \to t_0} \int_E \frac{f(x,t) - f(x,t_0)}{t - t_0} \,dx =
              \]
              \[
                  \lim_{t \to t_0} \left( \int_E h(x,t) dx \right) = \int_E \left( \lim_{t \to t_0} h(x,t) \right) dx = \int_E \frac{\partial}{\partial t} f(x,t) \,dx.
              \]

              Finalmente, como \( F' \) es continua en \( (a,b) \), se concluye que \( F \in
              C^1(a,b) \).

    \end{enumerate}
\end{proof}

\subsection{Relación entre la integral de Lebesgue y la integral de Riemann}

\begin{teorema}
    Sea $[a, b] \subset \mathbb{R}$ y $ f:[a,b] \to \mathbb{R}$ integrable Riemann en $[a, b]$. Entonces $f$ es integrable Lebesgue en $[a, b]$ y se cumple que:
    $$ (L) \int_{a}^{b}f = (R) \int_{a}^{b}f $$
\end{teorema}
\begin{observación}
Denotamos $\int_{a}^{b}f = \int_{[a, b]}f$
\end{observación}
\begin{proof}
    $\forall k \in \mathbb{N}$ sabemos que $\exists P_k = \{ a = x_0^k < x_1^k < \dots < x_{n(k)}^k = b \} \subset [a,b]$ tal que: $\bar{S}(f, P_k) - \underline{S}(f, P_k) < \frac{1}{k}$.
    Suponemos que $P_{k+1}$ es mas fina que $P_{k}$ y además que $$\text{diam}(P_k) = \sup_{i \in \{1, \dots, n(k)\}}(x_i^k - x_{i-1}^k) < \frac{1}{k}$$
    \\$\forall k \in \mathbb{N}$ denotamos $m_k = \inf\{f(x) : x \in [x_{i-1}^k, x_i^k]\}$ y $M_k = \sup\{f(x) : x \in [x_{i-1}^k, x_i^k]\}$.
    $$ \underline{S}(f, P_k) = \sum_{i=1}^{n(k)}m_k(x_i^k - x_{i-1}^k) = \int_{a}^{b}\varphi_k \quad \text{con} \quad \varphi_k = \sum_{i = 1}^{n(k)}m_i^k\cdot\chi_{[x_{i-1}^k, x_i^k)}$$
    $$\bar{S}(f, P_k) = \sum_{i=1}^{n(k)}M_k(x_i^k - x_{i-1}^k) = \int_{a}^{b} \psi_k \quad \text{con}  \quad \psi_k  = \sum_{i = 1}^{n(k)}M_i^k\cdot\chi_{[x_{i-1}^k, x_i^k)}$$
    Es claro que $\varphi_k  \leq f \leq \psi_k$ en [a,b].
    Además, como $P_{k+1}$ es más fino que $P_k \implies (\varphi_{k})\uparrow$ y $(\psi_k)\downarrow$
    Denotamos $\varphi = \lim_{k \to \infty}\varphi_k = \sup\varphi_k$
    y $\psi  = \lim_{k \to \infty}\psi_k = \inf\psi_k$ que son medibles y cumplen que $\varphi \leq f \leq \psi$.\\
    Como $f$ es integrable-Riemann $\implies f$ es acotada $\iff \exists M \in \mathbb{N}$ tal que $|f(x)| \leq M, \ \forall x \in [a, b]$.
    La función  $ g(x) = M $ es integrable en $[a, b]$ y puesto que $|\psi_k| \leq g$ y $|\varphi_k| \leq g$ entonces por el Teorema de la Convergencia Dominada: $$\underline{S}(f, P_{k}) = \int_{a}^{b}{\varphi_k} \to \int_{a}^{b}\varphi \qquad \bar{S}(f, P_{k}) = \int_a^b\psi_k \to \int_a^b\psi$$
    Pero a su vez, también se cumple que: \\ $$\underline{S}(f, P_k) \to (R)\int_a^b f \quad \text{y} \quad \bar{S}(f, P_k) \to (R)\int_{a}^{b} f \implies \int_{a}^{b} \varphi = (R)\int_{a}^{b} f = \int_{a}^{b} \psi$$
    Y como $\int_{a}^{b} \psi - \varphi = 0 \implies \psi - \varphi = 0$ en casi todo punto de $[a, b]$. Es decir $\varphi = f = \psi$ en casi todo punto de $[a, b]$. Y finalmente obtenemos que:
    $$(L)\int_{a}^{b}f = \int_{a}^{b}\varphi = \int_a^b\psi = (R)\int_{a}^{b}f$$
\end{proof}

\begin{teorema}
    Sean $[a, b]\subset \mathbb{R}^n$ y $f: [a, b] \to \mathbb{R}$ una función acotada. Entonces $f$ es integrable-Riemann en $[a, b] \iff D_f = \{ x \in [a, b] \mid f \text{ no es continua en } x \}$ tiene medida nula.
\end{teorema}
\ejemplo{
La función de Dirichlet
\[
    f = \chi_{\mathbb{Q} \cap [0,1]}: [0,1] \to \mathbb{R}, \quad f(x) = \begin{cases} 1 & x \in \mathbb{Q} \\ 0 & x \in \mathbb{R} \setminus \mathbb{Q} \end{cases}
\]
no es integrable-Riemann en $[0, 1]$. Pero $f = 0$ en casi todo punto $\implies
    f$ es integrable-Lebesgue y ésta vale: $\int_{[0, 1]}f = \int_{[0, 1]}0 = 0$ }

\begin{teorema}
    Sean $-\infty \leq \alpha < \beta \leq +\infty$ y $f: (\alpha, \beta) \to \mathbb{R}$ una función absolutamente integrable-Riemann impropia en el intervalo $(\alpha, \beta)$. Entonces $f$ es integrable-Lebesgue en $(\alpha, \beta)$ y se cumple que:
    $$ (L) \int_{\alpha}^{\beta}f = (R)\int_{\alpha}^{\beta}f $$
\end{teorema}
\begin{proof}
    Habría que realizar una distinción de casos según el tipo de intervalo que sea $(\alpha, \beta)$, en este caso trataremos el intervalo $[\alpha, \infty)$:
    Por hipótesis sabemos que:
    \begin{enumerate}
        \item $\forall k \in \mathbb{N}, f$ es integrable-Riemann en $[a, b]$
        \item $\lim_{b \to \infty} \int_{a}^{b}|f| < +\infty$
    \end{enumerate}
    Tomamos una sucesión $(b_n)_{n \in \mathbb{N}} \uparrow +\infty$ y definimos las sucesiones de funciones: $f_n = f\cdot\chi_{[a, b_n]}$ y $g_n = |f|\cdot\chi_{[a, b_n]}$ medibles. De manera que tenemos que $f_n \uparrow f$ y $g_n \uparrow |f|$. Entonces aplicamos el Teorema de la Convergencia Monóntona:
    \begin{enumerate}
        \item $(L)\int_{a}^{+\infty}|f| = \lim_{n \to \infty}(L)\int_{a}^{b_n}|f| = \lim_{n \to \infty}(R)\int_{a}^{b_n}|f| = (R)\int_{a}^{+\infty}|f| < \infty$
        \item Esto muestra que $f$ es integrable-Lebesgue en $[a, +\infty)$.
    \end{enumerate}
    Por otra parte, como $|f_n| \leq |f| \text{  }\forall n \in \mathbb{N}$ por el Teorema de la Convergencia Dominada se tiene que:
    \begin{enumerate}
        \item $(L)\int_{a}^{+\infty}f = \lim_{n \to \infty}(L)\int_{a}^{\infty}f_n = \lim_{n \to \infty}(R)\int_{a}^{b_n}f = (R)\int_{a}^{+\infty}f$
    \end{enumerate}
    Finalmente obtenemos el resultado de que $f$ es integrable de Riemann-impropia en $[a, +\infty)$.
    \\$\forall (b_n)_{n \in \mathbb{N}} : b_n \to \infty$ tenemos que $|\int_{b_n}^{b_m}f| \leq \int_{b_n}^{b_m}|f| \leq \varepsilon$
\end{proof}
\ejemplo{
(Hoja 3. Ej: 6.a)
Calculemos
\[ F(t) = \int_{0}^{+\infty} \frac{\sin(tx)}{x} e^{-x} \,dx, \quad \forall t \in \mathbb{R} \]
derivando con respecto al parámetro \( t \). Para ello, aplicamos el Teorema de
Leibniz:

Sea \( E \subset \mathbb{R}^n \) medible y \( (a,b) \subset \mathbb{R} \), con
\( f: E \times (a,b) \to \mathbb{R} \) tal que:

\begin{enumerate}
    \item \( \forall u \in (a,b), f(\cdot, u): E \to \mathbb{R} \) es integrable en \( E \).
    \item Para casi todo \( x \in E \), la función \( f(x, \cdot): (a,b) \to \mathbb{R}
          \) es de clase \( C^1 \) en \( (a,b) \).
    \item Existe \( g: E \to [0, +\infty] \) integrable en \( E \) tal que
          \[ \left| \frac{\partial f}{\partial t}(x, t) \right| \leq g(x) \quad \text{para casi todo } x \in E, \forall u \in (a,b). \]
\end{enumerate}

Entonces, \( F(t) \) es de clase \( C^1 \) en \( \mathbb{R} \) y se cumple:
\[ F'(t) = \int_{0}^{+\infty} \frac{\partial f}{\partial t}(x, t) \,dx. \]

Dado que
\[ f(x,t) = \frac{\sin(tx)}{x} e^{-x}, \]
calculamos la derivada parcial con respecto a \( t \):
\[ \frac{\partial f}{\partial t}(x,t) = \cos(tx) e^{-x}. \]

Verifiquemos cada una de las hipótesis del Teorema de Leibniz:
\begin{enumerate}
    \item \( \forall t \in \mathbb{R}, f(x,t) \) es integrable en \( [0, +\infty) \):
          \[ |f(x,t)| \leq e^{-x} = g(x). \]
          Como \( \int_{0}^{+\infty} e^{-x} \,dx = 1 < +\infty \), se cumple la
          integrabilidad.

    \item \( \forall x \in E, \frac{\partial f}{\partial t}(x,t) = \cos(tx) e^{-x} \) es continua en \( \mathbb{R} \), por lo que \( f(x, \cdot) \) es de clase \( C^1 \) en \( \mathbb{R} \).

    \item Se cumple que
          \[ \left| \frac{\partial f}{\partial t}(x, t) \right| = |\cos(tx) e^{-x}| \leq e^{-x} = g(x), \]
          que es integrable en \( [0, +\infty) \).
\end{enumerate}

Por lo tanto, \( F \) es de clase \( C^1 \) en \( \mathbb{R} \) y
\[ F'(t) = \int_{0}^{+\infty} \cos(tx) e^{-x} \,dx. \]

Ahora calculemos esta integral:
\[ I(t) = \int_{0}^{+\infty} \cos(tx) e^{-x} \,dx. \]

Usando integración por partes con
\[
    \begin{cases}
        u = \cos(tx),      & dv = e^{-x}dx, \\
        du = -t\sin(tx)dx, & v = -e^{-x},
    \end{cases}
\]
obtenemos:
\[ I(t) = [\cos(tx) e^{-x}]_{0}^{+\infty} - t \int_{0}^{+\infty} \sin(tx) e^{-x}dx. \]
Evaluando los límites y repitiendo el proceso para \( \sin(tx) e^{-x} \),
obtenemos:
\[ I(t) (1+t^2) = 1. \]
Despejando:
\[ I(t) = \frac{1}{1+t^2} = F'(t). \]

Finalmente, integramos:
\[ F(t) = \int \frac{dt}{1+t^2} = \arctan(t) + C. \]

Si \( t = 0 \), entonces
\[ F(0) = \int_{0}^{+\infty} 0 = 0 \Rightarrow C = 0. \]
Por lo tanto:
\[ F(t) = \arctan(t). \]
}

