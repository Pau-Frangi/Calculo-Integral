
\section{Medida de Lebesgue}

\subsection{Medida Exterior de Lebesgue en $\R^n$}

\begin{definición}[n-Réctangulo\label{n-rectángulo}]
Un n-rectángulo en $\R^n$ es un conjunto de la forma:
\[
    R = \prod_{i=1}^n [a_i, \ b_i] = [a_1, \ b_1] \times [a_2, \ b_2] \times ... \times [a_n, \ b_n] \text{ donde } a_i \leq b_i \ \forall i = 1, 2, \ldots, n
\]
Definimos el volúmen de $R$ como:
\[
    \text{vol}(R) := \prod_{i=1}^n (b_i - a_i)
\]
Consideramos también los n-rectángulos abiertos denotados por $\mathring{R}$,
que se definen de forma análoga. Si nos se especifica si un rectángulo es
abierto o cerrado, se asume que es cerrado.
\end{definición}

\begin{observación}
Dado $R$ n-rectángulo cerrado tal que $R = \prod_{i=1}^n [a_i, \ b_i]$, podemos considerar para cada $\delta > 0$ el n-rectángulo abierto $R_\delta = \prod_{i=1}^n (a_i - \delta, \ b_i + \delta)$. Se tiene que
$R \subset R_\delta$ y
$$\text{vol}(R_\delta) = \prod_{i=1}^n (b_i - a_i + 2\delta) = \text{vol}(R) + 2n\delta$$
Por tanto
\[
    \text{vol}(R) = \lim_{\delta \to 0} \text{vol}(R_\delta)
\]
\end{observación}

\begin{definición}[Medida exterior de Lebesgue\label{Medida de exterior de Lebesgue}]
Sea $A \subset \R^n$. Definimos la medida exterior de $A$ como:
\[
    m^*(A) := \inf \left\{ \sum_{i=1}^\infty \text{vol}(R_i) \ \middle| \ A \subset \bigcup_{i=1}^\infty R_i \text{ con } R_i \text{ n-rectángulos cerrados} \right\}
\]
Donde el ínfimo se toma sobre todas las colecciones numerables de n-rectángulos
que recubren $A$. A esta medida la llamamos medida de Lebesgue exterior.
\end{definición}

\begin{observación}
Sea $A \subset \R^n$ entonces:
\vspace{-0.5em}
\begin{enumerate}
    \item $m^*(A) = +\infty \iff \forall (R_j)_{j \in J} \text{ tal que } A \subset \bigcup_{j \in J} R_j \text{ se tiene que } \sum_{j \in J} \text{vol}(R_j) = +\infty$
    \item $m^*(A) = 0 \iff \forall \varepsilon > 0 \ \exists (R_j)_{j \in J} \text{ tal que } A \subset \bigcup_{j \in J} R_j \text{ y } \sum_{j \in J} \text{vol}(R_j) < \varepsilon$
    \item $m^*(A) = \alpha \in \R^+ \iff \forall \varepsilon > 0 \ \exists (R_j)_{j \in J} \text{ tal que } A \subset \bigcup_{j \in J} R_j \text{ y } \sum_{j \in J} \text{vol}(R_j) < \alpha + \varepsilon$
\end{enumerate}
\end{observación}

\begin{definición}[Conjunto nulo\label{Conjunto nulo}]
Se dice que $A \subset \R^n$ es un conjunto nulo si $m^*(A) = 0$.
\end{definición}

\ejemplo{
    \begin{enumerate}
        \item Si $R$ es un n-rectángulo degenerado, es decir, $R$ tiene alguno de los lados
              de longitud 0, entonces $R$ es un conjunto nulo ($m^*(R) = 0$).
        \item En $\mathbb{R}^2$, sea el conjunto $A = \{(x,x) : 0 \leq x \leq 1\}$. Dado
              $\varepsilon > 0$ tomamos $m \in \mathbb{N}$ tal que $m >
                  \frac{1}{\varepsilon}$. Consideramos $A \subset \bigcup_{i=1}^m [\frac{i-1}{m},
                  \frac{i}{m}] \times [\frac{i-1}{m}, \frac{i}{m}]$. Se tiene que $$m^*(A) \leq
                  \sum_{i=1}^m \text{vol}\left(\left[\frac{i-1}{m}, \frac{i}{m}\right] \times
                  \left[\frac{i-1}{m}, \frac{i}{m}\right]\right) = \frac{1}{m^2} \cdot m =
                  \frac{1}{m} < \varepsilon$$ Por tanto, $m^*(A) = 0$.
    \end{enumerate}
}

Denotamos por $\mathcal{P}(\R^n)$ al conjunto de todos los subconjuntos de
$\R^n$.
\begin{teorema}
    La función $m^* : \mathcal{P}(\R^n) \to [0, +\infty]$ satisface:
    \vspace{-0.5em}
    \begin{enumerate}
        \item $m^*(\emptyset) = 0$
        \item $m^*(A) \leq m^*(B)$ si $A \subset B$
        \item $m^*(\bigcup_{i=1}^\infty A_i) \leq \sum_{i=1}^\infty m^*(A_i)$
    \end{enumerate}
    \label{funcionMedida}
\end{teorema}

\begin{proof}
    % add a space between the proof text and the enumerate
    \leavevmode
    \begin{enumerate}
        \item $\emptyset \subset \bigcup_{i=1}^\infty R_j$ con $R_j$ n-rectángulos degenerados $\implies m^*(\emptyset) \leq \sum_{j=1}^\infty \text{vol}(R_j) = 0 \implies m^*(\emptyset) = 0$.
        \item Sea $A \subset B$ y sea $(R_j)_{j \in J}$ tal que $B \subset \bigcup_{j \in J}
                  R_j$. Entonces $(R_j)_{j \in J}$ es un recubrimiento de $A$ y por tanto $m^*(A)
                  \leq \sum_{j \in J} \text{vol}(R_j) \implies m^*(A) \leq m^*(B)$.
        \item Si $\sum_{j=1}^{\infty}{m^*(A_j)} = +\infty$ entonces el resultado es
              inmediato. Supongamos que $\sum_{j=1}^{\infty}{m^*(A_j)} < +\infty$. Sea
              $\varepsilon > 0$. Para cada $j \in \mathbb{N}$, $\exists (R_{j,i})_{i =
                      1}^\infty$ tal que $A_j \subset \bigcup_{i = 1}^\infty R_{j,i}$ y $\sum_{i =
                      1}^\infty \text{vol}(R_{j,i}) < m^*(A_j) + \frac{\varepsilon}{2^j}$. Entonces
              $\bigcup_{j=1}^{\infty} A_j \subset \bigcup_{j=1}^{\infty} \bigcup_{i =
                      1}^\infty R_{j,i}$ y por tanto se tiene que $$m^*\left(\bigcup_{j=1}^{\infty}
                  A_j\right) \leq \sum_{j=1}^{\infty} \sum_{i = 1}^\infty \text{vol}(R_{j,i}) <
                  \sum_{j=1}^{\infty} (m^*(A_j) + \frac{\varepsilon}{2^j}) = \sum_{j=1}^{\infty}
                  m^*(A_j) + \varepsilon$$ Como $\varepsilon$ es arbitrario, se tiene que
              $m^*(\bigcup_{j=1}^{\infty} A_j) \leq \sum_{j=1}^{\infty} m^*(A_j)$.
    \end{enumerate}
\end{proof}

\begin{corolario}
    La unión numerable de conjuntos nulos es un conjunto nulo.
\end{corolario}

\begin{proof}
    Sea $(A_j)_{j=1}^\infty \subset R^n$ tal que $m^*(A_j) = 0 \quad \forall j \in \mathbb{N}$ entonces $m^*(\bigcup_{j=1}^\infty A_j) \leq \sum_{j=1}^\infty m^*(A_j) = 0 \implies m^*(\bigcup_{j=1}^\infty A_j) = 0$.
\end{proof}

\begin{lema}
    Sea $A \in \mathbb{R}^n$. Entonces,
    \[
        m^*(A) = \inf \left\{ \sum_{i=1}^\infty \text{vol}(Q_i) \ \middle| \ A \subset \bigcup_{i=1}^\infty Q_i \text{ con } Q_i \text{ n-rectángulos abiertos} \right\}
    \]
\end{lema}

\begin{proof}
    Denotamos por $\beta$ el ínfimo de la expresión del enunciado del lema. Sea $(Q_j)_{j \in \N}$ una sucesión de rectángulos abiertos tal que
    \[
        A \subset \bigcup_{j \in \N} Q_j.
    \]
    De este modo,
    \[
        A \subset \bigcup_{j \in \N} Q_j \subset \bigcup_{j \in \N} \overline{Q}_j,
    \]
    y como
    \[
        \sum_{j \in \N} \text{vol}(\overline{Q}_j) = \sum_{j \in \N} \text{vol}(Q_j),
    \]
    se concluye $m^*(A) \leq \beta$.

    Veamos ahora la desigualdad opuesta, $\beta \leq m^*(A)$. Si $m^*(A) =
        +\infty$, entonces $\beta = +\infty$ y no hay nada que demostrar. Supongamos
    $m^*(A) < +\infty$. Sea $\varepsilon > 0$. Por definición de medida exterior,
    existe una sucesión $(R_j)_{j \in \N}$ de n-rectángulos cerrados con
    \[
        A \subset \bigcup_{j \in \N} R_j \quad \text{y} \quad \sum_{j \in \N} \text{vol}(R_j) < m^*(A) + \varepsilon.
    \]
    Para cada $j \in \N$, consideramos $\varepsilon_j = \frac{\varepsilon}{2^j}$.
    Eligiendo $\delta_j > 0$ lo suficientemente pequeño, se cumple
    \[
        \text{vol}(R_j)_{\delta_j} < \text{vol}(R_j) + \varepsilon_j \quad \text{para todo } j \in \N.
    \]

    Aquí, $\text{vol}(R_j)_{\delta_j}$ indica el volumen del n-rectángulo abierto
    $R_j$ con lados aumentados en $\delta_j$. Entonces,
    \[
        A \subset \bigcup_{j \in \N} R_j \subset \bigcup_{j \in \N} (R_j)_{\delta_j},
    \]
    y además,
    \[
        \sum_{j \in \N} \text{vol}(R_j)_{\delta_j} < \sum_{j \in \N} (\text{vol}(R_j) + \varepsilon_j) = \sum_{j \in \N} \text{vol}(R_j) + \varepsilon < m^*(A) + 2\varepsilon.
    \]
    Por lo tanto, $\beta \leq m^*(A)$.
\end{proof}

\begin{definición}[Partición de un conjunto\label{Partición de un Conjunto}]
Una partición del intervalo $[a,b]$ es una colección numerable de puntos
$$P = \{a = t_0 < t_1 < ... < t_n = b\}$$
Dado un n-rectángulo $R \subset \R^n$, una partición $P =  \{P_1, P_2, ..., P_n\}$ de $R$ es una colección particiones $P_i$ de $[a_i, b_i]$ para cada $i = 1, 2, ..., n$ siendo $R = \prod_{i=1}^n [a_i, b_i]$.
\end{definición}

Los subrectángulos de $P$ son los conjuntos de la forma
\[    S_{i_1, i_2, ..., i_n} = \prod_{j=1}^n [t_{i_j}^j, t_{i_j + 1}^j]
\]
Denotamos $S \in P$ para indicar que $S$ es un subrectángulo de $P$.

\begin{lema}
    Sea $R \subset \R^n$ un n-rectángulo y $P$ una partición de $R$. Entonces:
    \vspace{-0.5em}
    \begin{enumerate}
        \item $R = \bigcup_{S \in P} S$
        \item Si $S, S' \in P$ y $S \neq S'$ entonces $S \cap S' = \emptyset$
        \item $\text{vol}(R) = \sum_{S \in P} \text{vol}(S)$
    \end{enumerate}
\end{lema}

\begin{proposición}
Sea $R \subset \R^n$ un n-rectángulo. Entonces $m^*(R) = \text{vol}(R)$.
\end{proposición}

\begin{proof}
    \leavevmode
    \begin{itemize}
        \item $m^*(R) \leq \text{vol}(R)$\\
              Consideremos la sucesión $(R_j)_{j \in \mathbb{N}}$ definida por $R_1 = R$ y $R_j$ degenerados para $j > 1$. Así,
              \[
                  m^*(R) \leq \sum_{j \in \mathbb{N}} \text{vol}(R_j) = \text{vol}(R_1) + \sum_{j=2}^{\infty} \text{vol}(R_j) = \text{vol}(R_1) = \text{vol}(R).
              \]
        \item $m^*(R) \geq \text{vol}(R)$\\
              Sea $\varepsilon > 0$. Por definición de medida exterior, existe una sucesión $(Q_j)_{j \in \mathbb{N}}$ de n-rectángulos abiertos con
              \[
                  R \subset \bigcup_{j \in \mathbb{N}} Q_j \quad \text{y} \quad \sum_{j \in \mathbb{N}} \text{vol}(Q_j) < m^*(R) + \varepsilon.
              \]
              Como $R$ es compacto (cerrado y acotado), el recubrimiento $\bigcup_{j \in
                      \mathbb{N}} Q_j$ admite un subrecubrimiento finito $\{Q_1, Q_2, \ldots, Q_m\}$
              que aún cubre $R$. Así,
              \[
                  R \subset \bigcup_{i=1}^m Q_i \subset \bigcup_{i=1}^m \overline{Q}_i.
              \]
              Definimos $R_j = R \cap \overline{Q}_j$ para $j = 1, \ldots, m$. Obtenemos
              entonces
              \[
                  R = \bigcup_{j=1}^m R_j.
              \]
              Prolongando los lados, podemos construir una partición $P$ de $R$ donde cada
              subrectángulo de $P$ quede contenido en algún $R_j$ con $1 \leq j \leq m$. Por
              tanto,
              \[
                  \text{vol}(R) = \sum_{S \in P} \text{vol}(S) \leq \sum_{j=1}^m \text{vol}(R_j) \leq \sum_{j=1}^m \text{vol}(Q_j) < m^*(R) + \varepsilon.
              \]
              Como $\varepsilon > 0$ es arbitrario, se concluye que $m^*(R) \geq
                  \text{vol}(R)$.
    \end{itemize}
\end{proof}

\subsection{Medida de Lebesgue en $\R^n$}

\textbf{Notación:} Para $A \subset \R^n$ denotamos por $A^c$ al complementario de $A$ en $\R^n$.

\begin{definición}[Conjunto medible\label{Conjunto Medible}]
Un conjunto $A \subset \R^n$ es medible en el sentido de Lebesgue si para todo $R \subset \R^n$ n-rectángulo se tiene que:
\[
    m^*(R) = m^*(R \cap A) + m^*(R \cap A^c)
\]
\end{definición}

\begin{proposición}
Sea $A \subset \R^n$ entonces son equivalentes:
\vspace{-0.5em}
\begin{enumerate}
    \item $A$ es medible en el sentido de Lebesgue.
    \item $\forall E \subset \R^n$ conjunto se tiene que $m^*(E) = m^*(E \cap A) + m^*(E \cap A^c)$.
    \item $\forall E \subset \R^n$ conjunto se tiene que $m^*(E) \geq m^*(E \cap A) + m^*(E \cap A^c)$.
\end{enumerate}
\label{prop:MedibleEquivalente}
\end{proposición}

\begin{proof}
    \leavevmode
    \begin{itemize}
        \item \textbf{$(2) \implies (3)$}\\
              Es inmediato.
        \item \textbf{$(3) \implies (2)$}\\
              Sabemos que $m^*(E) \leq m^*(E \cap A) + m^*(E \cap A^c)$. Para ver la otra desigualdad, observamos que
              \[
                  m^*(E) = m^*((E \cap A) \cup (E \cap A^c)) \leq m^*(E \cap A) + m^*(E \cap A^c).
              \]
              Así, la igualdad siempre se cumple.
        \item \textbf{$(2) \implies (1)$}\\
              Basta tomar $E = R$, un n-rectángulo.
        \item \textbf{$(1) \implies (3)$}\\
              Sea $E \subset \R^n$. Si $m^*(E) = +\infty$, el resultado es inmediato. Supongamos que $m^*(E) < +\infty$. Sea $\varepsilon > 0$. Por definición de medida exterior, existe una sucesión $(R_j)_{j \in \N}$ de n-rectángulos cerrados tal que
              \[
                  E \subset \bigcup_{j \in \N} R_j \quad \text{y} \quad \sum_{j \in \N} \text{vol}(R_j) < m^*(E) + \varepsilon.
              \]
              Entonces
              \[
                  E \cap A \subset \bigcup_{j \in \N} R_j \cap A, \quad E \cap A^c \subset \bigcup_{j \in \N} R_j \cap A^c.
              \]
              Por tanto,
              \[
                  m^*(E \cap A) + m^*(E \cap A^c) \leq \sum_{j \in \N} m^*(R_j \cap A) + \sum_{j \in \N} m^*(R_j \cap A^c).
              \]
              Por hipótesis $A$ es medible, luego $m^*(R_j) = m^*(R_j \cap A) + m^*(R_j \cap A^c)$ para
              cada $j$, y como $m^*(R_j) = \text{vol}(R_j)$:
              \[
                  m^*(E \cap A) + m^*(E \cap A^c) \leq \sum_{j \in \N} \text{vol}(R_j) < m^*(E) + \varepsilon.
              \]
              Por la arbitrariedad de $\varepsilon$, se concluye que $m^*(E) \geq m^*(E \cap A) +
                  m^*(E \cap A^c)$.
    \end{itemize}
\end{proof}

\begin{definición}[$\sigma$-Álgebra\label{sigma-Algebra}]
Sea $X$ un conjunto y $\mathcal{A} \subset \mathcal{P}(X)$ una colección de subconjuntos de $X$. Se dice que $\mathcal{A}$ es una $\sigma$-álgebra si:
\vspace{-0.5em}
\begin{enumerate}
    \item $X \in \mathcal{A}$
    \item Si $A \in \mathcal{A} \implies A^c \in \mathcal{A}$
    \item $\forall (A_j)_{j \in \N} \subset \mathcal{A}$ se tiene que $\bigcup_{j \in \N} A_j \in \mathcal{A}$
\end{enumerate}
\end{definición}

\label{Medida}
\begin{definición}[Medida\label{Medida}]
Sea $X$ un conjunto y $\mathcal{A} \subset \mathcal{P}(X)$ una $\sigma$-álgebra, entonces una medida en $X$ es una función $\mu: \mathcal{A} \to [0, +\infty]$ tal que:
\begin{enumerate}
    \item $\mu(\emptyset) = 0$
    \item Si $(A_j)_{j \in \N} \subset \mathcal{A}$ es una colección numerable de
          conjuntos disjuntos dos a dos entonces: $$\mu\left(\bigcup_{j \in \N} A_j\right) = \sum_{j
                  \in \N} \mu(A_j)$$
\end{enumerate}
\end{definición}

\begin{teorema}[Medida de Lebesgue en $\R^n$]
    La familia $M$ de todos los conjuntos medibles de $\R^n$ es una $\sigma$-álgebra y $m = m^* \restriction_M$ es una medida numerablemente aditiva que llamaremos medida de Lebesgue en $\R^n$.\label{Medida de Lebesgue en Rn}
\end{teorema}

Demostraremos este teorema con los siguientes lemas:

\begin{lema}
    El conjunto $\R^n$ es medible en el sentido de Lebesgue.
\end{lema}

\begin{proof}
    Dado cualquier conjunto $E \subset \R^n$, tenemos:
    \[
        E \cap \R^n = E,
        \quad \text{y} \quad
        E \cap (\R^n)^c = E \cap \emptyset = \emptyset.
    \]
    
    Entonces:
    \[
        m^*(E \cap \R^n) = m^*(E),
        \quad \text{y} \quad
        m^*(E \cap (\R^n)^c) = m^*(\emptyset) = 0.
    \]
    
    Sustituyendo en la condición de medibilidad, se tiene:
    \[
        m^*(E \cap \R^n) + m^*(E \cap (\R^n)^c) = m^*(E) + 0 = m^*(E).
    \]
    
    Por tanto, se cumple para todo $E \subset \R^n$ la igualdad requerida, lo que demuestra que $\R^n$ es medible en el sentido de Lebesgue.
\end{proof}


\begin{lema}
    Sea $A \subset \R^n$ medible en el sentido de Lebesgue. Entonces, su complementario $A^c$ también es medible en el sentido de Lebesgue.
    \label{lemaComplementarioMedible}
\end{lema}

\begin{proof}
    Por hipótesis, $A$ es medible en el sentido de Lebesgue, lo que significa que para todo conjunto $E \subset \R^n$ se cumple:
    \[
        m^*(E) = m^*(E \cap A) + m^*(E \cap A^c).
    \]
    
    Observemos que el complementario del complementario de $A$ es $A$, es decir:
    \[
        (A^c)^c = A.
    \]
    
    Tomemos ahora $A^c$ y veamos si para todo $E \subset \R^n$ se cumple:
    \[
        m^*(E) = m^*(E \cap A^c) + m^*(E \cap (A^c)^c).
    \]
    
    Sustituyendo la identidad $(A^c)^c = A$:
    \[
        m^*(E) = m^*(E \cap A^c) + m^*(E \cap (A^c)^c) = m^*(E \cap A^c) + m^*(E \cap A).
    \]
    Luego $A^c$ también es medible en el sentido de Lebesgue. 
\end{proof}

\begin{corolario}
    El conjunto vacío $\emptyset$ es medible en el sentido de Lebesgue.
    \label{corolarioVacioMedible}
\end{corolario}

\begin{lema}
    Sean $A, B \subset \R^n$ medibles en el sentido de Lebesgue. Entonces $A \cup B$ y $A \cap B$ son medibles en el sentido de Lebesgue.\label{unionMedible}
\end{lema}

\begin{proof}
    Observemos primero que
    \[
        A \cup B = (A^c \cap B) \cup (A \cap B) \cup (A \cap B^c)
    \]
    luego entonces tenemos que
    \[
        m^*(A \cup B) \leq m^*(A^c \cap B) + m^*(A \cap B) + m^*(A \cap B^c)
    \]

    Sea \( E \subset \R^n \) un conjunto, entonces por la medibilidad de \( A \)
    se sigue
    \[
        m^*(E) = m^*(E \cap A) + m^*(E \cap A^c)
    \]

    Además, sabemos que \( B \) es medible, luego para los conjuntos \( E \cap A^c
    \) y \(E \cap E\) se verifica
    \[
        \begin{matrix}
            m^*(E \cap A^c) = m^*(E \cap A^c \cap B) + m^*(E \cap A^c \cap B^c) \\
            m^*(E \cap A) = m^*(E \cap A \cap B) + m^*(E \cap A \cap B^c)
        \end{matrix}
    \]

    Por tanto,
    \[
        m^*(E) = m^*(E \cap A) + m^*(E \cap A^c) = m^*(E \cap A) + m^*(E \cap A^c \cap B) + m^*(E \cap A^c \cap B^c)
    \]
    \[
        = \underbrace{m^*(E \cap A \cap B) + m^*(E \cap A \cap B^c) + m^*(E \cap A^c \cap B)}_{\geq \ m^*(E \cap (A \cup B))} + \underbrace{m^*(E \cap A^c \cap B^c)}_{m^*(E \cap (A \cup B)^c)}
    \]

    Finalmente, observamos que
    \[
        m^*(E) \geq m^*(E \cap (A \cup B)) + m^*(E \cap (A \cup B)^c)
    \]
    y por tanto \( A \cup B \) es medible.\\ Nótese que la medibilidad de la
    intersección es inmediata, pues \( A \cap B = (A^c \cup B^c)^c \), y ya hemos
    demostrado por el \cref{lemaComplementarioMedible} que el complementario de un conjunto medible es medible.
\end{proof}

\begin{lema}
    Sea $(A_j)_{j \in \N} \subset \R^n$ una colección numerable de conjuntos disjuntos medibles en el sentido de Lebesgue. Entonces $\bigcup_{j \in \N} A_j$ es medible en el sentido de Lebesgue y además $m^*(\bigcup_{j \in \N} A_j) = \sum_{j \in \N} m^*(A_j)$.
    \label{unionNumerableMedible}
\end{lema}

\begin{proof}
    Definimos la sucesión creciente de conjuntos \( B_k = A_1 \cup \ldots \cup A_k \). Entonces por el \cref{unionMedible}, \( B_k \) es medible en el sentido de Lebesgue. Sean \( B = \bigcup_{k \in \N} B_k = \bigcup_{j \in \N} A_j \) y \( E \subset \R^n \). Por la medibilidad de $A_k$, tenemos:
    \[
        m^*(E \cap B_k) = m^*(E \cap B_k \cap A_k) + m^*(E \cap B_k \cap A_k^c) = m^*(E \cap A_k) + m^*(E \cap B_{k-1})
    \]
    Nótese que $A_k^c = B_{k-1}$ precisamente porque los conjuntos $A_j$ son
    disjuntos. Reiterando el proceso, obtenemos:
    \[
        m^*(E \cap B_k) = \sum_{j=1}^k m^*(E \cap A_j)
    \]
    Por lo tanto, aplicando la mediblidad de \( B_k \):
    \[
        m^*(E) = m^*(E \cap B_k) + m^*(E \cap B_k^c) = \left( \sum_{j=1}^k m^*(E \cap A_j) \right) + m^*(E \cap B_k^c) \geq \sum_{j=1}^k m^*(E \cap A_j) + m^*(E \cap B^c)
    \]
    Se sigue entonces:
    \[
        m^*(E) \geq \sum_{j \in \N} m^*(E \cap A_j) + m^*(E \cap B^c) \geq m^*\left( \bigcup_{j \in \N} E \cap A_j \right) + m^*(E \cap B^c) \geq m^*(E \cap B) + m^*(E \cap B^c)
    \]
    Luego, \( B \) es medible.

    Tomando \( E = B \) en la desigualdad anterior, obtenemos:
    \[
        m^*\left( \bigcup_{j \in \N} A_j \right) = m^*(B) \geq \sum_{j \in \N} m^*(B \cap A_j) + m^*(B \cap B^c) = \sum_{j \in \N} m^*(B \cap A_j) = \sum_{j \in \N} m^*(A_j)
    \]
    Por otro lado, por el apartado 2 del \cref{funcionMedida} sabemos que la medida
    exterior de la union numerable de conjuntos es menor o igual que la suma de las
    medidas exteriores de los conjuntos:
    \[
        m^*\left( \bigcup_{j \in \N} A_j \right) \leq \sum_{j \in \N} m^*(A_j)
    \]
    Por tanto,
    \[
        m^*\left( \bigcup_{j \in \N} A_j \right) = \sum_{j \in \N} m^*(A_j)
    \]
\end{proof}

\begin{lema}
    La unión numerable de conjuntos medibles en el sentido de Lebesgue es un conjunto medible en el sentido de Lebesgue.
\end{lema}

\begin{proof}
    Sea $(B_j)_{j \in \N}$ una colección numerable de conjuntos medibles en el sentido de Lebesgue. Consideramos:
    \[\begin{matrix}
            A_1 = B_1                       \\
            A_2 = B_2 \cap B_1^c            \\
            A_3 = B_3 \cap B_2^c \cap B_1^c \\
            \vdots                          \\
            A_j = B_j \cap B_{j-1}^c \cap \ldots \cap B_1^c
        \end{matrix}\]
    Observemos que $\bigcup_{j \in \N} A_j = \bigcup_{j \in \N} B_j$ y que para
    todo $j \in \N$, $A_j$ es intersección finita de conjuntos medibles, por tanto,
    $A_j$ es medible. Además, $\forall i,j \in \N$ con $i \neq j$, $A_i \cap A_j =
        \emptyset$. Por el \cref{unionNumerableMedible}, $\bigcup_{j \in \N} A_j$ es
    medible $\implies \bigcup_{j \in \N} B_j$ es medible.
\end{proof}

\begin{proposición}
    Todo conjunto nulo es medible en el sentido de Lebesgue.
\end{proposición}

\begin{proof}
    Sea $A \subset \R^n$ un conjunto nulo, es decir, $m^*(A) = 0$.

    Tomemos un conjunto arbitrario $E \subset \R^n$. Observemos que:
    \[
        E \cap A \subset A \implies m^*(E \cap A) \leq m^*(A) = 0.
    \]
    Además, como la medida exterior siempre es no negativa, tenemos:
    \[
        m^*(E \cap A) \geq 0.
    \]
    Por lo tanto:
    \[
        m^*(E \cap A) = 0.
    \]

    Por otro lado, claramente:
    \[
        E \cap A^c \subset E \implies m^*(E \cap A^c) \leq m^*(E).
    \]

    Así, sumando las desigualdades obtenidas:
    \[
        m^*(E \cap A) + m^*(E \cap A^c) = 0 + m^*(E \cap A^c) \leq m^*(E).
    \]
    Por la \cref{prop:MedibleEquivalente} se concluye la medibilidad de $A$.
\end{proof}

Con esto, damos por concluida la demostración del \cref{Medida de Lebesgue en Rn}.


\begin{definición}[Propiedad en casi todo punto\label{Propiedad en casi todo punto}]
Se dice que una \textit{propiedad} se verifica en casi todo punto cuando el conjunto de puntos en los que no se verifica la propiedad es un conjunto nulo.
\end{definición}

\begin{proposición}
Todo n-rectángulo cerrado $R \subset \R^n$ es medible en el sentido de Lebesgue.
\end{proposición}

\begin{proof}
    Dado $R \subset \R^n$ n-rectángulo cerrado, tenemos que ver que $\forall Q \subset \R^n$ n-rectángulo cerrado se tiene que $\text{vol}(Q) \geq m^*(Q \cap R) + m^*(Q \cap R^c)$. Consideramos el n-rectángulo $Q_0 = Q \cap R$. Nótese que $Q \cap R^c$ es unión finita de n-rectángulos $\{Q_1,\ldots, Q_m\}$. Entonces $Q = Q_0 \cup Q_1 \cup \ldots \cup Q_m$ forman una partición de $Q$. Luego $\text{vol}(Q) = \sum_{i=0}^m \text{vol}(Q_i) = m^*(Q \cap R) + \sum_{i=1}^m m^*(Q_i) \geq m^*(Q \cap R) + m^*(Q \cap R^c)$.
\end{proof}

\begin{observación}
En $\R^n$ los rectángulos abiertos son medibles en el sentido de Lebesgue.
\end{observación}

\begin{definición}[n-Cubo\label{n-Cubo}]
Un n-cubo cerrado (respectivamente abierto) en $\R^n$ es un conjunto de la forma:
\[
    R = [a_1, b_1] \times \ldots \times [a_n, b_n] \text{ tal que } \forall i,j \in \{1,2,...,n\} \text{ se tiene que } b_i - a_i = b_j - a_j
\]
Análogamente se pueden definir los cubos n-dimensionales semi-abiertos.
\end{definición}

\observacion{
    Denotaremos la norma del supremo en $\rn$ como:
    \[
        \lVert x \rVert _{\infty} = \sup_{i=1}^n \{|x_i|\} \text{ para } x = (x_1, x_2, ..., x_n) \in \rn
    \]
    Llamaremos bola abierta de centro $x \in \rn$ y radio $r > 0$ al conjunto:
    \[
        B_{\infty}(x, r) = \{y \in \rn :  \lVert y - x \rVert _{\infty} < r\} \equiv (x_1 - r, x_1 + r) \times \ldots \times (x_n - r, x_n + r)
    \]
    Análogamente, llamaremos bola cerrada de centro $x \in \rn$ y radio $r > 0$ al conjunto:
    \[
        \overline{B}_{\infty}(x, r) = \{y \in \rn :  \lVert y - x \rVert _{\infty} \leq r\} \equiv [x_1 - r, x_1 + r] \times \ldots \times [x_n - r, x_n + r]
    \]
}

\begin{teorema}
    Sea $G \subset \mathbb{R}^n$ abierto. Entonces:
    \vspace{-0.5em}
    \begin{enumerate}
        \item $G$ es la unión numerable de $n$-cubos cerrados.
        \item $G$ es la unión numerable de $n$-cubos abiertos.
    \end{enumerate}
    \label{teoremaCubosAbiertosCerrados}
\end{teorema}

\begin{proof}
    Vamos a demostrar la primera parte; la segunda es análoga.

    Consideremos la familia de $n$-cubos cerrados (o bolas en norma infinito):
    \[
        \mathcal{B} = \left\{ \overline{B}_\infty(q, r) : q \in \mathbb{Q}^n, \, r \in \mathbb{Q}, \, r > 0, \, \overline{B}_\infty(q, r) \subset G \right\}.
    \]
    Queremos ver que:
    \[
        G = \bigcup_{B \in \mathcal{B}} B.
    \]

    \begin{itemize}
        \item La inclusión \(\bigcup_{B \in \mathcal{B}} B \subset G\) es clara, porque todos los cubos de \(\mathcal{B}\) están contenidos en \(G\) por construcción.
        \item Veamos la inclusión inversa \(G \subset \bigcup_{B \in \mathcal{B}} B\).

    Sea \(x \in G\). Como \(G\) es abierto, existe \(\delta > 0\) tal que:
    \[
        B_\infty(x, \delta) \subset G.
    \]

    Elegimos \(r \in \mathbb{Q}\) con \(0 < r < \frac{\delta}{2}\). Por la densidad de \(\mathbb{Q}^n\) en \(\mathbb{R}^n\), existe \(q \in \mathbb{Q}^n\) tal que:
    \[
        \|x - q\|_\infty < r.
    \]

    Ahora, consideremos el cubo cerrado \(\overline{B}_\infty(q, r)\). Veamos que:
    \[
        \overline{B}_\infty(q, r) \subset B_\infty(x, \delta).
    \]
    En efecto, para todo \(y \in \overline{B}_\infty(q, r)\):
    \[
        \|y - x\|_\infty \leq \|y - q\|_\infty + \|q - x\|_\infty < r + r = 2r < \delta.
    \]
    Por lo tanto:
    \[
        \overline{B}_\infty(q, r) \subset B_\infty(x, \delta) \subset G.
    \]

    Además, \(q \in \mathbb{Q}^n\) y \(r \in \mathbb{Q}^+\), por lo que \(\overline{B}_\infty(q, r) \in \mathcal{B}\) y \(x \in \overline{B}_\infty(q, r)\).

    Como \(x \in G\) es arbitrario, se concluye que:
    \[
        G \subset \bigcup_{B \in \mathcal{B}} B.
    \]
    \end{itemize}
    
    Así:
    \[
        G = \bigcup_{B \in \mathcal{B}} B.
    \]
    La numerabilidad de la familia $\mathcal{B}$ se debe a cómo se construyen sus elementos. Cada cubo cerrado en $\mathcal{B}$ está definido por un centro $q \in \Q^n$ y un radio $r \in \Q$, $r>0$, de manera que $\overline{B}_\infty(q, r) \subset G$. Dado que $\Q^n$ es numerable (ya que es producto finito de conjuntos numerables) y $\Q$ también lo es, el conjunto de pares $(q, r)$ donde $q \in \Q^n$ y $r \in \Q$, $r>0$, es numerable. Esto se debe a que el producto de conjuntos numerables es numerable:
    \[
        \Q^n \times \Q^+ \quad \text{es numerable}.
    \]
    Por lo tanto, la familia de todos los cubos cerrados con centro racional y radio racional positivo también es numerable:
    \[
        \{ \overline{B}_\infty(q, r) : q \in \Q^n, r \in \Q, r>0 \}.
    \]
    La familia $\mathcal{B}$ es simplemente un subconjunto de esa colección de cubos cerrados, es decir, $\mathcal{B} \subset \{ \overline{B}_\infty(q, r) : q \in \Q^n, r \in \Q, r>0 \}$, y como todo subconjunto de un conjunto numerable sigue siendo numerable, concluimos que $\mathcal{B}$ es numerable.
\end{proof}


\begin{corolario}
    Todos los conjuntos abiertos y cerrados de $\R^n$ son medibles en el sentido de Lebesgue.
    \label{abiertoMedible}
\end{corolario}

\begin{teorema} [Regularidad de la medida\label{Regularidad de la medida}]
    Sea $E \subset \mathbb{R}^n$, entonces son equivalentes:
    \begin{enumerate}
        \item $E$ es medible en el sentido de Lebesgue.
        \item $\forall \varepsilon > 0 \quad \exists G \subset \mathbb{R}^n$ abierto tal que $E \subset G$ y $m^*(G \setminus E) < \varepsilon$.
        \item $\forall \varepsilon > 0 \quad \exists F \subset \mathbb{R}^n$ cerrado tal que $F \subset E$ y $m^*(E \setminus F) < \varepsilon$.
        \item $\forall \varepsilon$ existen $F$ cerrado y $G$ abierto tales que $F \subset E \subset G$ y $m^*(G \setminus F) < \varepsilon$.
    \end{enumerate}
\end{teorema}

\begin{proof}
    \leavevmode
    \begin{itemize}
        \item $(1) \implies (2)$\\
              Distinción de casos:
              \begin{enumerate}
                  \item Supongamos que $m^*(E) < +\infty$: Sea $\varepsilon > 0$. Por definición de
                        medida exterior, $\exists (R_j)_{j \in \N}$ sucesión de n-rectángulos abiertos
                        tales que $E \subset \unj(R_j)$ y $\sum_{j \in \N} \text{vol}(R_j) < m^*(E) +
                            \varepsilon$. Considerando el abierto $G = \unj(R_j)$, se tiene que $G$ es
                        medible por el \cref{abiertoMedible}, además, como $E \subset G$ entonces
                        $$m^*(G) = m^*(\underbrace{G \cap E}_{E}) + m^*(\underbrace{G \cap E^c}_{G
                                \setminus E}) = m^*(E) + m^*(G \setminus E)$$ Por tanto, $$m^*(G \setminus E) =
                            m^*(G) - m^*(E) < \sum_{j \in \N} \text{vol}(R_j) - m^*(E) < \varepsilon$$
                  \item Supongamos que $m^*(E) = +\infty$: $\forall k \in \N$ sea $E_k = E \cap
                            [-k,k]^n$, que es medible por ser intersección finita de conjuntos medibles.
                        Además $m^*(E_k) < +\infty$ por ser $E_k$ acotado, y $E = \unk{E_k}$. Luego por
                        el apartado anterior, dado $\varepsilon > 0$, $\forall k \in \N$ existe $G_k$
                        abierto tal que $E_k \subset G_k$ y $m^*(G_k \setminus E_k) <
                            \frac{\varepsilon}{2^k}$.\\
                            Entonces $G = \unk{G_k}$ abierto y $E = \unk{E_k}
                            \subset \unk{G_k} = G$ por lo que $$m^*(G \setminus E) \leq m^*\left(\unk(G_k
                            \setminus E_k)\right) \leq \sk{m^*(G_k \setminus E_k)} < \sk{\frac{\varepsilon}{2^k}}
                            = \varepsilon$$
              \end{enumerate}
        \item $(2) \implies (1)$\\
        Sea \( j \in \N \). Tomemos \(\varepsilon = \frac{1}{j}\). Por hipótesis, existe un conjunto abierto \( G_j \) tal que \( E \subset G_j \) y \( m^*(G_j \setminus E) < \frac{1}{j} \). Consideremos \( B = \intj{G_j} \), que es medible y abierto, y cumple \( E \subset B \).

        Además, para cada \( j \in \N \), se tiene que \( B \setminus E \subset G_j \setminus E \). Así,
        \[
        m^*(B \setminus E) \leq m^*(G_j \setminus E) < \frac{1}{j}.
        \]
        Por lo tanto, \( m^*(B \setminus E) = 0 \), lo que implica que \( B \setminus E \) es medible.
        
        Por otro lado, dado que \( B = E \cup (B \setminus E) \), podemos escribir \( E = B \setminus (B \setminus E) \). Como tanto \( B \) como \( B \setminus E \) son medibles, se deduce que \( E \) es medible.
        
        Finalmente, \( E \) se puede expresar como la diferencia \( E = B \setminus Z \), donde \( B \) es la intersección numerable de abiertos y \( Z \) es un conjunto nulo.
        
        \item (1) $\implies$ (3)\\
              Como E es medible entonces tenemos que $E^c$ también es medible, por lo que, dado $\varepsilon > 0$ por (2) $\exists G$-abierto tal que $E^c \subset G$ y $m^*(G \setminus E^c) < \varepsilon$. Entonces $F = G^c$ es cerrado y $F \subset E$. Además, 
              $$E \setminus F = E \cap F^c = E \cap G = G \setminus E^c \implies m^*(E \setminus F) = m^*(G \setminus E^c) < \varepsilon$$
        \item (3) $\implies$ (1)\\
        Para todo \( j \in \mathbb{N} \), existe un conjunto cerrado \( F_j \) tal que \( F_j \subset E \) y \( m(E \setminus F_j) < \frac{1}{j} \). Sea 
        \[
        A = \bigcup_{j=1}^{\infty} F_j,
        \]
        que es un conjunto medible y satisface \( A \subset E \). Además, dado que
        \[
        m(E \setminus A) \leq m(E \setminus F_j) < \frac{1}{j} \quad \forall j \in \mathbb{N},
        \]
        se concluye que \( m(E \setminus A) = 0 \).
        
        Por lo tanto, se tiene
        \[
        E = A \cup (E \setminus A) = \left( \bigcup_{j=1}^{\infty} F_j \right) \cup (E \setminus A).
        \]
        Dado que \( E \setminus A \) es un conjunto nulo y, por lo tanto, medible, y que \( \bigcup_{j=1}^{\infty} F_j \) es medible por ser la unión numerable de conjuntos cerrados, se concluye que \( E \) es medible.\\
        Observemos que en este caso $E = A \cup N$, siendo $A$ unión numerable de cerrados y $N = E \cup A$ un conjunto nulo.        
    \end{itemize}
\end{proof}

\begin{definición} [$\sigma$-Álgebra de Borel\label{sigma-Algebra de Borel}]
La $\sigma$-álgebra de Borel en $\rn$ se define como la menor $\sigma$-álgebra que contiene a todos los abiertos de $\rn$ (o equivalentemente, la menor $\sigma$-álgebra que contiene a todos los cerrados de $\rn$). Los conjuntos de $\mathcal{B}(\rn)$ se llaman conjuntos de Borel o conjuntos Borelianos.\\
\end{definición}

\begin{definición} [Conjuntos $G_{\delta}$ y $F_{\sigma}$]
Decimos que $A \subset \rn$ es $G_\delta$ si $A$ es intersección numerable de abiertos. Análogamente, decimos que un conjunto $B \subset \rn$ es $F_\sigma$ si $B$ es unión numerable de cerrados.
\end{definición}

\begin{corolario}
    Sea $E \subset \rn$, entonces son equivalentes:
    \vspace{-0.5em}
    \begin{enumerate}
        \item $E$ es medible en el sentido de Lebesgue.
        \item $E = A \setminus N$ con $A$ siendo $G_\delta$ y $N$ un conjunto nulo.
        \item $E = B \cup N$ con $B$ siendo $F_\sigma$ y $N$ un conjunto nulo.
    \end{enumerate}
\end{corolario}

\begin{lema}
    Sea \((A_j)_{j \in \N}\) una familia numerable y creciente de conjuntos medibles en el sentido de Lebesgue. Entonces, \(\bigcup_{j \in \N} A_j\) es medible en el sentido de Lebesgue y
    \[
    m^*\left(\bigcup_{j \in \N} A_j\right) = \lim_{j \to \infty} m^*(A_j).
    \]
    \label{limUnionMedible}
\end{lema}

\begin{proof}
    Sea \((A_j)_{j \in \N}\) una sucesión creciente de conjuntos medibles en el sentido de Lebesgue. Definimos la sucesión \((B_j)_{j \in \N}\) como
    \[\begin{matrix}
        A_1 = B_1                       \\
        A_2 = B_2 \cap B_1^c            \\
        A_3 = B_3 \cap B_2^c \cap B_1^c \\
        \vdots                          \\
        A_j = B_j \cap B_{j-1}^c \cap \ldots \cap B_1^c
    \end{matrix}\]
    De esta forma, la unión
    \[
    \bigcup_{j=1}^\infty B_j = \bigcup_{j=1}^\infty A_j
    \]
    es una unión disjunta de conjuntos. Por lo tanto, y usando el \cref{unionNumerableMedible}, se tiene que
    \[
    m^*\left(\bigcup_{j=1}^\infty A_j\right) = m^*\left(\bigcup_{j=1}^\infty B_j\right) = \sum_{j=1}^\infty m^*(B_j) = \lim_{k \to \infty} m(A_k),
    \]
    donde la última igualdad se debe a que para todo \( j \in \N \) se cumple
    \[
    m^*(A_j) = \sum_{i=1}^j m^*(B_i).
    \]
\end{proof}


\begin{corolario}
    Sea $E \subset \rn$ medible entonces:
    \vspace{-0.5em}
    \begin{enumerate}
        \item $m^*(E) = \text{inf}\{m^*(G) : G \text{ abierto y } E \subset G\}$.
        \item $m^*(E) = \text{sup}\{m^*(K) : K \text{ compacto y } K \subset E\}$.
    \end{enumerate}
\end{corolario}

\begin{proof}
    \leavevmode
    \begin{enumerate}
        \item Dado $\varepsilon > 0$ por el \cref{Regularidad de la medida} existe $G$
              abierto tal que $E \subset G$ y $m(G \setminus E) < \varepsilon$. Entonces usando la medibilidad de $E$ deducimos:
              $$m^*(E) \leq m^*(G) = m^*(\underbrace{G \cap E}_{E}) + m^*(G \setminus E) < m^*(E) + \varepsilon$$
              Por tanto,
              $m^*(E) = \text{inf}\{m^*(G) : G \text{ abierto y } E \subset G\}$.
        \item Sea \(E = \bigcup_{k=1}^\infty E_k\), donde definimos \(E_k = E \cap [-k, k]^n\) para cada \(k \in \mathbb{N}\). Entonces, \((E_k)_k\) es una sucesión creciente de conjuntos medibles y, por el \cref{limUnionMedible}, se cumple que
        \[
        m^*(E) = \lim_{k \to \infty} m^*(E_k).
        \]
        Además, por el \cref{Regularidad de la medida} existe un conjunto cerrado \(F_k \subset E_k\) para cada \(k \in \mathbb{N}\) tal que
        \[
        m(E_k \setminus F_k) < \frac{1}{k}.
        \]
        Como \(F_k\) es cerrado y está contenido en el cubo compacto \([-k,k]^n\), entonces \(F_k\) es compacto. En particular $F_k$ es medible por ser cerrado (\cref{abiertoMedible}), luego 
        \[
        m^*(E_k) = m^*(\underbrace{E_k \cap F_k}_{F_k}) + m^*(E_k \setminus F_k) \leq m^*(F_k) + \frac{1}{k}
        \]
        Al tomar el límite cuando \(k \to \infty\), obtenemos
        \[
        m^*(E) = \lim_{k \to \infty} m^*(E_k) = \lim_{k \to \infty} m^*(F_k),
        \]
        y finalmente concluimos que
        \[
        m^*(E) = \sup\{ m^*(F_k) : k \in \mathbb{N} \} = \sup \{ m^*(K) : K \text{ compacto y } K \subset E \}.
        \]        
    \end{enumerate}
\end{proof}

\begin{definición}[Cubo diádico\label{Cubo Diadico}]
Se dice que un cubo en $\R^n$ es diádico si sus lados miden $2^{-m}$ para algún $m \in \N$.
Es decir, si el rectángulo Q es de la forma:
\[Q = \left[\frac{k_1}{2^m}, \frac{k_1 + 1}{2^m}\right] \times \dots \times \left[\frac{k_n}{2^m}, \frac{k_n + 1}{2^m}\right],\]
con $m \in \mathbb{Z} \text{ (nivel de escala u orden) y } k_1, k_2, \dots k_n
    \in \mathbb{Z}$
\end{definición}

\begin{teorema}
    Todo conjunto abierto $U$ de $\mathbb{R}^n$ es unión numerable y disjunta de cubos diádicos.
\end{teorema}

\begin{proof}
    Denotemos por $\mathcal{F}$ la familia de todos los cubos cerrados de la forma
    \[
        \left[\frac{k_1}{2^m}, \frac{k_1 + 1}{2^m}\right] \times \dots \times \left[\frac{k_n}{2^m}, \frac{k_n + 1}{2^m}\right],
    \]
    donde $k_i \in \mathbb{Z}$ y $m \in \mathbb{N}$.  
    
    Sea $\mathcal{Q}_1$ la colección de todos los cubos cerrados de la forma
    \[
        [k_1, k_1 + 1] \times \dots \times [k_n, k_n + 1],
    \]
    con $k_i \in \mathbb{Z}$, que además satisfacen $Q \subset U$.  
    
    Supongamos definida $\mathcal{Q}_m$. Construimos $\mathcal{Q}_{m+1}$ como la familia de todos los cubos cerrados de la forma
    \[
        \left[\frac{k_1}{2^{m}}, \frac{k_1+1}{2^{m}}\right] \times \dots \times \left[\frac{k_n}{2^{m}}, \frac{k_n+1}{2^{m}}\right],
    \]
    donde $k_i \in \mathbb{Z}$, que están contenidos en $U$ y que no están incluidos en ningún cubo de $\mathcal{Q}_j$ para $j \leq m$.  
    
    Por inducción, hemos definido así las familias $\mathcal{Q}_m$ para todo $m \in \mathbb{N}$. Definimos
    \[
        \mathcal{Q} = \bigcup_{m=1}^{\infty} \mathcal{Q}_m.
    \]
    
    Por construcción, los cubos de $\mathcal{Q}$ tienen interiores disjuntos: si $Q, Q' \in \mathcal{Q}$ y $Q \neq Q'$, entonces $\operatorname{int}(Q) \cap \operatorname{int}(Q') = \emptyset$. Además, es claro que
    \[
        \bigcup_{Q \in \mathcal{Q}} Q \subset U.
    \]
    
    Veamos ahora que en realidad
    \[
        U = \bigcup_{Q \in \mathcal{Q}} Q.
    \]
    
    Sea $x \in U$. Dado que $U$ es abierto, existe $\delta > 0$ tal que la bola $B(x, \delta) \subset U$. Como el conjunto de números de la forma $k/2^m$, con $k \in \mathbb{Z}$ y $m \in \mathbb{N} \cup \{0\}$, es denso en $\mathbb{R}$, podemos encontrar un cubo cerrado $Q_x \in \mathcal{F}$ que contiene a $x$ y está contenido en $B(x, \delta) \subset U$.  
    
    El lado de $Q_x$ es $2^{-m_x}$ para algún $m_x \in \mathbb{N} \cup \{0\}$. Si $Q_x \in \mathcal{Q}_{m_x}$, entonces $x$ pertenece a un cubo de $\mathcal{Q}$. Si no es así, por definición de $\mathcal{Q}_{m_x}$, existe algún $j < m_x$ y un cubo $Q_x' \in \mathcal{Q}_j$ que contiene a $Q_x$, y en particular a $x$.  
    
    En ambos casos, vemos que $x \in \bigcup_{Q \in \mathcal{Q}} Q$, lo que muestra que
    \[
        U = \bigcup_{Q \in \mathcal{Q}} Q.
    \]
    Así, $U$ es la unión numerable y disjunta de los cubos diádicos de $\mathcal{Q}$.
    \end{proof}
    
