
\section{Medida de Lebesgue}

\subsection{Medida Exterior de Lebesgue en $\R^n$}

\begin{definición}[n-Réctangulo\label{n-rectángulo}]
Un n-rectángulo en $\R^n$ es un conjunto de la forma:
\begin{equation}
    R = \prod_{i=1}^n [a_i, \ b_i] = [a_1, \ b_1] \times [a_2, \ b_2] \times ... \times [a_n, \ b_n] \text{ donde } a_i \leq b_i \ \forall i
\end{equation}
Definimos el volúmen de $R$ como:
\begin{equation}
    \text{vol}(R) = \prod_{i=1}^n (b_i - a_i)
\end{equation}
Consideramos también los n-rectángulos abiertos denotados por $\mathring{R}$, que se definen de forma análoga. Si nos se especifica si un rectángulo es abierto o cerrado, se asume que es cerrado.
\end{definición}

\begin{observación}
Dado $R$ n-rectángulo cerrado tal que $R = \prod_{i=1}^n [a_i, \ b_i]$, podemos considerar para cada $\delta > 0$ el n-rectángulo abierto $R_\delta = \prod_{i=1}^n (a_i - \delta, \ b_i + \delta)$. Se tiene que $R \subset R_\delta$ y $\text{vol}(R_\delta) = \prod_{i=1}^n (b_i - a_i + 2\delta) = \text{vol}(R) + 2n\delta$. Por tanto:
\begin{equation}
    \text{vol}(R) = \lim_{\delta \to 0} \text{vol}(R_\delta)
\end{equation}
\end{observación}

\begin{definición}[Medida Exterior de Lebesgue\label{Medida de exterior de Lebesgue}]
Sea $A \subset \R^n$. Definimos la medida exterior de $A$ como:
\begin{equation}
    m^*(A) = \inf \left\{ \sum_{i=1}^\infty \text{vol}(R_i) \ | \ A \subset \bigcup_{i=1}^\infty R_i \text{ con } R_i \text{ n-rectángulos cerrados} \right\}
\end{equation}
Donde el ínfimo se toma sobre todas las colecciones numerables de n-rectángulos que recubren $A$. A esta medida exterior la llamamos medida de Lebesgue exterior.
\end{definición}

\begin{observación}
Sea $A \subset \R^n$ entonces:
\vspace{-0.5em}
\begin{enumerate}
    \item $m^*(A) = +\infty \iff \forall (R_j)_{j \in J} \text{ tal que } A \subset \bigcup_{j \in J} R_j \text{ se tiene que } \sum_{j \in J} \text{vol}(R_j) = +\infty$
    \item $m^*(A) = 0 \iff \forall \epsilon > 0 \ \exists (R_j)_{j \in J} \text{ tal que } A \subset \bigcup_{j \ in J} R_j \text{ y } \sum_{j \in J} \text{vol}(R_j) < \epsilon$
    \item $m^*(A) = \alpha \in \R^+ \iff \forall \epsilon > 0 \ \exists (R_j)_{j \in J} \text{ tal que } A \subset \bigcup_{j \in J} R_j \text{ y } \sum_{j \ in J} \text{vol}(R_j) < \alpha + \epsilon$
\end{enumerate}
\end{observación}

\begin{definición}[Conjunto Nulo\label{Conjunto nulo}]
Se dice que $A \subset \R^n$ es un conjunto nulo si $m^*(A) = 0$.
\end{definición}

\ejemplo{
    \begin{enumerate}
        \item Si $R$ es un n-rectángulo degenerado, es decir, $R$ tiene alguno de los lados
              de longitud 0, entonces $R$ es un conjunto nulo ($m^*(R) = 0$).
        \item En $\mathbb{R}^2$, sea el conjunto $A = \{(x,x) : 0 \leq x \leq 1\}$. Dado
              $\epsilon > 0$ tomamos $m \in \mathbb{N}$ tal que $m > \frac{1}{\epsilon}$.
              Consideramos $A \subset \bigcup_{i=1}^m [\frac{i-1}{m}, \frac{i}{m}] \times
                  [\frac{i-1}{m}, \frac{i}{m}]$. Se tiene que $m^*(A) \leq \sum_{i=1}^m
                  \text{vol}([\frac{i-1}{m}, \frac{i}{m}] \times [\frac{i-1}{m}, \frac{i}{m}]) =
                  \frac{1}{m^2} \cdot m = \frac{1}{m} < \epsilon$. Por tanto, $m^*(A) = 0$.
    \end{enumerate}
}

Denotamos por $\mathcal{P}(\R^n)$ al conjunto de todos los subconjuntos de
$\R^n$.
\begin{teorema}
    La función $m^* : \mathcal{P}(\R^n) \to [0, +\infty]$ satisface:
    \vspace{-0.5em}
    \begin{enumerate}
        \item $m^*(\emptyset) = 0$
        \item $m^*(A) \leq m^*(B)$ si $A \subset B$
        \item $m^*(\bigcup_{i=1}^\infty A_i) \leq \sum_{i=1}^\infty m^*(A_i)$
    \end{enumerate}
    \label{funcionMedida}
\end{teorema}

\begin{proof}
    % add a space between the proof text and the enumerate
    \leavevmode
    \begin{enumerate}
        \item $\emptyset \subset \bigcup_{i=1}^\infty R_j$ con $R_j$ n-rectángulos degenerados $\implies m^*(\emptyset) \leq \sum_{j=1}^\infty \text{vol}(R_j) = 0 \implies m^*(\emptyset) = 0$.
        \item Sea $A \subset B$ y sea $(R_j)_{j \in J}$ tal que $B \subset \bigcup_{j \in J}
                  R_j$. Entonces $(R_j)_{j \in J}$ es un recubrimiento de $A$ y por tanto $m^*(A)
                  \leq \sum_{j \in J} \text{vol}(R_j) \implies m^*(A) \leq m^*(B)$.
        \item Si $\sum_{j=1}^{\infty}{m^*(A_j)} = +\infty$ entonces el resultado es
              inmediato. Supongamos que $\sum_{j=1}^{\infty}{m^*(A_j)} < +\infty$. Sea
              $\epsilon > 0$. Para cada $j \in \mathbb{N}$, $\exists (R_{j,i})_{i =
                      1}^\infty$ tal que $A_j \subset \bigcup_{i = 1}^\infty R_{j,i}$ y $\sum_{i =
                      1}^\infty \text{vol}(R_{j,i}) < m^*(A_j) + \frac{\epsilon}{2^j}$. Entonces
              $\bigcup_{j=1}^{\infty} A_j \subset \bigcup_{j=1}^{\infty} \bigcup_{i =
                      1}^\infty R_{j,i}$ y por tanto se tiene que $m^*(\bigcup_{j=1}^{\infty} A_j)
                  \leq \sum_{j=1}^{\infty} \sum_{i = 1}^\infty \text{vol}(R_{j,i}) <
                  \sum_{j=1}^{\infty} (m^*(A_j) + \frac{\epsilon}{2^j})$ = $\sum_{j=1}^{\infty}
                  m^*(A_j) + \epsilon$. Como $\epsilon$ es arbitrario, se tiene que
              $m^*(\bigcup_{j=1}^{\infty} A_j) \leq \sum_{j=1}^{\infty} m^*(A_j)$.
    \end{enumerate}
\end{proof}

\begin{corolario}
    La unión numerable de conjuntos nulos es un conjunto nulo.
\end{corolario}

\begin{proof}
    Sea $(A_j)_{j=1}^\infty \subset R^n$ tal que $m^*(A_j) = 0 \quad \forall j \in \mathbb{N}$ entonces $m^*(\bigcup_{j=1}^\infty A_j) \leq \sum_{j=1}^\infty m^*(A_j) = 0 \implies m^*(\bigcup_{j=1}^\infty A_j) = 0$.
\end{proof}

\begin{lema}
    Sea $A \in \R^n$ entonces $m^*(A) = \inf \left\{ \sum_{i=1}^\infty \text{vol}(Q_i) \ | \ A \subset \bigcup_{i=1}^\infty Q_i \text{ con } Q_i \text{ n-rectángulos abiertos} \right\}$
\end{lema}

\begin{proof}
    Denotamos por $\beta$ el ínfimo de la expresión del enunciado del lema. Sea $(Q_j)_{j \in \N}$ una sucesión de rectángulos abiertos tal que $A \subset \bigcup_{j \in \N} Q_j$. Tenemos entonces que $A \subset \bigcup_{j \in \N} Q_j \subset \bigcup_{j \in \N} \overline{Q}_j$ y puesto que $\sum_{j \in \N} \text{vol}(\overline{Q}_j) = \sum_{j \in \N} \text{vol}(Q_j)$, se tiene que $m^*(A) \leq \beta$. Veamos ahora la otra desigualdad $\beta \leq m^*(A)$. Si $m^*(A) = +\infty$ entonces $\beta = +\infty$ y no hay nada que demostrar. Supongamos que $m^*(A) < +\infty$. Sea $\epsilon > 0$. Por definición de medida exterior, $\exists (R_j)_{j \in \N}$ sucesión de n-rectángulos cerrados tal que $A \subset \bigcup_{j \in \N} R_j$ y $\sum_{j \in \N} \text{vol}(R_j) < m^*(A) + \epsilon$. Para cada $j \in \N$ consideramos $\epsilon_j = \frac{\epsilon}{2^j}$. Escogiendo $\delta_j > 0$ lo suficientemente pequeño, se tiene que $\text{vol}(R_j)_{\delta_j} < \text{vol}(R_j) + \epsilon_j$ para todo $j \in \N$. Nótese que aquí $\text{vol}(R_j)_{\delta_j}$ denota el volumen del n-rectángulo abierto $R_j$ con lados aumentados en $\delta_j$. Entonces $A \subset \bigcup_{j \in \N} R_j \subset \bigcup_{j \in \N} (R_j)_{\delta_j}$ y $\sum_{j \in \N} \text{vol}(R_j)_{\delta_j} < \sum_{j \in \N} (\text{vol}(R_j) + \epsilon_j) = \sum_{j \in \N} \text{vol}(R_j) + \epsilon < m^*(A) + 2\epsilon$. Por tanto, $\beta \leq m^*(A)$.
\end{proof}

\begin{definición}[Partición de un Conjunto\label{Partición de un Conjunto}]
Una partición del intervalo $[a,b]$ es una colección numerable de puntos $P = \{a = t_0 < t_1 < ... < t_n = b\}$. Dado un n-rectángulo $R \subset \R^n$, una partición $P =  \{P_1, P_2, ..., P_n\}$ de $R$ es una colección particiones $P_i$ de $[a_i, b_i]$ para cada $i = 1, 2, ..., n$ siendo $R = \prod_{i=1}^n [a_i, b_i]$.
\end{definición}

Los subrectángulos de $P$ son los conjuntos de la forma
\begin{equation}
    S_{i_1, i_2, ..., i_n} = \prod_{j=1}^n [t_{i_j}^j, t_{i_j + 1}^j]
\end{equation}
Denotamos $S \in P$ para indicar que $S$ es un subrectángulo de $P$.

\begin{lema}
    Sea $R \subset \R^n$ un n-rectángulo y $P$ una partición de $R$. Entonces:
    \vspace{-0.5em}
    \begin{enumerate}
        \item $R = \bigcup_{S \in P} S$
        \item Si $S, S' \in P$ y $S \neq S'$ entonces $S \cap S' = \emptyset$
        \item $\text{vol}(R) = \sum_{S \in P} \text{vol}(S)$
    \end{enumerate}
\end{lema}

\begin{proposición}
Sea $R \subset \R^n$ un n-rectángulo entonces $m^*(R) = \text{vol}(R)$.
\end{proposición}

\begin{proof}
    \leavevmode
    \begin{itemize}
        \item $m^*(R) \leq \text{vol}(R)$\\
              Sea \( R \subset \bigcup_{j \in \mathbb{N}} R_j \) con \( R_1 = R \) y \( R_j \) degenerados para \( j > 1 \). Entonces:
              \[
                  m^*(R) \leq \sum_{j \in \mathbb{N}} \text{vol}(R_j) = \text{vol}(R_1) + \sum_{j=2}^{\infty} \text{vol}(R_j) = \text{vol}(R_1) = \text{vol}(R).
              \]
        \item $m^*(R) \geq \text{vol}(R)$\\
              Dado $\epsilon > 0$ existe $(Q_j)_{j \in \mathbb{N}}$ sucesión de n-rectángulos abiertos tal que $R \subset \bigcup_{j \in \mathbb{N}} Q_j$ y $\sum_{j \in \mathbb{N}} \text{vol}(Q_j) < m^*(R) + \epsilon$. Sabemos que R es compacto al ser cerrado y acotado y, por tanto, al ser $\bigcup_{j \in \mathbb{N}} Q_j$ un recubrimiento abierto de R, existe un subrecubrimiento finito $\{Q_1, Q_2, ..., Q_m\}$ de R. Entonces $R \subset \bigcup_{i=1}^m Q_i \subset \bigcup_{i = 1}^m \overline{Q}_i$. Consideramos $R_j = R \cap \overline{Q}_j$ para $j = 1, 2, ..., m$. Tenemos entonces que $R = \bigcup_{j = 1}^{m} \overline{Q}_j$ y además prolongando los lados podemos obtener una partición $P$ de $R$ tal que cada subrectángulo de $P$ está contenido el algún $R_j$ para $1 \leq j \leq m$. Por tanto, $\text{vol}(R) = \sum_{S \in P} \text{vol}(S) \leq \sum_{j = 1}^{m} \text{vol}(R_j) \leq \sum_{j = 1}^{m} \text{vol}(Q_j) < m^*(R) + \epsilon$. Por tanto, $m^*(R) \geq \text{vol}(R)$.
    \end{itemize}
\end{proof}

\subsection{Medida de Lebesgue en $\R^n$}

\textbf{Notación:} Para $A \subset \R^n$ denotamos por $A^c$ al complementario de $A$ en $\R^n$.

\begin{definición}[Conjunto Medible\label{Conjunto Medible}]
Un conjunto $A \subset \R^n$ es medible en el sentido de Lebesgue si para todo $R \subset \R^n$ n-rectángulo se tiene que:
\begin{equation}
    m^*(R) = m^*(R \cap A) + m^*(R \cap A^c)
\end{equation}
\end{definición}

\begin{proposición}
Sea $A \subset \R^n$ entonces son equivalentes:
\vspace{-0.5em}
\begin{enumerate}
    \item $A$ es medible en el sentido de Lebesgue.
    \item $\forall E \subset \R^n$ conjunto se tiene que $m^*(E) = m^*(E \cap A) + m^*(E \cap A^c)$.
    \item $\forall E \subset \R^n$ conjunto se tiene que $m^*(E) \geq m^*(E \cap A) + m^*(E \cap A^c)$.
\end{enumerate}
\end{proposición}

\begin{proof}
    \leavevmode
    \begin{itemize}
        \item $(2) \implies (3)$\\
              Trivial.
        \item $(3) \implies (2)$\\
              Sabemos que $m^*(E) \leq m^*(E \cap A) + m^*(E \cap A^c)$. Veamos que la otra desigualdad se cumple siempre: $m^*(E) = m^*((E \cap A) \cup (E \cap A^c)) \leq m^*(E \cap A) + m^*(E \cap A^c)$.
        \item $(2) \implies (1)$\\
              Inmediato, tomando $E = R$.
        \item $(1) \implies (3)$\\
              Sea $E \subset \R^n$ conjunto, si $m^*(E) = +\infty$ entonces el resultado es inmediato. Supongamos que $m^*(E) < +\infty$. Sea $\epsilon > 0$. Por definición de medida exterior, $\exists (R_j)_{j \in \N}$ sucesión de n-rectángulos cerrados tal que $E \subset \bigcup_{j \in \N} R_j$ y $\sum_{j \in \N} \text{vol}(R_j) < m^*(E) + \epsilon$. Entonces $E \cap A \subset \bigcup_{j \in \N} R_j \cap A$ y $E \cap A^c \subset \bigcup_{j \in \N} R_j \cap A^c$. Por tanto, $m^*(E \cap A) + m^*(E \cap A^c) \leq \sum_{j \in \N} m^*(R_j \cap A) + \sum_{j \in \N} m^*(R_j \cap A^c) = \sum_{j \in \N} \text{vol}(R_j) < m^*(E) + \epsilon$. Por tanto, $m^*(E) \geq m^*(E \cap A) + m^*(E \cap A^c)$.
    \end{itemize}
\end{proof}

\begin{definición}[$\sigma$-Álgebra\label{sigma-Algebra}]
Sea $X$ un conjunto y $\mathcal{A} \subset \mathcal{P}(X)$ una colección de subconjuntos de $X$. Se dice que $\mathcal{A}$ es una $\sigma$-álgebra si:
\vspace{-0.5em}
\begin{enumerate}
    \item $X \in \mathcal{A}$
    \item Si $A \in \mathcal{A} \implies A^c \in \mathcal{A}$
    \item $\forall (A_j)_{j \in \N} \subset \mathcal{A}$ se tiene que $\bigcup_{j \in \N} A_j \in \mathcal{A}$
\end{enumerate}
\end{definición}

\label{Medida}
\begin{definición}[Medida\label{Medida}]
Sea $X$ un conjunto y $\mathcal{A} \subset \mathcal{P}(X)$ una $\sigma$-álgebra, entonces una medida en $X$ es una función $\mu: \mathcal{A} \to [0, +\infty]$ tal que:
\begin{enumerate}
    \item $\mu(\emptyset) = 0$
    \item Si $(A_j)_{j \in \N} \subset \mathcal{A}$ es una colección numerable de
          conjuntos disjuntos dos a dos entonces: $$\mu(\bigcup_{j \in \N} A_j) = \sum_{j
                  \in \N} \mu(A_j)$$
\end{enumerate}
\end{definición}

\begin{teorema}[Medida de Lebesgue en $\R^n$\label{Medida de Lebesgue en Rn}]
    La familia $M$ de todos los conjuntos medibles de $\R^n$ es una $\sigma$-álgebra y $m = m^* \restriction_M$ es una medida numerablemente aditiva que llamaremos medida de Lebesgue en $\R^n$.
\end{teorema}

Demostraremos este teorema con los siguientes lemas:

\begin{lema}
    $\R^n$ es medible en el sentido de Lebesgue.
\end{lema}

\begin{proof}
    Sea $E \subset \R^n$ conjunto. Entonces $m^*(E) = m^*(E \cap \R^n) + m^*(E \cap (\R^n)^c) = m^*(E) + m^*(\emptyset) = m^*(E) + 0 = m^*(E)$.
\end{proof}

\begin{lema}
    Sea $A \subset \R^n$ medible en el sentido de Lebesgue. Entonces $A^c$ es medible en el sentido de Lebesgue.
\end{lema}

\begin{proof}
    Sea $E \subset \R^n$ conjunto. Entonces $m^*(E \cap A^c) + m^*(E \cap (A^c)^c) = m^*(E \cap A^c) + m^*(E \cap A) = m^*(E)$
\end{proof}

Con los dos lemas anteriores obtenemos como colorario que $\emptyset$ es
medible en el sentido de Lebesgue.

\begin{lema}\label{unionMedible}
    Sean $A, B \subset \R^n$ medibles en el sentido de Lebesgue. Entonces $A \cup B$ y $A \cap B$ son medibles en el sentido de Lebesgue.
\end{lema}

\begin{proof}
    Observemos primero que
    \[
        A \cup B = (A^c \cap B) \cup (A \cap B) \cup (A \cap B^c)
    \]
    luego entonces tenemos que
    \[
        m^*(A \cup B) \leq m^*(A^c \cap B) + m^*(A \cap B) + m^*(A \cap B^c)
    \]

    Sea \( E \subset \R^n \) un conjunto, entonces por la medibilidad de \( A \)
    tenemos que
    \[
        m^*(E) = m^*(E \cap A) + m^*(E \cap A^c)
    \]

    Además, sabemos que \( B \) es medible, luego para los conjuntos \( E \cap A^c
    \) y \(E \cap E\) se verifica
    \[
        \begin{matrix}
            m^*(E \cap A^c) = m^*(E \cap A^c \cap B) + m^*(E \cap A^c \cap B^c) \\
            m^*(E \cap A) = m^*(E \cap A \cap B) + m^*(E \cap A \cap B^c)
        \end{matrix}
    \]

    Por tanto,
    \[
        m^*(E) = m^*(E \cap A) + m^*(E \cap A^c) = m^*(E \cap A) + m^*(E \cap A^c \cap B) + m^*(E \cap A^c \cap B^c)
    \]
    \[
        = \underbrace{m^*(E \cap A \cap B) + m^*(E \cap A \cap B^c) + m^*(E \cap A^c \cap B)}_{\geq \ m^*(E \cap (A \cup B))} + \underbrace{m^*(E \cap A^c \cap B^c)}_{m^*(E \cap (A \cup B)^c)}
    \]

    Finalmente, observamos que
    \[
        m^*(E) \geq m^*(E \cap (A \cup B)) + m^*(E \cap (A \cup B)^c)
    \]
    y por tanto \( A \cup B \) es medible.\\ Nótese que la medibilidad de la
    intersección es inmediata, pues \( A \cap B = (A^c \cup B^c)^c \), y ya hemos
    demostrado que el complementario de un conjunto medible es medible.
\end{proof}

\begin{lema}
    Sea $(A_j)_{j \in \N} \subset \R^n$ una colección numerable de conjuntos disjuntos medibles en el sentido de Lebesgue. Entonces $\bigcup_{j \in \N} A_j$ es medible en el sentido de Lebesgue y además $m^*(\bigcup_{j \in \N} A_j) = \sum_{j \in \N} m^*(A_j)$.
    \label{unionNumerableMedible}
\end{lema}

\begin{proof}
    Definimos la sucesión creciente de conjuntos \( B_k = A_1 \cup \ldots \cup A_k \). Entonces por el \cref{unionMedible}, \( B_k \) es medible en el sentido de Lebesgue. Sean \( B = \bigcup_{k \in \N} B_k = \bigcup_{j \in \N} A_j \) y \( E \subset \R^n \). Por la medibilidad de $A_k$, tenemos:
    \[
        m^*(E \cap B_k) = m^*(E \cap B_k \cap A_k) + m^*(E \cap B_k \cap A_k^c) = m^*(E \cap A_k) + m^*(E \cap B_{k-1})
    \]
    Nótese que $A_k^c = B_{k-1}$ precisamente porque los conjuntos $A_j$ son
    disjuntos. Reiterando el proceso, obtenemos:
    \[
        m^*(E \cap B_k) = \sum_{j=1}^k m^*(E \cap A_j)
    \]
    Por lo tanto, aplicando la mediblidad de \( B_k \):
    \[
        m^*(E) = m^*(E \cap B_k) + m^*(E \cap B_k^c) = \left( \sum_{j=1}^k m^*(E \cap A_j) \right) + m^*(E \cap B_k^c) \geq \sum_{j=1}^k m^*(E \cap A_j) + m^*(E \cap B^c)
    \]
    Se sigue entonces:
    \[
        m^*(E) \geq \sum_{j \in \N} m^*(E \cap A_j) + m^*(E \cap B^c) \geq m^*\left( \bigcup_{j \in \N} E \cap A_j \right) + m^*(E \cap B^c) \geq m^*(E \cap B) + m^*(E \cap B^c)
    \]
    Luego, \( B \) es medible.

    Tomando \( E = B \) en la desigualdad anterior, obtenemos:
    \[
        m^*\left( \bigcup_{j \in \N} A_j \right) = m^*(B) \geq \sum_{j \in \N} m^*(B \cap A_j) + m^*(B \cap B^c) = \sum_{j \in \N} m^*(B \cap A_j) = \sum_{j \in \N} m^*(A_j)
    \]
    Por otro lado, por el apartado 2 del \cref{funcionMedida} sabemos que la medida
    exterior de la union numerable de conjuntos es menor o igual que la suma de las
    medidas exteriores de los conjuntos:
    \[
        m^*\left( \bigcup_{j \in \N} A_j \right) \leq \sum_{j \in \N} m^*(A_j)
    \]
    Por tanto,
    \[
        m^*\left( \bigcup_{j \in \N} A_j \right) = \sum_{j \in \N} m^*(A_j)
    \]
\end{proof}

\begin{lema}
    La unión numerable de conjuntos medibles en el sentido de Lebesgue es un conjunto medible en el sentido de Lebesgue.
\end{lema}

\begin{proof}
    Sea $(B_j)_{j \in \N}$ una colección numerable de conjuntos medibles en el sentido de Lebesgue. Considermos:
    \[\begin{matrix}
            A_1 = B_1                       \\
            A_2 = B_2 \cap B_1^c            \\
            A_3 = B_3 \cap B_2^c \cap B_1^c \\
            \vdots                          \\
            A_j = B_j \cap B_{j-1}^c \cap \ldots \cap B_1^c
        \end{matrix}\]
    Observemos que $\bigcup_{j \in \N} A_j = \bigcup_{j \in \N} B_j$ y que para
    todo $j \in \N$, $A_j$ es intersección finita de conjuntos medibles, por tanto,
    $A_j$ es medible. Además, $\forall i,j \in \N$ con $i \neq j$, $A_i \cap A_j =
        \emptyset$. Por el \cref{unionNumerableMedible}, $\bigcup_{j \in \N} A_j$ es
    medible $\implies \bigcup_{j \in \N} B_j$ es medible.
\end{proof}

\begin{proposición}
Todo conjunto nulo es medible en el sentido de Lebesgue.
\end{proposición}

\begin{proof}
    Sea $A \subset \R^n$ nulo, entonces $m^*(A) = 0$. $\forall E \in \R^n$ se tiene que $E \cap A \subset A \implies 0 \leq m^*(E \cap A) \leq m^*(A) = 0 \implies m^*(E \cap A) = 0$. Análogamente, $E \cap A^c \subset E \implies m^*(E \cap A^c) \leq m^*(E)$. Luego $m^*(E \cap A) + m^*(E \cap A^c) = m^*(E \cap A^c) \leq m^*(E)$ y por tanto $A$ es medible en el sentido de Lebesgue.
\end{proof}

\begin{definición}[Propiedad en casi todo punto\label{Propiedad en casi todo punto}]
Se dice que una \textit{propiedad} se verifica en casi todo punto cuando el conjunto de puntos en los que no se verifica la propiedad es un conjunto nulo.
\end{definición}

\begin{proposición}
Todo n-rectángulo cerrado $R \subset \R^n$ es medible en el sentido de Lebesgue.
\end{proposición}

\begin{proof}
    Dado $R \subset \R^n$ n-rectángulo cerrado, tenemos que ver que $\forall Q \subset \R^n$ n-rectángulo cerrado se tiene que $\text{vol}(Q) \geq m^*(Q \cap R) + m^*(Q \cap R^c)$. Consideramos el n-rectángulo $Q_0 = Q \cap R$. Nótese que $Q \cap R^c$ es unión finita de n-rectángulos $\{Q_1,\ldots, Q_m\}$. Entonces $Q = Q_0 \cup Q_1 \cup \ldots \cup Q_m$ forman una partición de $Q$. Luego $\text{vol}(Q) = \sum_{i=0}^m \text{vol}(Q_i) = m^*(Q \cap R) + \sum_{i=1}^m m^*(Q_i) \geq m^*(Q \cap R) + m^*(Q \cap R^c)$.
\end{proof}

\begin{observación}
En $\R^n$ los rectángulos abiertos son medibles en el sentido de Lebesgue.
\end{observación}

\begin{definición}[n-Cubo\label{n-Cubo}]
Un n-cubo cerrado (respectivamente abierto) en $\R^n$ es un conjunto de la forma:
\begin{equation}
    R = [a_1, b_1] \times \ldots \times [a_n, b_n] \text{ tal que } \forall i,j \in \{1,2,...,n\} \text{ se tiene que } b_i - a_i = b_j - a_j
\end{equation}
Análogamente se pueden definir los cubos n-dimensionales semi-abiertos.
\end{definición}

\observacion{
    Denotaremos la norma del supremo en $\rn$ como:
    \begin{equation}
        \lVert x \rVert _{\infty} = \sup_{i=1}^n \{|x_i|\} \text{ para } x = (x_1, x_2, ..., x_n) \in \rn
    \end{equation}
    Llamaremos bola abierta de centro $x \in \rn$ y radio $r > 0$ al conjunto:
    \begin{equation}
        B_{\infty}(x, r) = \{y \in \rn :  \lVert y - x \rVert _{\infty} < r\} \equiv (x_1 - r, x_1 + r) \times \ldots \times (x_n - r, x_n + r)
    \end{equation}
    Análogamente, llamaremos bola cerrada de centro $x \in \rn$ y radio $r > 0$ al conjunto:
    \begin{equation}
        \overline{B}_{\infty}(x, r) = \{y \in \rn :  \lVert y - x \rVert _{\infty} \leq r\} \equiv [x_1 - r, x_1 + r] \times \ldots \times [x_n - r, x_n + r]
    \end{equation}
}

\begin{teorema}
    Sea $G \subset \mathbb{R}^n$ abierto entonces se tiene:
    \vspace{-0.5em}
    \begin{enumerate}
        \item G es unión numerable de n-cubos cerrados.
        \item G es unión numerable de n-cubos abiertos.
    \end{enumerate}
\end{teorema}

\begin{proof}
    Consideremos la familia de n-cubos $\mathcal{B} = \{\overline{B}_\infty(q,r):q\in \mathbb{Q}^n, r \in \mathbb{Q}, r>0, \overline{B}_\infty(q,r) \subset G\}$. Veamos que $G = \bigcup_{B \in \mathcal{B}}B$. Dado que $B \in G \quad \forall B \in \mathcal{B}$ entonces es inmediato ver que $\bigcup_{B \in \mathcal{B}}B \subset G$. Por ser G abierto, $\exists \delta > 0$ tal que $B_{\infty}(x, \delta) \subset G$. Sea $r \in \Q$ con $0 < r < \frac{\delta}{2}$, por la densidad de $\Q^n$ en $\R^n$, sabemos que $\exists q \in \Q^n$ tal que $ \lVert x - q \rVert _{\infty} < r$. Veamos entonces que $x \in B_{\infty}(q, r) \subset B_{\infty}(x, \delta) \subset G$.\\
    Dado $y \inrn$ con $ \lVert y - q \rVert _{\infty} < r$ se sigue:
    \[
        \lVert y - x \rVert _{\infty} \leq  \lVert y - q \rVert _{\infty} +  \lVert q - x \rVert _{\infty} < r + r = 2r < \delta
    \]
    Por tanto $y \in B_{\infty}(x, \delta) \implies x \in \overline{B}_{\infty}(q,
        r) \subset G$. Luego $G = \bigcup_{B \in \mathcal{B}}B$.\\ Nótese que
    numerabilidad de la familia $\mathcal{B}$ es inmediata por la numerabilidad de
    $\Q^n$ que, a su vez, es numerable por ser $\Q$ numerable.\\ La segunda parte
    del teorema es análoga a la primera.
\end{proof}

\begin{corolario}
    Todos los conjuntos abiertos y cerrados de $\R^n$ son medibles en el sentido de Lebesgue.
    \label{abiertoMedible}
\end{corolario}

\begin{teorema} [Regularidad de la Medida\label{Regularidad de la medida}]
    Sea $E \inrn$, entonces son equivalentes:
    \begin{enumerate}
        \item $E$ es medible en el sentido de Lebesgue.
        \item $\forall \epsilon > 0 \quad \exists G \subset \mathbb{R}^n$ abierto tal que $E \subset G$ y $m^*(G \setminus E) < \epsilon$.
        \item $\forall \epsilon > 0 \quad \exists F \subset \mathbb{R}^n$ cerrado tal que $F \subset E$ y $m^*(E \setminus F) < \epsilon$.
        \item $\forall \epsilon$ existen $F$ cerrado y $G$ abierto tales que $F \subset E \subset G$ y $m^*(G \setminus F) < \epsilon$.
    \end{enumerate}
\end{teorema}

\begin{proof}
    \leavevmode
    \begin{itemize}
        \item $(1) \implies (2)$\\
              Distinción de casos:
              \begin{enumerate}
                  \item Supongamos que $m^*(E) < +\infty$: Sea $\epsilon > 0$. Por definición de medida
                        exterior, $\exists (R_j)_{j \in \N}$ sucesión de n-rectángulos abiertos tales
                        que $E \subset \unj(R_j)$ y $\sum_{j \in \N} \text{vol}(R_j) < m^*(E) +
                            \epsilon$. Considerando el abierto $G = \unj(R_j)$, se tiene que $G$ es medible
                        por el \cref{abiertoMedible}, además, como $E \subset G$ entonces $$m^*(G) =
                            m^*(\underbrace{G \cap E}_{E}) + m^*(\underbrace{G \cap E^c}_{G \setminus E}) =
                            m^*(E) + m^*(G \setminus E)$$ Por tanto, $$m^*(G \setminus E) = m^*(G) - m^*(E)
                            < \sum_{j \in \N} \text{vol}(R_j) - m^*(E) < \epsilon$$
                  \item Supongamos que $m^*(E) = +\infty$: $\forall k \in \N$ sea $E_k = E \cap
                            [-k,k]^n$, que es medible por ser intersección finita de conjuntos medibles.
                        Además $m^*(E_k) < +\infty$ por ser $E_k$ acotado, y $E = \unk{E_k}$. Dado
                        $\epsilon > 0$, $\forall k \in \N$ existe $G_k$ abierto tal que $E_k \subset
                            G_k$ y $m^*(G_k \setminus E_k) < \frac{\epsilon}{2^k}$. Entonces $G =
                            \unk{G_k}$ abierto y $E = \unk{E_k} \subset \unk{G_k} = G$ por lo que $m^*(G
                            \setminus E) \leq m^*(\unk(G_k \setminus E_k)) \leq \sk{m^*(G_k \setminus E_k)}
                            < \sk{\frac{\epsilon}{2^k}} = \epsilon$.
              \end{enumerate}
        \item $(2) \implies (1)$\\
              $\forall j \in \N$ tomando $\epsilon = \frac{1}{j}$ entonces $\exists G_j$ abierto tal que $E \subset G_j$ y $m^*(G_j \setminus E) < \frac{1}{j}$. Entonces considerando $B = \intj{G_j}$ que es medible y abierto se tiene que $E \subset B$. Luego $B \setminus E \subset G_j \setminus E$ para todo $j \in \N$. Por tanto, $m^*(B \setminus E) \leq m^*(G_j \setminus E) < \frac{1}{j}$. En consecuencia $m^*(B \setminus E) = 0 \implies B \setminus E$ es medible.\\
              Por otro lado, $B = E \cup (B \setminus E)$ o que es lo mismo $E = B \setminus (B \setminus E)$. Tanto $B$ como $(B \setminus E)$ son medibles, luego $E$ es medible.\\
              \textit{Observación:} Además, $E = B \setminus Z$, donde $B$ es intersección numerable de abiertos o $Z$ es un conjunto nulo.
        \item (1) $\implies$ (3)\\
              Como E es medible entonces tenemos que $E^c$ también es medible, por lo que, dado $\epsilon > 0$ por (2) $\exists G$-abierto tal que $E^c \subset G$ y $m^*(G \setminus E^c) < \epsilon$. Entonces $F = G^c$ es cerrado y $F \subset E$. Además, $E \setminus F = E \cap F^c = E \cap G = G \setminus E^c \implies m^*(E \setminus F) = m^*(G \setminus E^c) < \epsilon$.
        \item (3) $\implies$ (1)\\
              $\forall j \in \mathbb{N}$ $\exists F_j$ cerrado tal que $F_j \subset E$, $m(E \setminus F_j) < 1/j$. Sea $A = \bigcup_{j = 1}^{\infty} F_j$ conjunto medible y $A \subset E$. Además, $m(E \setminus A) \leq m(E \setminus F_j) < 1/j$ $\forall j \in \mathbb{N}$. Por tanto, $E = A \cup (E\setminus A) = (\bigcup_{j = 1}^{\infty}F_j) \cup (E\setminus A)$ Entonces dado que $E\setminus A$ es un conjunto medible por ser nulo y $\bigcup_{j = 1}^{\infty}F_j$ es medible por ser unión numerable de conjuntos cerrados, entonces $E$ es medible.
    \end{itemize}
\end{proof}

\begin{definición} [$\sigma$-Álgebra de Borel\label{sigma-Algebra de Borel}]
La $\sigma$-álgebra de Borel en $\rn$ es la menor $\sigma$-álgebra que contiene a todos los abiertos de $\rn$ (o equivalentemente, la menor $\sigma$-álgebra que contiene a todos los cerrados de $\rn$). Los conjuntos de $\mathcal{B}(\rn)$ se llaman conjuntos de Borel o conjuntos Borelianos.\\
\end{definición}

\begin{definición} [Conjuntos $G_{\delta}$ y $F_{\sigma}$]
Decimos que $A \subset \rn$ es $G_\delta$ si $A$ es intersección numerable de abiertos. Análogamente, decimos que un conjunto $B \subset \rn$ es $F_\sigma$ si $B$ es unión numerable de cerrados.
\end{definición}

\begin{corolario}
    Sea $E \subset \rn$, entonces son equivalentes:
    \vspace{-0.5em}
    \begin{enumerate}
        \item $E$ es medible en el sentido de Lebesgue.
        \item $E = A \setminus N$ con $A$ siendo $G_\delta$ y $N$ un conjunto nulo.
        \item $E = B \cup N$ con $B$ siendo $F_\sigma$ y $N$ un conjunto nulo.
    \end{enumerate}
\end{corolario}

\begin{lema}
    Sea $(A_j)_{j \in \N}$ familia numerable y creciente de conjuntos medibles en el sentido de Lebesgue. Entonces $\bigcup_{j \in \N} A_j$ es medible en el sentido de Lebesgue y $m(\bigcup_{j \in \N} A_j) = \lim_{j \to \infty} m(A_j)$.
    \label{limUnionMedible}
\end{lema}

\begin{proof}
    Sea $\{B_j\}_{j \in \N}$ una colección numerable de conjuntos medibles en el sentido de Lebesgue. Considermos:
    \[\begin{matrix}
            A_1 = B_1                       \\
            A_2 = B_2 \cap B_1^c            \\
            A_3 = B_3 \cap B_2^c \cap B_1^c \\
            \vdots                          \\
            A_j = B_j \cap B_{j-1}^c \cap \ldots \cap B_1^c
        \end{matrix}\]
    De esta manera obtenemos que $\bigcup_{j = 1}^{\infty} A_j = \bigcup_{j =
            1}^{\infty} B_j$ y que $(B_j)_{j \in \mathbb{N}}$ es una sucesión disjunta de
    conjuntos Entonces $m^*(\bigcup_{j = 1}^{\infty}A_j) = m^*(\bigcup_{j =
            1}^{\infty}B_j) = \sum_{j = 1}^{\infty} = \lim_{k \to \infty} m(A_k)$ Dado que
    $m^*(A_j) = m(B_1) + m(B_2) + \ldots + m(B_j)$ $ \forall j \geq 1$
\end{proof}

\begin{corolario}
    Sea $E \subset \rn$ medible entonces:
    \vspace{-0.5em}
    \begin{enumerate}
        \item $m(E) = \text{inf}\{m(G) : G \text{ abierto y } E \subset G\}$.
        \item $m(E) = \text{sup}\{m(K) : K \text{ compacto y } K \subset E\}$.
    \end{enumerate}
\end{corolario}

\begin{proof}
    \leavevmode 
    \begin{enumerate}
    \item Dado $\epsilon > 0$ por el \cref{Regularidad de la medida} $\exists G$ abierto tal que $E \subset G$ y $m(G \setminus E) < \epsilon$. Entonces $m(E) \leq m(G) = m(E) + m(G \setminus E) < m(E) + \epsilon$. Por tanto, $m(E) = \text{inf}\{m(G) : G \text{ abierto y } E \subset G\}$.
    \item   $E = \bigcup_{k = 1}^{\infty} E_k : E_k = E \cap [-k, k]^n$ $\forall k \in \mathbb{N}$
    Entonces $(E_k)_k \in \mathbb{N}$ es una sucesión creciente de conjuntos medibles y por el lema anterior tenemos que $m(E) = \lim_{k \to \infty} m(E_k)$
    Además, $\forall k \in \mathbb{N}$ $\exists F_k$ $ \subset E_k$ cerrado tal que $m(E_k \setminus F_k) < \frac{1}{k}$
    Entonces como $F_k$ es un conjunto cerrado y acotado, tenemos que el conjunto es compacto.
    Por tanto $m(E_k) = m(E_k \setminus F_k) + m(F_k) \geq m(F_k) + \frac{1}{k}$ y por tanto $m(E) = \lim_{k \to \infty} m(F_k)$ y finalmente obtenemos que
    $m(E) = \sup\{m(F_k) : k \in \mathbb{N} \} = \sup\{m(K) : K \text{ compacto y } K \subset E\}$
    \end{enumerate}  
\end{proof}

\begin{definición}[Cubo Diádico\label{Cubo Diadico}]
Se dice que un cubo en $\R^n$ es diádico si sus lados miden $2^{-m}$ para algún $m \in \N$.
Es decir, si el rectángulo Q es de la forma:
\[Q = \left[\frac{k_1}{2^m}, \frac{k_1 + 1}{2^m}\right] \times \dots \times \left[\frac{k_n}{2^m}, \frac{k_n + 1}{2^m}\right],\]
con $m \in \mathbb{Z} \text{(nivel de escala u orden) y } k_1, k_2, \dots k_n
    \in \mathbb{Z}$
\end{definición}

\begin{teorema}
    Todo conjunto abierto $U$ de $\mathbb{R}^n$ es unión numerable y disjunta de n-cubos semiabiertos, que son cubos diádicos.
\end{teorema}

\begin{proof}
    Denotemos por $\mathcal{F}$ la familia de todos los cubos cerrados de la forma
    \[
        \left[\frac{k_1}{2^m}, \frac{k_1 + 1}{2^m}\right] \times \dots \times \left[\frac{k_n}{2^m}, \frac{k_n + 1}{2^m}\right],
    \]
    con $k_i \in \mathbb{Z}$ y $m \in \mathbb{N}$. Sea $\mathcal{Q}_1$ la familia
    de todos los cubos cerrados $Q$ de la forma $[k_1, k_1 + 1] \times \dots \times
        [k_n, k_n + 1]$, donde los $k_i \in \mathbb{Z}$, y tales que $Q \subset U$.
    Supuesto definida $\mathcal{Q}_m$, sea $\mathcal{Q}_{m+1}$ la familia de todos
    los cubos $Q$ de la forma
    \[
        \left[\frac{k_1}{2^m}, \frac{k_1+1}{2^m}\right] \times \dots \times \left[\frac{k_n}{2^m}, \frac{k_n+1}{2^m}\right],
    \]
    donde $k_i \in \mathbb{Z}$, tales que no están contenidos en ningún cubo $Q'
        \in \mathcal{Q}_j$ para $j \leq m$, y tales que $Q \subset U$. Por inducción
    queda definida $\mathcal{Q}_m$ para todo $m \in \mathbb{N}$, y ponemos
    \[
        \mathcal{Q} = \bigcup_{m=1}^{\infty} \mathcal{Q}_m.
    \]

    Es obvio por construcción que si $Q, Q' \in \mathcal{Q}$ y $Q \neq Q'$,
    entonces $Q$ y $Q'$ tienen interiores disjuntos. También es claro que 
    $\bigcup_{Q \in \mathcal{Q}} Q \subset U$. Veamos que de hecho
    \[
        U = \bigcup_{Q \in \mathcal{Q}} Q.
    \]

    Dado $x \in U$, usando que $U$ es abierto y que el conjunto $\{k/2^m : k \in
        \mathbb{Z}, m \in \mathbb{N} \cup \{0\}\}$ es denso en $\mathbb{R}$, es fácil
    ver que existe algún cubo $Q_x \in \mathcal{F}$ tal que $x \in Q_x$ y $Q
        \subset U$. El lado de $Q_x$ mide $2^{-m_x}$ para algún $m_x \in \mathbb{N}
        \cup \{0\}$. Si $Q_x \in \mathcal{Q}_{m_x}$ ya hemos terminado. En otro caso,
    por definición de $\mathcal{Q}_{m_x}$, existe algún $j < m_x$ tal que $Q_x$
    está contenido en algún cubo $Q_x' \in \mathcal{Q}_j$, y por tanto $x$
    pertenece a este cubo. En cualquier caso se ve que $x \in
        \bigcup_{i=1}^{\infty} Q_i$.
\end{proof}
