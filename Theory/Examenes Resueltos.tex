\section{Examenes Resueltos}

\subsection{Test de seguimiento (Marzo-2023). Modelo A}

\begin{problem}{1}
    Indicar si las afirmaciones siguientes son verdaderas o falsas:
    \begin{enumerate}
        \item Si $E \in \mathbb{R}^n$ es medible, entonces $m(E) = m(\bar{E})$.
        \item Sea $E \in \mathbb{R}^n$. Si $m(\delta E)=0$, entonces $E$ es medible.
    \end{enumerate}
\end{problem}

\begin{sol}
    \begin{enumerate}
        \item Falso. \\
        Tengamos como ejemplo $\mathbb{Q}$ cuya medida es $0$ por ser numerable, pero su adherencia $\bar{\mathbb{Q}}=\mathbb{R}$ tiene medida infinita.
        \item Verdadero. \\
        Si $m(\delta E)=0$, se deduce que $m^{*}(\bar{E})=m^{*}(\mathring{E})$. Entonces apliquemos la definición de conjunto medible: \\
        Sea $S \subset \mathbb{R}^n$ arbitrario, queremos comprobar que $m^{*}(S)=m^{*}(S \cap E)+m^{*}(S \cap E^c)$. \\
        Como $m^{*}(\bar{E})=m^{*}(\mathring{E})$, y teniendo en cuenta que $\mathring{E} \subset E \subset \bar{E}$, se tiene que: \\
        $m^{*}(S \cap \mathring{E}) \leq m^{*}(S \cap E) \leq m^{*}(S \cap \bar{E})$. \\
        Ademas, se tiene: \\
        $m^{*}(S \cap \mathring{E}^c) \geq m^{*}(S \cap E^c) \geq m^{*}(S \cap \bar{E}^c)$. \\
        Entonces tenemos que $m^{*}(S \cap \mathring{E})+m^{*}(S \cap \bar{E}^c) \leq m^{*}(S \cap E)+m^{*}(S \cap E^c) \leq m^{*}(S \cap \bar{E})+m^{*}(S \cap \mathring{E}^c)$. \\
        Teniendo en cuenta que $\mathring{E}$ es abierta y $\bar{E}$ es cerrado, ambos son medibles, por lo que 
    \end{enumerate}
\end{sol}