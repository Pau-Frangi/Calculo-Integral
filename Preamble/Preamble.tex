\usepackage[utf8]{inputenc}

%Colors:
\usepackage{xcolor}
\usepackage{colortbl}

%Math
\usepackage[makeroom]{cancel}%cancel lines math
\usepackage{dsfont}%fonts
\usepackage{multicol}
\usepackage{amsmath}
\usepackage{esint}
\usepackage{bigints}
\usepackage{amsfonts}
\usepackage{amsmath,xparse}%matrices
\usepackage{amssymb}%math symbols
\usepackage{braket}
\usepackage{mathtools}
\usepackage{amsmath,amssymb}
\usepackage{enumitem}
\usepackage{xhfill}
\usepackage{tikz}
\usepackage[most]{tcolorbox}
\usepackage{lipsum}
%Hyperlinks




\usepackage{url}

\usetikzlibrary{patterns}
\usepackage{tikz-3dplot}
\usetikzlibrary{math}

\usetikzlibrary{calc}
\newcommand{\s}{\text{ }}%Space in math mode
\newcommand{\trace}{\text{trace}\,}%Trace command
\newcommand{\rank}{\text{rank}\,}%Rank command
\newcommand{\R}{\mathbb{R}}%Real Numbers
\newcommand{\Z}{\mathbb{Z}}%Integers
\newcommand{\Q}{\mathbb{Q}}%Rational Numbers
\newcommand{\Hbt}{\mathcal{H}}%Hilbert Space
\newcommand{\F}{\mathcal{F}}%Hilbert Space
\newcommand{\C}{\mathbb{C}}%Complex Numbers
\newcommand{\N}{\mathbb{N}}%Natural Numbers
\newcommand{\Imat}{\mathds{1}}%Identity Matrix

\newcommand{\uni}{\bigcup_{i=1}^{\infty}}%Union i=1 to infinity
\newcommand{\unj}{\bigcup_{j=1}^{\infty}}%Union j=1 to infinity
\newcommand{\unk}{\bigcup_{k=1}^{\infty}}%Union k=1 to infinity
\newcommand{\inti}{\bigcap_{i=1}^{\infty}}%Intersection i=1 to infinity
\newcommand{\intj}{\bigcap_{j=1}^{\infty}}%Intersection j=1 to infinity
\newcommand{\intk}{\bigcap_{k=1}^{\infty}}%Intersection k=1 to infinity
\newcommand{\si}{\sum_{i=1}^{\infty}}%Sum i=1 to infinity
\newcommand{\sj}{\sum_{j=1}^{\infty}}%Sum j=1 to infinity
\newcommand{\sk}{\sum_{k=1}^{\infty}}%Sum k=1 to infinity

\newcommand{\rn} {\mathbb{R}^n}%R^n
\newcommand{\inr} {\in \mathbb{R}} %in R
\newcommand{\inrn} {\in \mathbb{R}^n} %in R^n
\newcommand{\vol} {\text{vol}} %Volume

% \newcommand{\ket}[1]{\left|#1\right\rangle}%Prints |#1⟩
% \newcommand{\bra}[1]{\left\langle#1\right|}%Prints ⟨#1|
% \newcommand{\braket}[2]{\langle#1|#2\rangle}%Prints ⟨#1|#2⟩
% \newcommand{\Braket}[2]{\left\langle#1\left|#2\right.\right\rangle}%Prints ⟨#1|#2⟩
% \newcommand{\braopket}[3]{\left\langle#1\left|#2\left|#3\right.\right.\right\rangle}%Prints ⟨#1|#2|#3⟩
\newcommand{\expected}[1]{\left\langle#1\right\rangle}%Prints ⟨#1⟩
\newcommand{\naturalset}[2]{#1=1,\;2,\;...,\;#2}%Prints #1= #1= 1, 2, ..., #2 in math mode
\newcommand{\vectorset}[3]{#1=\{#2_1,\;#2_2,\;...,\;#2_{#3}\}}%Prints #1= {#2_1,#2_2, ..., #2_{#3}} in math mode
\newcommand{\diagmatrix}[9]{%
  \begin{pmatrix}
    \lambda_1 & 1 & 0 & 0 & 0 \\0&\lambda_1&0&0&0\\0&0&\lambda_3&1&0\\0&0&0&\lambda_4&1\\0&0&0&0&\lambda_4
  \end{pmatrix}
}%Matrix with diagonal entries specified

% Setup matha font (mathabx.sty) for extracting specific symbols:
\DeclareFontFamily{U}{matha}{\hyphenchar\font45}
\DeclareFontShape{U}{matha}{m}{n}{
  <5> <6> <7> <8> <9> <10> gen * matha
  <10.95> matha10 <12> <14.4> <17.28> <20.74> <24.88> matha12
}{}
\DeclareSymbolFont{matha}{U}{matha}{m}{n}
% Define a plus/minus character from that font (from mathabx.dcl):
\DeclareMathSymbol{\PM}{2}{matha}{"08}% to completely replace the \pm character, replace \PM with \pm
\ExplSyntaxOn

\NewDocumentCommand{\diagonal}{O{b}m}
 {% #1 = fences, #2 = entries, comma separated
  \egreg_diagonal:nn { #1 } { #2 }
 }

\seq_new:N \l__egreg_diagonal_entries_seq
\seq_new:N \l__egreg_diagonal_row_seq

\cs_new_protected:Nn \egreg_diagonal:nn
 {
  \seq_set_from_clist:Nn \l__egreg_diagonal_entries_seq { #2 }
  \begin{#1matrix}
  \int_step_function:nN { \seq_count:N \l__egreg_diagonal_entries_seq } \__egreg_diagonal:n
  \end{#1matrix}
 }

\cs_new_protected:Nn \__egreg_diagonal:n
 {% #1 = row number
  \seq_clear:N \l__egreg_diagonal_row_seq
  \int_step_inline:nn { \seq_count:N \l__egreg_diagonal_entries_seq }
   {
    \int_compare:nTF { #1 == ##1 }
     {
      \seq_put_right:Nx \l__egreg_diagonal_row_seq
       {
        \seq_item:Nn \l__egreg_diagonal_entries_seq { #1 }
       }
     }
     {
      \seq_put_right:Nn \l__egreg_diagonal_row_seq { 0 }
     }
   }
  \seq_use:Nn \l__egreg_diagonal_row_seq { & } \\
 }

\ExplSyntaxOff

%Page Margins:
\usepackage[a4paper, total={7in, 10in}]{geometry}

%paragraph Indent and Spacing
\setlength{\parindent}{0em}
\setlength{\parskip}{0.5em}

%Images:
\usepackage{graphicx}
\graphicspath{ {./images/} }
\usepackage{eso-pic}
\usepackage{wrapfig}%Wrapping Functionality
\usepackage{float}
\usepackage{chngcntr}% Count Figures as below v v v
\counterwithout{figure}{section}%          CONTINUOUS
\counterwithout{table}{section}%           CONTINUOUS

%Enumerate
\usepackage{enumitem}

%Language
\usepackage[english]{babel}

%Environments
\usepackage{amsthm, amsfonts}
\usepackage{thmtools}
\usepackage{float}
\usepackage{framed}
\usepackage{tcolorbox}

\colorlet{LightGray}{gray!15}
\colorlet{LightOrange}{orange!15}
\colorlet{LightGreen}{green!15}
\colorlet{LightBlue}{blue!15}
\colorlet{LightRed}{red!15}
\colorlet{veccol}{green!70!black}
\colorlet{vcol}{green!70!black}
\colorlet{xcol}{blue!85!black}
\colorlet{projcol}{xcol!60}
\colorlet{unitcol}{xcol!60!black!85}
\colorlet{unitcol2}{vcol!60!black!85}
\colorlet{myblue}{blue!70!black}
\colorlet{myred}{red!70!black}
\tikzstyle{vector}=[->,very thick,xcol]
\tikzstyle{mydashed}=[dash pattern=on 2pt off 2pt]
\def\tick#1#2{\draw[thick] (#1) ++ (#2:0.1) --++ (#2-180:0.2)} %0.03*\xmax


\usetikzlibrary{angles,quotes} % for pic
\usepackage{contour}
\contourlength{1.3pt}

\newcommand{\HRule}[1]{\rule{\linewidth}{#1}}

\usepackage[framemethod=TikZ]{mdframed}
\usepackage{amsthm}

\declaretheoremstyle[
headfont=\bfseries,
notebraces={[}{]},
%notefont=\bfseries,
bodyfont=\normalfont\itshape,
headpunct={},
postheadspace=\newline,
postheadhook={\textcolor{black}{\rule[.6ex]{\linewidth}{0.4pt}}\\},
spacebelow= 5pt,
spaceabove= 10pt,
mdframed={
    backgroundcolor=LightRed,
    innertopmargin=6pt,
    roundcorner=5pt,
    innerbottommargin=6pt,
    skipabove=\parsep,
    skipbelow=\parsep }
]{myTheoremStyle}

% example environment - thmtools
\declaretheorem[
  style=myTheoremStyle,
  numberwithin=subsection
]{teorema}

\declaretheoremstyle[
headfont=\bfseries,
notebraces={[}{]},
bodyfont=\normalfont\itshape,
headpunct={},
postheadspace=\newline,
postheadhook={\textcolor{black}{\rule[.6ex]{\linewidth}{0.4pt}}\\},
spacebelow= 5pt,
spaceabove= 10pt,
mdframed={
    backgroundcolor=LightOrange,
    innertopmargin=6pt,
    roundcorner=5pt,
    innerbottommargin=6pt,
    skipabove=\parsep,
    skipbelow=\parsep }
]{myPropositionStyle}

% example environment - thmtools
\declaretheorem[
  style=myPropositionStyle,
  numberwithin=subsection
]{proposición}

\declaretheoremstyle[
headfont=\bfseries,
notebraces={[}{]},
bodyfont=\normalfont\itshape,
headpunct={},
postheadspace=\newline,
postheadhook={\textcolor{black}{\rule[.6ex]{\linewidth}{0.4pt}}\\},
spacebelow= 5pt,
spaceabove= 10pt,
mdframed={
    backgroundcolor=LightOrange,
    innertopmargin=6pt,
    roundcorner=5pt,
    innerbottommargin=6pt,
    skipabove=\parsep,
    skipbelow=\parsep }
]{myCorollaryStyle}

% example environment - thmtools
\declaretheorem[
  style=myCorollaryStyle,
  numberwithin=subsection
]{corolario}

\declaretheoremstyle[
headfont=\bfseries,
notebraces={[}{]},
bodyfont=\normalfont\itshape,
headpunct={},
postheadspace=\newline,
postheadhook={\textcolor{black}{\rule[.6ex]{\linewidth}{0.4pt}}\\},
spacebelow= 5pt,
spaceabove= 10pt,
mdframed={
    backgroundcolor=LightOrange,
    innertopmargin=6pt,
    roundcorner=5pt,
    innerbottommargin=6pt,
    skipabove=\parsep,
    skipbelow=\parsep }
]{myLemmaStyle}

% example environment - thmtools
\declaretheorem[
  style=myLemmaStyle,
  numberwithin=subsection
]{lema}

\declaretheoremstyle[
headfont=\bfseries,
notebraces={[}{]},
bodyfont=\normalfont\itshape,
headpunct={},
postheadspace=\newline,
postheadhook={\textcolor{black}{\rule[.6ex]{\linewidth}{0.4pt}}\\},
spacebelow= 5pt,
spaceabove= 10pt,
mdframed={
    backgroundcolor=LightBlue,
    innertopmargin=6pt,
    roundcorner=5pt,
    innerbottommargin=6pt,
    skipabove=\parsep,
    skipbelow=\parsep }
]{myDefinitionStyle}

% example environment - thmtools
\declaretheorem[
  style=myDefinitionStyle,
  numberwithin=subsection
]{definición}

\declaretheoremstyle[
headfont=\bfseries,
notebraces={[}{]},
bodyfont=\normalfont\itshape,
headpunct={},
postheadspace=\newline,
postheadhook={\textcolor{black}{\rule[.6ex]{\linewidth}{0.4pt}}\\},
spacebelow= 5pt,
spaceabove= 10pt,
mdframed={
    backgroundcolor=LightGreen,
    innertopmargin=6pt,
    roundcorner=5pt,
    innerbottommargin=6pt,
    skipabove=\parsep,
    skipbelow=\parsep }
]{myObservationStyle}

% example environment - thmtools
\declaretheorem[
  style=myObservationStyle,
  numberwithin=subsection
]{observación}



% theoremstyle shortcut commands
\newcommand{\observacion}[1]{\begin{observación}#1\end{observación}}
\newcommand{\definicion}[1]{\begin{definición}#1\end{definición}}
\newcommand{\proposicion}[1]{\begin{proposición}#1\end{proposición}}
\newcommand{\teo}[1]{\begin{teorema}#1\end{teorema}}
\newcommand{\lem}[1]{\begin{lema}#1\end{lema}}
\newcommand{\cor}[1]{\begin{corolario}#1\end{corolario}}
\newcommand{\obs}[1]{\begin{observación}#1\end{observación}}
\newcommand{\defi}[1]{\begin{definición}#1\end{definición}}
\newcommand{\prop}[1]{\begin{proposición}#1\end{proposición}}

%TOC: Fill Table of Contents with Dots
\usepackage{tocloft} %Fill ToC with dotted line
\renewcommand{\cftpartleader}{\cftdotfill{\cftdotsep}} % for parts
\renewcommand{\cftsecleader}{\cftdotfill{\cftdotsep}} % for sections

% Create a new command
\newcommand{\hr}{\centerline{\rule{3.5in}{1pt}}}
%\colorbox[HTML]{e4e4e4}{\makebox[\textwidth-2\fboxsep][l]{texto}

% Create a new command to write text in between two lines
\newcommand{\nc}[2][]{%
  \tikz \draw [draw=black, ultra thick, #1]
  ($(current page.center)-(0.5\linewidth,0)$) --
  ($(current page.center)+(0.5\linewidth,0)$)
  node [midway, fill=white] {#2};
}% tomado de https://tex.stackexchange.com/questions/179425/a-new-command-of-the-form-tex

\newtcolorbox{mybox}[2][]{%
  colback=white, % Box background color
  colframe=black, % Border color
  colbacktitle=black, % Title background color
  coltitle=white, % Title text color
  fonttitle=\bfseries, % Bold title text
  enhanced, % Enables better rendering
  boxrule=0.5pt, % Border width
  left=10pt, right=10pt, top=10pt, bottom=10pt, % Padding inside box
  breakable, % Allows the box to continue across pages
  overlay first ={ % Title always appears but only at the top of the first part
      \node[draw=blue!50!black, thick,
        fill=black, rounded corners=1mm, yshift=0pt, xshift=1cm, left,
        text=white, anchor=west, font=\bfseries] at (frame.north west) {#2};
    },
  overlay unbroken ={ % Title appears if the box is unbroken
      \node[draw=blue!50!black, thick,
        fill=black, rounded corners=1mm, yshift=0pt, xshift=1cm, left,
        text=white, anchor=west, font=\bfseries] at (frame.north west) {#2};
    },
  #1
}

\newcommand{\ejemplo}[1]{%
  \begin{mybox}{Ejemplo}
    #1
  \end{mybox}
}

\addto\captionsenglish{\renewcommand{\proofname}{Demostración}}

% Define \dem to take an argument and wrap it in proof environment
\newcommand{\dem}[1]{\begin{proof}#1\end{proof}}


\usepackage{hyperref}
\usepackage{cleveref}
% Configure cleveref to recognize your custom theorem styles
\crefname{teorema}{Teorema}{Teoremas}
\crefname{proposición}{Proposición}{Proposiciones}
\crefname{definición}{Definición}{Definiciones}
\crefname{lema}{Lema}{Lemas}
\crefname{corolario}{Corolario}{Corolarios}
\crefname{observación}{Observación}{Observaciones}