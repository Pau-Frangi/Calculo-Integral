\usepackage[utf8]{inputenc}

%Colors:
\usepackage{xcolor}
\usepackage{colortbl}

%Math
\usepackage[makeroom]{cancel}%cancel lines math
\usepackage{dsfont}%fonts
\usepackage{url}
\usepackage{multicol}
\usepackage{amsmath}
\usepackage{esint}
\usepackage{bigints}
\usepackage{amsfonts}
\usepackage{amsmath,xparse}%matrices
\usepackage{amssymb}%math symbols
\usepackage{braket}
\usepackage{mathtools}
\usepackage{amsmath,amssymb}
\usepackage{enumitem}
\usepackage{xhfill}
\usepackage{tikz}
\usetikzlibrary{calc}
\usetikzlibrary{decorations.pathmorphing}
\newcommand{\s}{\text{ }}%Space in math mode
\newcommand{\trace}{\text{trace}\,}%Trace command
\newcommand{\rank}{\text{rank}\,}%Rank command
\newcommand{\R}{\mathbb{R}}%Real Numbers
\newcommand{\Z}{\mathbb{Z}}%Integers
\newcommand{\Q}{\mathbb{Q}}%Rational Numbers
\newcommand{\Hbt}{\mathcal{H}}%Hilbert Space
\newcommand{\F}{\mathcal{F}}%Hilbert Space
\newcommand{\C}{\mathbb{C}}%Complex Numbers
\newcommand{\N}{\mathbb{N}}%Natural Numbers
\newcommand{\Imat}{\mathds{1}}%Identity Matrix


\newcommand{\uni}{\bigcup_{i=1}^{\infty}}%Union i=1 to infinity
\newcommand{\unj}{\bigcup_{j=1}^{\infty}}%Union j=1 to infinity
\newcommand{\unk}{\bigcup_{k=1}^{\infty}}%Union k=1 to infinity
\newcommand{\inti}{\bigcap_{i=1}^{\infty}}%Intersection i=1 to infinity
\newcommand{\intj}{\bigcap_{j=1}^{\infty}}%Intersection j=1 to infinity
\newcommand{\intk}{\bigcap_{k=1}^{\infty}}%Intersection k=1 to infinity
\newcommand{\si}{\sum_{i=1}^{\infty}}%Sum i=1 to infinity
\newcommand{\sj}{\sum_{j=1}^{\infty}}%Sum j=1 to infinity
\newcommand{\sk}{\sum_{k=1}^{\infty}}%Sum k=1 to infinity

\newcommand{\rn} {\mathbb{R}^n}%R^n
\newcommand{\inr} {\in \mathbb{R}} %in R
\newcommand{\inrn} {\in \mathbb{R}^n} %in R^n
\newcommand{\vol} {\text{vol}} %Volume




% \newcommand{\ket}[1]{\left|#1\right\rangle}%Prints |#1⟩
% \newcommand{\bra}[1]{\left\langle#1\right|}%Prints ⟨#1|
% \newcommand{\braket}[2]{\langle#1|#2\rangle}%Prints ⟨#1|#2⟩
% \newcommand{\Braket}[2]{\left\langle#1\left|#2\right.\right\rangle}%Prints ⟨#1|#2⟩
% \newcommand{\braopket}[3]{\left\langle#1\left|#2\left|#3\right.\right.\right\rangle}%Prints ⟨#1|#2|#3⟩
\newcommand{\expected}[1]{\left\langle#1\right\rangle}%Prints ⟨#1⟩
\newcommand{\naturalset}[2]{#1=1,\;2,\;...,\;#2}%Prints #1= #1= 1, 2, ..., #2 in math mode
\newcommand{\vectorset}[3]{#1=\{#2_1,\;#2_2,\;...,\;#2_{#3}\}}%Prints #1= {#2_1,#2_2, ..., #2_{#3}} in math mode
\newcommand{\diagmatrix}[9]{%
\begin{pmatrix}
\lambda_1&1&0&0&0\\0&\lambda_1&0&0&0\\0&0&\lambda_3&1&0\\0&0&0&\lambda_4&1\\0&0&0&0&\lambda_4
\end{pmatrix}
}%Matrix with diagonal entries specified

% Setup matha font (mathabx.sty) for extracting specific symbols:
\DeclareFontFamily{U}{matha}{\hyphenchar\font45}
\DeclareFontShape{U}{matha}{m}{n}{
      <5> <6> <7> <8> <9> <10> gen * matha
      <10.95> matha10 <12> <14.4> <17.28> <20.74> <24.88> matha12
      }{}
\DeclareSymbolFont{matha}{U}{matha}{m}{n}
% Define a plus/minus character from that font (from mathabx.dcl):
\DeclareMathSymbol{\PM}{2}{matha}{"08}% to completely replace the \pm character, replace \PM with \pm
\input{Preamble/Diagonal Matrix Command}

%Page Margins:
\usepackage[a4paper, total={7in, 10in}]{geometry}

%paragraph Indent and Spacing
\setlength{\parindent}{0em}
\setlength{\parskip}{0.5em}

%Images:
\usepackage{graphicx}
\graphicspath{ {./images/} }
\usepackage{eso-pic}
\usepackage{wrapfig}%Wrapping Functionality
\usepackage{float}
\usepackage{chngcntr}% Count Figures as below v v v
\counterwithout{figure}{section}%          CONTINUOUS
\counterwithout{table}{section}%           CONTINUOUS

%Enumerate
\usepackage{enumitem} 

%Language
\usepackage[english]{babel}

%Environments
\usepackage{amsthm, amsfonts}
\usepackage{thmtools}
\usepackage{float}
\usepackage{framed}
\usepackage{tcolorbox}

\colorlet{LightGray}{gray!15}
\colorlet{LightOrange}{orange!15}
\colorlet{LightGreen}{green!15}
\colorlet{LightBlue}{blue!15}
\colorlet{LightRed}{red!15}

\newcommand{\HRule}[1]{\rule{\linewidth}{#1}}

\usepackage[framemethod=TikZ]{mdframed}
\usepackage{amsthm}

\newtheorem{theorem}{Theorem}[subsection] % Numbered as "Teorema 1.1.1" in Section 1, Subsection 1
\numberwithin{theorem}{subsection} % Ensure numbering within subsections

% Apply styles to the theorem
\declaretheoremstyle[
preheadhook={\begin{mdframed}[backgroundcolor=LightRed, 
  innertopmargin=12pt, innerbottommargin=10pt,
  skipbelow=0pt, skipabove=10pt, 
  roundcorner=5pt]},
postfoothook=\end{mdframed}
]{thrmstyle}
\declaretheorem[style=thrmstyle, sibling=theorem]{teorema} % Use 'sibling' to share numbering with 'theorem'

\newtheorem{proposition}{Proposition}[subsection]
\numberwithin{proposition}{subsection} 

\declaretheoremstyle[
preheadhook={\begin{mdframed}[backgroundcolor=LightOrange, 
  innertopmargin=12pt, innerbottommargin=10pt,
  skipbelow=0pt, skipabove=10pt, 
  roundcorner=5pt]},
postfoothook=\end{mdframed}
]{thrmstyle}
\declaretheorem[style=thrmstyle, sibling=proposition]{proposición} % Use 'sibling' to share numbering with 'theorem'

\newtheorem{corolary}{Corolary}[subsection]
\numberwithin{corolary}{subsection} 

\declaretheoremstyle[
preheadhook={\begin{mdframed}[backgroundcolor=LightOrange, 
  innertopmargin=12pt, innerbottommargin=10pt,
  skipbelow=0pt, skipabove=10pt, 
  roundcorner=5pt]},
postfoothook=\end{mdframed}
]{thrmstyle}
\declaretheorem[style=thrmstyle, sibling=corolary]{corolario} % Use 'sibling' to share numbering with 'theorem'

\newtheorem{lemma}{Lemma}[subsection]
\numberwithin{lemma}{subsection} 

\declaretheoremstyle[
preheadhook={\begin{mdframed}[backgroundcolor=LightOrange, 
  innertopmargin=12pt, innerbottommargin=10pt,
  skipbelow=0pt, skipabove=10pt, 
  roundcorner=5pt]},
postfoothook=\end{mdframed}
]{thrmstyle}
\declaretheorem[style=thrmstyle, sibling=lemma]{lema} % Use 'sibling' to share numbering with 'theorem'

\newtheorem{definition}{Defintion}[subsection]
\numberwithin{definition}{subsection} 

\declaretheoremstyle[
preheadhook={\begin{mdframed}[backgroundcolor=LightBlue, 
  innertopmargin=12pt, innerbottommargin=10pt,
  skipbelow=0pt, skipabove=10pt, 
  roundcorner=5pt]},
postfoothook=\end{mdframed}
]{thrmstyle}
\declaretheorem[style=thrmstyle, sibling=definition]{definición} % Use 'sibling' to share numbering with 'theorem'

\newtheorem{observation}{Observation}[subsection]
\numberwithin{observation}{subsection} 

\declaretheoremstyle[
preheadhook={\begin{mdframed}[backgroundcolor=LightGreen, 
  innertopmargin=12pt, innerbottommargin=10pt,
  skipbelow=0pt, skipabove=10pt, 
  roundcorner=5pt]},
postfoothook=\end{mdframed}
]{thrmstyle}
\declaretheorem[style=thrmstyle, sibling=observation]{observación} % Use 'sibling' to share numbering with 'theorem'


% theoremstyle shortcut commands
\newcommand{\observacion}[1]{\begin{observación}#1\end{observación}}
\newcommand{\definicion}[1]{\begin{definición}#1\end{definición}}
\newcommand{\proposicion}[1]{\begin{proposición}#1\end{proposición}}
\newcommand{\teo}[1]{\begin{teorema}#1\end{teorema}}
\newcommand{\lem}[1]{\begin{lema}#1\end{lema}}
\newcommand{\cor}[1]{\begin{corolario}#1\end{corolario}}
\newcommand{\obs}[1]{\begin{observación}#1\end{observación}}
\newcommand{\defi}[1]{\begin{definición}#1\end{definición}}
\newcommand{\prop}[1]{\begin{proposición}#1\end{proposición}}



%TOC: Fill Table of Contents with Dots
\usepackage{tocloft} %Fill ToC with dotted line
\renewcommand{\cftpartleader}{\cftdotfill{\cftdotsep}} % for parts
\renewcommand{\cftsecleader}{\cftdotfill{\cftdotsep}} % for sections

% Create a new command
\newcommand{\hr}{\centerline{\rule{3.5in}{1pt}}}
%\colorbox[HTML]{e4e4e4}{\makebox[\textwidth-2\fboxsep][l]{texto}


% Create a new command to write text in between two lines
\newcommand{\nc}[2][]{%
\tikz \draw [draw=black, ultra thick, #1]
    ($(current page.center)-(0.5\linewidth,0)$) -- 
    ($(current page.center)+(0.5\linewidth,0)$)
    node [midway, fill=white] {#2};
}% tomado de https://tex.stackexchange.com/questions/179425/a-new-command-of-the-form-tex


\newcommand{\createbox}[2]{
    \begin{tikzpicture}
    \node [mybox, align=left, text width=\textwidth-20pt] (box){% Adjust text width to stay within bounds
        #2
    };
    \node[fancytitle, right=10pt] at (box.north west) {#1};
    \end{tikzpicture}
    % Remove some space between the boxes
    \vspace{-0.35cm}
}

% TikZ styles
\tikzstyle{mybox} = [draw=black, fill=white,
    rectangle, rounded corners, inner sep=10pt, inner ysep=10pt]
\tikzstyle{fancytitle} = [fill=black, text=white, font=\bfseries, rounded corners]


\newcommand{\ejemplo}[1]{\createbox{Ejemplo}{#1}}

\addto\captionsenglish{\renewcommand{\proofname}{Demostración}}

% Define \dem to take an argument and wrap it in proof environment
\newcommand{\dem}[1]{\begin{proof}#1\end{proof}}



%Hyperlinks
\usepackage{hyperref}

